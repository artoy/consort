
\section{Proof of \Cref{lem:preservation}}
\label{sec:preservation-proof}

We first prove two additional lemmas. 
\Cref{lem:stack-well-typed,lem:callfunc} give key facts used in the return and
call cases respectively; we have separated them into separate lemmas for clarity.

\begin{lemma} % L25
  \label{lem:stack-well-typed}
  For any $\Gamma_{{\mathrm{0}}}$ such that $ \Theta   \mid   \ell  \ottsym{:}  \oldvec{\ell}   \mid   \Gamma_{{\mathrm{0}}}   \vdash   \mathit{x}  :  \tau_{{\mathrm{1}}}   \produces   \Gamma_{{\mathrm{1}}} $ and $\Theta  \mid  \HOLE  \ottsym{:}  \tau_{{\mathrm{1}}}  \produces  \Gamma_{{\mathrm{1}}}  \mid  \oldvec{\ell}  \vdash_{\mathit{ectx} }   \ottnt{E} [\LET  \mathit{y}  =   \HOLE^ \ell   \IN  \ottnt{e}  ]   \ottsym{:}  \tau_{{\mathrm{2}}}  \produces  \Gamma_{{\mathrm{2}}}$ then
  $ \Theta   \mid   \oldvec{\ell}   \mid   \Gamma_{{\mathrm{0}}}   \vdash    \ottnt{E} [\LET  \mathit{y}  =   \HOLE^ \ell   \IN  \ottnt{e}  ]   \ottsym{[}  \mathit{x}  \ottsym{]}  :  \tau_{{\mathrm{2}}}   \produces   \Gamma_{{\mathrm{2}}} $.
\end{lemma}
\begin{proof}
  It suffices to show that $ \Theta   \mid   \oldvec{\ell}   \mid   \Gamma_{{\mathrm{0}}}   \vdash    \LET  \mathit{y}  =  \mathit{x}  \IN  \ottnt{e}   :  \tau'_{{\mathrm{1}}}   \produces   \Gamma'_{{\mathrm{1}}} $
  and $\Theta  \mid  \HOLE  \ottsym{:}  \tau'_{{\mathrm{1}}}  \produces  \Gamma'_{{\mathrm{1}}}  \mid  \oldvec{\ell}  \vdash_{\mathit{ectx} }  \ottnt{E}  \ottsym{:}  \tau_{{\mathrm{2}}}  \produces  \Gamma_{{\mathrm{2}}}$ for some $\tau'_{{\mathrm{1}}}$ and $\Gamma'_{{\mathrm{1}}}$
  whereby the result will hold from \Cref{lem:ectxt-sub-well-typed}.

  By inversion on
  $\Theta  \mid  \HOLE  \ottsym{:}  \tau_{{\mathrm{1}}}  \produces  \Gamma_{{\mathrm{1}}}  \mid  \oldvec{\ell}  \vdash_{\mathit{ectx} }   \ottnt{E} [\LET  \mathit{y}  =   \HOLE^ \ell   \IN  \ottnt{e}  ]   \ottsym{:}  \tau_{{\mathrm{2}}}  \produces  \Gamma_{{\mathrm{2}}}$ we have
  \begin{align}
    &  \Theta   \mid   \oldvec{\ell}   \mid   \Gamma_{{\mathrm{1}}}  \ottsym{,}  \mathit{y}  \ottsym{:}  \tau_{{\mathrm{1}}}   \vdash   \ottnt{e}  :  \tau''_{{\mathrm{1}}}   \produces   \Gamma''_{{\mathrm{1}}}  \label{eqn:let-body-ty} \\
    & \Theta  \mid  \HOLE  \ottsym{:}  \tau''_{{\mathrm{1}}}  \produces  \Gamma''_{{\mathrm{1}}}  \mid  \mathcal{L}  \vdash_{\mathit{ectx} }  \ottnt{E}  \ottsym{:}  \tau_{{\mathrm{2}}}  \produces  \Gamma_{{\mathrm{2}}} \label{eqn:ctxt-typed} \\
    &  \mathit{y}  \not\in   \DOM( \Gamma''_{{\mathrm{1}}} )   \label{eqn:y-not-free-var}
  \end{align}
  We take $\Gamma'_{{\mathrm{1}}}  \ottsym{=}  \Gamma''_{{\mathrm{1}}}$, $\tau'_{{\mathrm{1}}}  \ottsym{=}  \tau''_{{\mathrm{1}}}$, and then \Cref{eqn:ctxt-typed}
  gives the necessary typing for $\ottnt{E}$.

  It remains to to show that
  \[
     \Theta   \mid   \oldvec{\ell}   \mid   \Gamma_{{\mathrm{0}}}   \vdash    \LET  \mathit{y}  =  \mathit{x}  \IN  \ottnt{e}   :  \tau'_{{\mathrm{1}}}   \produces   \Gamma'_{{\mathrm{1}}} 
  \]
  (That $ \oldvec{\ell}   \vdash _{\wf}  \tau'_{{\mathrm{1}}}   \produces   \Gamma'_{{\mathrm{1}}} $ follows from \Cref{eqn:let-body-ty})
  
  By \Cref{lem:inversion}, from $ \Theta   \mid   \ell  \ottsym{:}  \oldvec{\ell}   \mid   \Gamma_{{\mathrm{0}}}   \vdash   \mathit{x}  :  \tau_{{\mathrm{1}}}   \produces   \Gamma_{{\mathrm{1}}} $ we conclude there
  exists some $\Gamma_{\ottmv{p}}$, $\tau_{\ottmv{p}}$, and $\Gamma'_{\ottmv{p}}$ such:
  \begin{align}
    & \Gamma_{{\mathrm{0}}}  \leq  \Gamma_{\ottmv{p}} \label{eqn:rin-sub} \\
    & \Gamma'_{\ottmv{p}}  \ottsym{,}  \tau_{\ottmv{p}}  \leq  \Gamma_{{\mathrm{1}}}  \ottsym{,}  \tau_{{\mathrm{1}}} \label{eqn:rout-sub} \\
    & \Gamma'_{\ottmv{p}}  \ottsym{=}  \Gamma_{\ottmv{p}}  \ottsym{[}  \mathit{x}  \hookleftarrow  \tau'_{\ottmv{p}}  \ottsym{]} \label{eqn:gp-up-def} \\
    & \Gamma_{\ottmv{p}}  \ottsym{(}  \mathit{x}  \ottsym{)}  \ottsym{=}  \tau_{\ottmv{p}}  \ottsym{+}  \tau'_{\ottmv{p}} \label{eqn:gp-x-def} \\
    &  \oldvec{\ell}   \vdash _{\wf}  \Gamma_{\ottmv{p}}  \label{eqn:renv-sub-in-wf}
  \end{align} 

  We first apply \rn{T-Sub} with \Cref{eqn:rin-sub,eqn:renv-sub-in-wf}, so it remains to show
  \[
     \Theta   \mid   \oldvec{\ell}   \mid   \Gamma_{\ottmv{p}}  \ottsym{[}  \mathit{x}  \ottsym{:}  \tau_{\ottmv{p}}  \ottsym{+}  \tau'_{\ottmv{p}}  \ottsym{]}   \vdash    \LET  \mathit{y}  =  \mathit{x}  \IN  \ottnt{e}   :  \tau'_{{\mathrm{1}}}   \produces   \Gamma'_{{\mathrm{1}}} 
  \]
  which, by \rn{T-Let} holds if we show that:
  \[
     \Theta   \mid   \oldvec{\ell}   \mid   \Gamma_{\ottmv{p}}  \ottsym{[}  \mathit{x}  \hookleftarrow   \tau'_{\ottmv{p}}  \wedge_{ \mathit{x} }   \mathit{x}  =_{ \tau'_{\ottmv{p}} }  \mathit{y}    \ottsym{]}  \ottsym{,}  \mathit{y}  \ottsym{:}   \tau_{\ottmv{p}}  \wedge_{ \mathit{y} }   \mathit{y}  =_{ \tau_{\ottmv{p}} }  \mathit{x}     \vdash   \ottnt{e}  :  \tau'_{{\mathrm{1}}}   \produces   \Gamma'_{{\mathrm{1}}} 
  \]
  ($ \mathit{y}  \not\in   \DOM( \Gamma'_{{\mathrm{1}}} )  $ follows from \Cref{eqn:y-not-free-var}, and the well-formedness of
  $\Gamma_{\ottmv{p}}  \ottsym{[}  \mathit{x}  \hookleftarrow   \tau'_{\ottmv{p}}  \wedge_{ \mathit{x} }   \mathit{x}  =_{ \tau'_{\ottmv{p}} }  \mathit{y}    \ottsym{]}  \ottsym{,}  \mathit{y}  \ottsym{:}   \tau_{\ottmv{p}}  \wedge_{ \mathit{y} }   \mathit{y}  =_{ \tau_{\ottmv{p}} }  \mathit{x}  $ follows from the well-formedness of
  $\Gamma_{\ottmv{p}}$, $\tau_{\ottmv{p}}$ and $\tau'_{\ottmv{p}}$ and that $\mathit{x}$ and $\mathit{y}$ appear in the refinements
  iff they are mapped to integer types in the new type environment.)
    
  We can use \rn{T-Sub} to weaken the type environment to:
  \[
     \Theta   \mid   \oldvec{\ell}   \mid   \Gamma_{\ottmv{p}}  \ottsym{[}  \mathit{x}  \hookleftarrow  \tau'_{\ottmv{p}}  \ottsym{]}  \ottsym{,}  \mathit{y}  \ottsym{:}  \tau_{\ottmv{p}}   \vdash   \ottnt{e}  :  \tau'_{{\mathrm{1}}}   \produces   \Gamma'_{{\mathrm{1}}} 
  \]
  From \Cref{eqn:rout-sub} above, we have that $\Gamma_{\ottmv{p}}  \ottsym{[}  \mathit{x}  \hookleftarrow  \tau'_{\ottmv{p}}  \ottsym{]}  \ottsym{,}  \mathit{y}  \ottsym{:}  \tau_{\ottmv{p}}  \leq  \Gamma_{{\mathrm{1}}}  \ottsym{,}  \mathit{y}  \ottsym{:}  \tau_{{\mathrm{1}}}$,
  whereby one final application of \rn{T-Sub} allows us to use to
  \Cref{eqn:let-body-ty} above.
\end{proof}

\begin{lemma} % L28
  \label{lem:callfunc}
  Let $\ottnt{E}  \ottsym{[}   \LET  \mathit{x}  =   \mathit{f} ^ \ell (  \mathit{y_{{\mathrm{1}}}} ,\ldots, \mathit{y_{\ottmv{n}}}  )   \IN  \ottnt{e'}   \ottsym{]}$ be a term such that:
  
  \begin{bcpcasearray}
     \Theta   \mid   \oldvec{\ell}   \mid   \Gamma_{{\mathrm{0}}}   \vdash    \LET  \mathit{x}  =   \mathit{f} ^ \ell (  \mathit{y_{{\mathrm{1}}}} ,\ldots, \mathit{y_{\ottmv{n}}}  )   \IN  \ottnt{e'}   :  \tau_{{\mathrm{1}}}   \produces   \Gamma_{{\mathrm{1}}}  &  \sigma_{\alpha}  \ottsym{=}  \ottsym{[}  \ell  \ottsym{:}  \oldvec{\ell}  \ottsym{/}  \lambda  \ottsym{]} \\
    \Theta  \mid  \HOLE  \ottsym{:}  \tau_{{\mathrm{1}}}  \produces  \Gamma_{{\mathrm{1}}}  \mid  \oldvec{\ell}  \vdash_{\mathit{ectx} }  \ottnt{E}  \ottsym{:}  \tau_{{\mathrm{2}}}  \produces  \Gamma_{{\mathrm{2}}} & \sigma_{x}  \ottsym{=}    [  \mathit{y_{{\mathrm{1}}}}  /  \mathit{x_{{\mathrm{1}}}}  ]  \cdots  [  \mathit{y_{\ottmv{n}}}  /  \mathit{x_{\ottmv{n}}}  ]   \\
     \mathit{f}  \mapsto  \ottsym{(}  \mathit{x_{{\mathrm{1}}}}  \ottsym{,} \, .. \, \ottsym{,}  \mathit{x_{\ottmv{n}}}  \ottsym{)}  \ottnt{e}  \in  \ottnt{D}  & \Theta  \vdash  \mathit{f}  \mapsto  \ottsym{(}  \mathit{x_{{\mathrm{1}}}}  \ottsym{,} \, .. \, \ottsym{,}  \mathit{x_{\ottmv{n}}}  \ottsym{)}  \ottnt{e} \\
     \vdash _{\wf}  \Theta 
  \end{bcpcasearray}

  where $\Theta  \ottsym{(}  \mathit{f}  \ottsym{)}  \ottsym{=}   \forall  \lambda .\tuple{ \mathit{x_{{\mathrm{1}}}} \COL \tau_{{\mathrm{1}}} ,\dots, \mathit{x_{\ottmv{n}}} \COL \tau_{\ottmv{n}} }\ra\tuple{ \mathit{x_{{\mathrm{1}}}} \COL \tau'_{{\mathrm{1}}} ,\dots, \mathit{x_{\ottmv{n}}} \COL \tau'_{\ottmv{n}}  \mid  \tau_{\ottmv{q}} } $.

  Then there exist some $\tau_{{\mathrm{3}}}$ and $\Gamma_{{\mathrm{3}}}$:
  \begin{align*}
    &  \Theta   \mid   \ell  \ottsym{:}  \oldvec{\ell}   \mid   \Gamma_{{\mathrm{0}}}   \vdash    \sigma_{x}   \ottnt{e}   :  \tau_{{\mathrm{3}}}   \produces   \Gamma_{{\mathrm{3}}}  \\
    & \Theta  \mid  \HOLE  \ottsym{:}  \tau_{{\mathrm{3}}}  \produces  \Gamma_{{\mathrm{3}}}  \mid  \oldvec{\ell}  \vdash_{\mathit{ectx} }   \ottnt{E} [\LET  \mathit{x}  =   \HOLE^ \ell   \IN  \ottnt{e}  ]   \ottsym{:}  \tau_{{\mathrm{2}}}  \produces  \Gamma_{{\mathrm{2}}}
  \end{align*}
\end{lemma}
\begin{proof}
  From \Cref{lem:inversion} on
  $ \Theta   \mid   \oldvec{\ell}   \mid   \Gamma_{{\mathrm{0}}}   \vdash    \LET  \mathit{x}  =   \mathit{f} ^ \ell (  \mathit{y_{{\mathrm{1}}}} ,\ldots, \mathit{y_{\ottmv{n}}}  )   \IN  \ottnt{e'}   :  \tau_{{\mathrm{1}}}   \produces   \Gamma_{{\mathrm{1}}} $ we have, for some
  $\Gamma_{\ottmv{p}}, \tau_{\ottmv{p}}, \Gamma'_{\ottmv{p}}$, that:
  \begin{align}
    & \Gamma_{{\mathrm{0}}}  \leq  \Gamma_{\ottmv{p}} \label{eqn:in-sub} \\
    & \Gamma'_{\ottmv{p}}  \ottsym{,}  \tau_{\ottmv{p}}  \leq  \Gamma_{{\mathrm{1}}}  \ottsym{,}  \tau_{{\mathrm{1}}} \label{eqn:out-sub} \\
    & \Gamma_{\ottmv{p}}  \ottsym{(}  \mathit{y_{\ottmv{i}}}  \ottsym{)}  \ottsym{=}  \sigma_{\alpha} \, \sigma_{x} \, \tau_{\ottmv{i}} \label{eqn:arg-typed} \\
    &  \oldvec{\ell}   \vdash _{\wf}  \Gamma_{\ottmv{p}}  \label{eqn:sub-in-env-wf} \\
    &  \Theta   \mid   \oldvec{\ell}   \mid   \Gamma_{\ottmv{p}}  \ottsym{[}  \mathit{y_{\ottmv{i}}}  \hookleftarrow  \sigma_{\alpha} \, \sigma_{x} \, \tau'_{\ottmv{i}}  \ottsym{]}  \ottsym{,}  \mathit{x}  \ottsym{:}  \sigma_{\alpha} \, \sigma_{x} \, \tau_{\ottmv{q}}   \vdash   \ottnt{e'}  :  \tau_{\ottmv{p}}   \produces   \Gamma'_{\ottmv{p}}  \label{eqn:let-body-well-typed} \\
    &  \mathit{x}  \not\in   \DOM( \Gamma'_{\ottmv{p}} )   \label{eqn:cbind-erased}
  \end{align}
  
  To prove the first part, from the well-typing of the function body, we have
  $ \Theta   \mid   \lambda   \mid    \mathit{x_{{\mathrm{1}}}} \COL \tau_{{\mathrm{1}}} ,\ldots, \mathit{x_{\ottmv{n}}} \COL \tau_{\ottmv{n}}    \vdash   \ottnt{e}  :  \tau_{\ottmv{q}}   \produces    \mathit{x_{{\mathrm{1}}}} \COL \tau'_{{\mathrm{1}}} ,\ldots, \mathit{x_{\ottmv{n}}} \COL \tau'_{\ottmv{n}}  $.
  From our assumption that all variable names are distinct,
  by $n$ applications of the substitution lemma (\Cref{lem:substitution}) we have:
  $ \Theta   \mid   \lambda   \mid    \mathit{y_{{\mathrm{1}}}} \COL \sigma_{x} \, \tau_{{\mathrm{1}}} ,\ldots, \mathit{y_{\ottmv{n}}} \COL \sigma_{x} \, \tau_{\ottmv{n}}    \vdash    \sigma_{x}   \ottnt{e}   :  \sigma_{x} \, \tau_{\ottmv{q}}   \produces    \mathit{y_{{\mathrm{1}}}} \COL \sigma_{x} \, \tau'_{{\mathrm{1}}} ,\ldots, \mathit{y_{\ottmv{n}}} \COL \sigma_{x} \, \tau'_{\ottmv{n}}  $.
  By \subref{lem:ctxt-substitution}{itm:ctxt-subst-well-typed} we then have
  $ \Theta   \mid   \ell  \ottsym{:}  \oldvec{\ell}   \mid    \mathit{y_{{\mathrm{1}}}} \COL \sigma_{\alpha} \, \sigma_{x} \, \tau_{{\mathrm{1}}} ,\ldots, \mathit{y_{\ottmv{n}}} \COL \sigma_{\alpha} \, \sigma_{x} \, \tau_{\ottmv{n}}    \vdash    \sigma_{x}   \ottnt{e}   :  \sigma_{\alpha} \, \sigma_{x} \, \tau_{\ottmv{q}}   \produces    \mathit{y_{{\mathrm{1}}}} \COL \sigma_{\alpha} \, \sigma_{x} \, \tau'_{{\mathrm{1}}} ,\ldots, \mathit{y_{\ottmv{n}}} \COL \sigma_{\alpha} \, \sigma_{x} \, \tau'_{\ottmv{n}}  $.
  We take $\tau_{{\mathrm{3}}}  \ottsym{=}  \sigma_{\alpha} \, \sigma_{x} \, \tau_{\ottmv{q}}$ and $\Gamma_{{\mathrm{3}}}  \ottsym{=}  \Gamma_{\ottmv{p}}  \ottsym{[}  \mathit{y_{\ottmv{i}}}  \hookleftarrow  \sigma_{\alpha} \, \sigma_{x} \, \tau'_{\ottmv{i}}  \ottsym{]}$.
  
  By the well-formedness of function types and well-formedness of $\Gamma_{\ottmv{p}}$,
  we must have that $ \ell  \ottsym{:}  \oldvec{\ell}   \vdash _{\wf}  \Gamma_{{\mathrm{3}}} $.
  Then by \Cref{eqn:arg-typed,eqn:sub-in-env-wf,lem:tyenv-weaken} we have
  $ \Theta   \mid   \ell  \ottsym{:}  \oldvec{\ell}   \mid   \Gamma_{\ottmv{p}}   \vdash    \sigma_{x}   \ottnt{e}   :  \tau_{{\mathrm{3}}}   \produces   \Gamma_{{\mathrm{3}}} $, whereby \Cref{eqn:in-sub}
  and an application of \rn{T-Sub}
  gives $ \Theta   \mid   \ell  \ottsym{:}  \oldvec{\ell}   \mid   \Gamma_{{\mathrm{0}}}   \vdash    \sigma_{x}   \ottnt{e}   :  \tau_{{\mathrm{3}}}   \produces   \Gamma_{{\mathrm{3}}} $, i.e., the first result.

  To prove the second part, from the typing rule for \rn{TE-Stack} we must show:
  \begin{align}
    & \Theta  \mid  \HOLE  \ottsym{:}  \tau_{{\mathrm{1}}}  \produces  \Gamma_{{\mathrm{1}}}  \mid  \oldvec{\ell}  \vdash_{\mathit{ectx} }  \ottnt{E}  \ottsym{:}  \tau_{{\mathrm{2}}}  \produces  \Gamma_{{\mathrm{2}}} \label{eqn:context-well-typed} \\
    &  \mathit{x}  \not\in   \DOM( \Gamma_{{\mathrm{1}}} )   \label{eqn:x-not-free} \\
    &  \Theta   \mid   \oldvec{\ell}   \mid   \Gamma_{{\mathrm{3}}}  \ottsym{,}  \mathit{x}  \ottsym{:}  \tau_{{\mathrm{3}}}   \vdash   \ottnt{e'}  :  \tau_{{\mathrm{1}}}   \produces   \Gamma_{{\mathrm{1}}}  \label{eqn:let-body-sub-typed}
  \end{align}
  \Cref{eqn:context-well-typed} holds by assumption, and \Cref{eqn:x-not-free} follows from
  \Cref{eqn:cbind-erased} and that $\Gamma'_{\ottmv{p}}  \leq  \Gamma_{{\mathrm{1}}}$ implies $  \DOM( \Gamma_{{\mathrm{1}}} )   \subseteq   \DOM( \Gamma'_{\ottmv{p}} )  $.
  $ \oldvec{\ell}   \vdash _{\wf}  \tau_{{\mathrm{1}}}   \produces   \Gamma_{{\mathrm{1}}} $ follows from the well-typing of the function call term,
  and $ \oldvec{\ell}   \vdash _{\wf}  \Gamma_{{\mathrm{3}}}  \ottsym{,}  \mathit{x}  \ottsym{:}  \tau_{{\mathrm{3}}} $ follows from \Cref{eqn:let-body-well-typed}.

  From \Cref{eqn:out-sub,eqn:let-body-well-typed} we then have
  \Cref{eqn:let-body-sub-typed} via an application of \rn{T-Sub}.
\end{proof}

\begin{proof}[Preservation; \Cref{lem:preservation}]
  The proof is organized by cases analysis on the transition rule used of $\ottnt{e}$, and showing that the output configuration is well typed by
  $ \vdash_{\mathit{conf} }^D $. We must therefore find a $\Gamma''$ that is consistent with $\ottnt{H'}$ and $\ottnt{R'}$ and also satisfies the other conditions imposed by
  the definition of $ \vdash_{\mathit{conf} }^D $. Here $\Gamma'',\ottnt{H'}, \ottnt{R'}$ represent the type environment, heap and register after the transition respectively.
  We identify the heap and register file before transition with $\ottnt{H}$ and $\ottnt{R}$ respectively.
  In order to show that the ownership invariant is preserved, we need to prove that $\forall \,  \ottmv{a}  \in \DOM( \ottnt{H} )   \ottsym{.}  \ottkw{Own} \, \ottsym{(}  \ottnt{H}  \ottsym{,}  \ottnt{R}  \ottsym{,}  \Gamma''  \ottsym{)}  \ottsym{(}  \ottmv{a}  \ottsym{)}  \le  \ottsym{1}$.
  In many cases, we will show that $\ottkw{Own} \, \ottsym{(}  \ottnt{H}  \ottsym{,}  \ottnt{R}  \ottsym{,}  \Gamma  \ottsym{)}  \ottsym{=}  \ottkw{Own} \, \ottsym{(}  \ottnt{H'}  \ottsym{,}  \ottnt{R'}  \ottsym{,}  \Gamma''  \ottsym{)}$, whereby
  from the assumption that $\ottkw{Cons} \, \ottsym{(}  \ottnt{H}  \ottsym{,}  \ottnt{R}  \ottsym{,}  \Gamma  \ottsym{)}$ as implied by $ \vdash_{\mathit{conf} }^D $
  we have $\forall \,  \ottmv{a}  \in \DOM( \ottnt{H} )   \ottsym{.}  \ottkw{Own} \, \ottsym{(}  \ottnt{H}  \ottsym{,}  \ottnt{R}  \ottsym{,}  \Gamma  \ottsym{)}  \ottsym{(}  \ottmv{a}  \ottsym{)}  \le  \ottsym{1}$, giving the desired result.
  % FIXED
  \begin{rneqncase}{R-Var}{
     \vdash_{\mathit{conf} }^D   \tuple{ \ottnt{H} ,  \ottnt{R} ,  F_{{\ottmv{n}-1}}  \ottsym{:}  \oldvec{F} ,  \mathit{x} }  ,   \tuple{ \ottnt{H} ,  \ottnt{R} ,  F_{{\ottmv{n}-1}}  \ottsym{:}  \oldvec{F} ,  \mathit{x} }     \longrightarrow _{ \ottnt{D} }     \tuple{ \ottnt{H} ,  \ottnt{R} ,  \oldvec{F} ,  F_{{\ottmv{n}-1}}  \ottsym{[}  \mathit{x}  \ottsym{]} }   \\
  }
  By inversion on configuration typing $ \vdash_{\mathit{conf} }^D   \tuple{ \ottnt{H} ,  \ottnt{R} ,  F_{{\ottmv{n}-1}}  \ottsym{:}  \oldvec{F} ,  \mathit{x} }  $, we have:
    \begin{align*}
      &  \Theta   \mid   \oldvec{\ell}   \mid   \Gamma   \vdash   \mathit{x}  :  \tau_{\ottmv{n}}   \produces   \Gamma_{\ottmv{n}}  \\
      & \forall i\in\set{1..n}.\Theta  \mid  \HOLE  \ottsym{:}  \tau_{\ottmv{i}}  \produces  \Gamma_{\ottmv{i}}  \mid  \oldvec{\ell}_{{\ottmv{i}-1}}  \vdash_{\mathit{ectx} }  F_{{\ottmv{i}-1}}  \ottsym{:}  \tau_{{\ottmv{i}-1}}  \produces  \Gamma_{{\ottmv{i}-1}}
    \end{align*}
    
    Using \Cref{lem:stack-well-typed}, we can conclude that $ \Theta   \mid   \oldvec{\ell}_{{\ottmv{n}-1}}   \mid   \Gamma   \vdash   F_{{\ottmv{n}-1}}  \ottsym{[}  \mathit{x}  \ottsym{]}  :  \tau_{{\ottmv{n}-1}}   \produces   \Gamma_{{\ottmv{n}-1}} $. We therefore take $\Gamma''  \ottsym{=}  \Gamma$.

    It remains to show that $\ottkw{Cons} \, \ottsym{(}  \ottnt{H}  \ottsym{,}  \ottnt{R}  \ottsym{,}  \Gamma''  \ottsym{)}$ which follows immediately from
    $\ottkw{Cons} \, \ottsym{(}  \ottnt{H}  \ottsym{,}  \ottnt{R}  \ottsym{,}  \Gamma  \ottsym{)}$.
  \end{rneqncase}

  % FIXED
  \begin{rneqncase}{R-Deref}{
       \vdash_{\mathit{conf} }^D   \tuple{ \ottnt{H} ,  \ottnt{R} ,  \oldvec{F} ,  \ottnt{E}  \ottsym{[}   \LET  \mathit{x}  =   *  \mathit{y}   \IN  \ottnt{e}   \ottsym{]} }   \\
        \tuple{ \ottnt{H} ,  \ottnt{R} ,  \oldvec{F} ,  \ottnt{E}  \ottsym{[}   \LET  \mathit{x}  =   *  \mathit{y}   \IN  \ottnt{e}   \ottsym{]} }     \longrightarrow _{ \ottnt{D} }     \tuple{ \ottnt{H} ,  \ottnt{R}  \ottsym{\{}  \mathit{x'}  \mapsto  \ottnt{v}  \ottsym{\}} ,  \oldvec{F} ,  \ottnt{E}  \ottsym{[}    [  \mathit{x'}  /  \mathit{x}  ]    \ottnt{e}   \ottsym{]} }   \\
      \ottnt{H}  \ottsym{(}  \ottmv{a}  \ottsym{)} \, \ottsym{=} \, \ottnt{v} \andalso \ottnt{R}  \ottsym{(}  \mathit{y}  \ottsym{)} \, \ottsym{=} \, \ottmv{a} \andalso \ottnt{R'}  \ottsym{=}  \ottnt{R}  \ottsym{\{}  \mathit{x'}  \mapsto  \ottnt{v}  \ottsym{\}}\\
    }
    By inversion on the configuration typing relationship, we have that:
    \begin{align*}
       \Theta   \mid   \oldvec{\ell}   \mid   \Gamma_{{\mathrm{0}}}   \vdash   \ottnt{E}  \ottsym{[}   \LET  \mathit{x}  =   *  \mathit{y}   \IN  \ottnt{e}   \ottsym{]}  :  \tau_{\ottmv{n}}   \produces   \Gamma_{\ottmv{n}}  && \ottkw{Cons} \, \ottsym{(}  \ottnt{H}  \ottsym{,}  \ottnt{R}  \ottsym{,}  \Gamma_{{\mathrm{0}}}  \ottsym{)}
    \end{align*}
    By \Cref{lem:stack_var}, we have some $\tau, \Gamma'_{{\mathrm{0}}}$ such that:
    \begin{align*}
      \Theta  \mid  \HOLE  \ottsym{:}  \tau  \produces  \Gamma'_{{\mathrm{0}}}  \mid  \oldvec{\ell}  \vdash_{\mathit{ectx} }  \ottnt{E}  \ottsym{:}  \tau_{\ottmv{n}}  \produces  \Gamma_{\ottmv{n}} &&  \Theta   \mid   \oldvec{\ell}   \mid   \Gamma_{{\mathrm{0}}}   \vdash    \LET  \mathit{x}  =   *  \mathit{y}   \IN  \ottnt{e}   :  \tau   \produces   \Gamma'_{{\mathrm{0}}} 
    \end{align*}
    Using \Cref{lem:inversion}, we have some $\Gamma_{\ottmv{p}}$, $\Gamma'_{\ottmv{p}}$ and $\tau_{\ottmv{p}}$ such
    that:
    
    \begin{gather*}
      \Gamma_{{\mathrm{0}}}  \leq  \Gamma_{\ottmv{p}} \quad\quad  \oldvec{\ell}   \vdash _{\wf}  \Gamma_{\ottmv{p}}  \quad\quad \Gamma'_{\ottmv{p}}  \ottsym{,}  \tau_{\ottmv{p}}  \leq  \Gamma'_{{\mathrm{0}}}  \ottsym{,}  \tau \\
      \Gamma_{\ottmv{p}}  \ottsym{(}  \mathit{y}  \ottsym{)}  \ottsym{=}   \ottsym{(}  \tau_{{\mathrm{1}}}  \ottsym{+}  \tau_{{\mathrm{2}}}  \ottsym{)}  \TREF^{ r }  \quad\quad  \mathit{x}  \not\in \DOM( \Gamma'_{\ottmv{p}} )  \\
      \tau'' = \begin{cases}
        \ottsym{(}   \tau_{{\mathrm{1}}}  \wedge_{ \mathit{y} }   \mathit{y}  =_{ \tau_{{\mathrm{1}}} }  \mathit{x}    \ottsym{)} &  r   \ottsym{>}   \ottsym{0}  \\
        \tau_{{\mathrm{1}}} & r  \ottsym{=}  \ottsym{0}
      \end{cases} \\
       \Theta   \mid   \oldvec{\ell}   \mid   \Gamma_{\ottmv{p}}  \ottsym{[}  \mathit{y}  \hookleftarrow   \tau''  \TREF^{ r }   \ottsym{]}  \ottsym{,}  \mathit{x}  \ottsym{:}  \tau_{{\mathrm{2}}}   \vdash   \ottnt{e}  :  \tau_{\ottmv{p}}   \produces   \Gamma'_{\ottmv{p}} 
    \end{gather*}
    
    From \Cref{lem:subtyp-preserves-cons}, we then have $\ottkw{Cons} \, \ottsym{(}  \ottnt{H}  \ottsym{,}  \ottnt{R}  \ottsym{,}  \Gamma_{\ottmv{p}}  \ottsym{)}$.
    We will now show that:
    \begin{align}
      \label{eqn:new-tyenv-cons} & \ottkw{Cons} \, \ottsym{(}  \ottnt{H}  \ottsym{,}  \ottnt{R}  \ottsym{\{}  \mathit{x'}  \mapsto  \ottnt{v}  \ottsym{\}}  \ottsym{,}  \Gamma''  \ottsym{)} \\
                                 &  \Theta   \mid   \oldvec{\ell}   \mid   \Gamma''   \vdash     [  \mathit{x'}  /  \mathit{x}  ]    \ottnt{e}   :   [  \mathit{x'}  /  \mathit{x}  ]  \, \tau_{\ottmv{p}}   \produces    [  \mathit{x'}  /  \mathit{x}  ]  \, \Gamma'_{\ottmv{p}} 
    \end{align}
    where $\Gamma''  \ottsym{=}  \Gamma_{\ottmv{p}}  \ottsym{[}  \mathit{y}  \hookleftarrow   \ottsym{(}   [  \mathit{x'}  /  \mathit{x}  ]  \, \tau''  \ottsym{)}  \TREF^{ r }   \ottsym{]}  \ottsym{,}  \mathit{x'}  \ottsym{:}  \tau_{{\mathrm{2}}}$.

    Together these give our desired result. To see how,
    from $ \Theta   \mid   \oldvec{\ell}   \mid   \Gamma_{\ottmv{p}}  \ottsym{[}  \mathit{y}  \hookleftarrow   \tau''  \TREF^{ r }   \ottsym{]}  \ottsym{,}  \mathit{x}  \ottsym{:}  \tau_{{\mathrm{2}}}   \vdash   \ottnt{e}  :  \tau_{\ottmv{p}}   \produces   \Gamma'_{\ottmv{p}} $ above,
    we must have that $ \oldvec{\ell}   \vdash _{\wf}  \tau_{\ottmv{p}}   \produces   \Gamma'_{\ottmv{p}} $.
    From $ \mathit{x}  \not\in \DOM( \Gamma'_{\ottmv{p}} ) $ we must therefore have that
    $ [  \mathit{x'}  /  \mathit{x}  ]  \, \tau_{\ottmv{p}}  \ottsym{=}  \tau_{\ottmv{p}}$ and $ [  \mathit{x'}  /  \mathit{x}  ]  \, \Gamma'_{\ottmv{p}}  \ottsym{=}  \Gamma'_{\ottmv{p}}$. As $\Gamma'_{\ottmv{p}}  \ottsym{,}  \tau_{\ottmv{p}}  \leq  \Gamma'_{{\mathrm{0}}}  \ottsym{,}  \tau$
    an application of \rn{T-Sub} gives
    $ \Theta   \mid   \oldvec{\ell}   \mid   \Gamma''   \vdash     [  \mathit{x'}  /  \mathit{x}  ]    \ottnt{e}   :  \tau   \produces   \Gamma'_{{\mathrm{0}}} $. Then \Cref{lem:ectxt-sub-well-typed}
    will give that $ \Theta   \mid   \oldvec{\ell}   \mid   \Gamma''   \vdash   \ottnt{E}  \ottsym{[}    [  \mathit{x'}  /  \mathit{x}  ]    \ottnt{e}   \ottsym{]}  :  \tau_{\ottmv{n}}   \produces   \Gamma_{\ottmv{n}} $.
    
    As $\ottnt{E}$ and the stack $\oldvec{F}$ remained unchanged, combined with
    \Cref{eqn:new-tyenv-cons} this gives
    $ \vdash_{\mathit{conf} }^D   \tuple{ \ottnt{H} ,  \ottnt{R}  \ottsym{\{}  \mathit{x'}  \mapsto  \ottnt{v}  \ottsym{\}} ,  \oldvec{F} ,  \ottnt{E}  \ottsym{[}    [  \mathit{x'}  /  \mathit{x}  ]    \ottnt{e}   \ottsym{]} }  $ as required.
    As the above argument is used almost completely unchanged
    in all of the following cases, we will invert the redex without regard
    for the \rn{T-Sub} rule, with the understanding that the subtyping rule
    is handled with an argument identical to the above.
    
    We now show that $ \Theta   \mid   \oldvec{\ell}   \mid   \Gamma''   \vdash     [  \mathit{x'}  /  \mathit{x}  ]    \ottnt{e}   :   [  \mathit{x}  /  \mathit{x'}  ]  \, \tau_{\ottmv{p}}   \produces    [  \mathit{x'}  /  \mathit{x}  ]  \, \Gamma' $ and
    $\ottkw{Cons} \, \ottsym{(}  \ottnt{H}  \ottsym{,}  \ottnt{R}  \ottsym{\{}  \mathit{x'}  \mapsto  \ottnt{v}  \ottsym{\}}  \ottsym{,}  \Gamma''  \ottsym{)}$.
    The first is easy to obtain using \Cref{lem:substitution}; from $\ottkw{Cons} \, \ottsym{(}  \ottnt{H}  \ottsym{,}  \ottnt{R}  \ottsym{,}  \Gamma  \ottsym{)}$,
    we have that $\forall \,  \mathit{x}  \in \DOM( \Gamma )   \ottsym{.}   \mathit{x}  \in   \DOM( \ottnt{R} )  $, whereby from $ \mathit{x}  \not\in   \DOM( \ottnt{R} )  $ we have
    $ \mathit{x'}  \not\in   \DOM( \Gamma )  $.
    It therefore remains to show $\ottkw{Cons} \, \ottsym{(}  \ottnt{H}  \ottsym{,}  \ottnt{R}  \ottsym{\{}  \mathit{x'}  \mapsto  \ottnt{v}  \ottsym{\}}  \ottsym{,}  \Gamma''  \ottsym{)}$.

    To show $\ottkw{SAT} \, \ottsym{(}  \ottnt{H}  \ottsym{,}  \ottnt{R}  \ottsym{,}  \Gamma''  \ottsym{)}$, it suffices to
    show that $ \ottkw{SATv} ( \ottnt{H} , \ottnt{R'} , \ottnt{R'}  \ottsym{(}  \mathit{x'}  \ottsym{)} , \tau_{{\mathrm{2}}} ) $
    and $ \ottkw{SATv} ( \ottnt{H} , \ottnt{R'} , \ottnt{H}  \ottsym{(}  \ottnt{R'}  \ottsym{(}  \mathit{y}  \ottsym{)}  \ottsym{)} , \tau'' ) $ (that $\ottkw{SATv}$ holds for all other variables
    $\mathit{z}$ follows from $\Gamma  \ottsym{(}  \mathit{z}  \ottsym{)}  \ottsym{=}  \Gamma''  \ottsym{(}  \mathit{z}  \ottsym{)}$ and \Cref{lem:r-valid-subst,lem:register}).
    If $\tau_{{\mathrm{1}}}$ is an integer type and $ r   \ottsym{>}   \ottsym{0} $, then
    by the definition of the strengthening operator, the latter is equivalent to show
    that $ \ottkw{SATv} ( \ottnt{H} , \ottnt{R'} , \ottnt{H}  \ottsym{(}  \ottnt{R'}  \ottsym{(}  \mathit{y}  \ottsym{)}  \ottsym{)} , \tau_{{\mathrm{1}}} ) $
    and that $\ottnt{R'}  \ottsym{(}  \mathit{x'}  \ottsym{)}  \ottsym{=}  \ottnt{H}  \ottsym{(}  \ottnt{R'}  \ottsym{(}  \mathit{y}  \ottsym{)}  \ottsym{)}  \ottsym{=}  \ottnt{H}  \ottsym{(}  \ottnt{R}  \ottsym{(}  \mathit{y}  \ottsym{)}  \ottsym{)}$, which is immediate from the definition of $\rn{R-Deref}$.
    If $\tau_{{\mathrm{1}}}$ is not an integer or if $r  \ottsym{=}  \ottsym{0}$,
    then we must only show that $ \ottkw{SATv} ( \ottnt{H} , \ottnt{R'} , \ottnt{H}  \ottsym{(}  \ottnt{R'}  \ottsym{(}  \mathit{y}  \ottsym{)}  \ottsym{)} , \tau_{{\mathrm{1}}} ) $.
    
    From $\ottkw{Cons} \, \ottsym{(}  \ottnt{H}  \ottsym{,}  \ottnt{R}  \ottsym{,}  \Gamma_{\ottmv{p}}  \ottsym{)}$, we know that $\ottkw{SAT} \, \ottsym{(}  \ottnt{H}  \ottsym{,}  \ottnt{R}  \ottsym{,}  \Gamma_{\ottmv{p}}  \ottsym{)}$, in particular, $ \ottkw{SATv} ( \ottnt{H} , \ottnt{R} , \ottnt{R}  \ottsym{(}  \mathit{y}  \ottsym{)} , \Gamma_{\ottmv{p}}  \ottsym{(}  \mathit{y}  \ottsym{)} )  $.
    Then by \Cref{lem:satadd,lem:r-valid-subst,lem:register}, from $ \ottnt{R}  \sqsubseteq  \ottnt{R'} $, $ \oldvec{\ell}   \vdash _{\wf}  \Gamma_{\ottmv{p}} $, and $ \ottkw{SATv} ( \ottnt{H} , \ottnt{R} , \ottnt{v} , \tau_{{\mathrm{1}}}  \ottsym{+}  \tau_{{\mathrm{2}}} ) $ we obtain
    that $ \ottkw{SATv} ( \ottnt{H} , \ottnt{R'} , \ottnt{v} , \tau_{{\mathrm{1}}} ) $ and $ \ottkw{SATv} ( \ottnt{H} , \ottnt{R'} , \ottnt{v} , \tau_{{\mathrm{2}}} ) $, where $\ottnt{v} \, \ottsym{=} \, \ottnt{H}  \ottsym{(}  \ottnt{R}  \ottsym{(}  \mathit{y}  \ottsym{)}  \ottsym{)}$.
    We thus have that $ \ottkw{SATv} ( \ottnt{H} , \ottnt{R'} , \ottnt{R'}  \ottsym{(}  \mathit{x'}  \ottsym{)} , \tau_{{\mathrm{2}}} ) $ and $ \ottkw{SATv} ( \ottnt{H} , \ottnt{R'} , \ottnt{H}  \ottsym{(}  \ottnt{R'}  \ottsym{(}  \mathit{y}  \ottsym{)}  \ottsym{)} , \tau_{{\mathrm{1}}} ) $ are satisfied.
    
    We must also show that the ownership invariant is preserved.
    Then, it's to show $\forall \,  \ottmv{a}  \in \DOM( \ottnt{H} )   \ottsym{.}  \ottkw{Own} \, \ottsym{(}  \ottnt{H}  \ottsym{,}  \ottnt{R'}  \ottsym{,}  \Gamma''  \ottsym{)}  \ottsym{(}  \ottmv{a}  \ottsym{)}  \le  \ottsym{1}$. Define $\ottnt{O'_{{\mathrm{0}}}}$ and $\ottnt{O'_{{\mathrm{1}}}}$ as follows:
    \begin{align*}
      \ottkw{Own} \, \ottsym{(}  \ottnt{H}  \ottsym{,}  \ottnt{R}  \ottsym{,}  \Gamma_{\ottmv{p}}  \ottsym{)} & = \ottnt{O'_{{\mathrm{0}}}}  \ottsym{+}  \ottkw{own} \, \ottsym{(}  \ottnt{H}  \ottsym{,}  \ottnt{R}  \ottsym{(}  \mathit{y}  \ottsym{)}  \ottsym{,}  \Gamma_{\ottmv{p}}  \ottsym{(}  \mathit{y}  \ottsym{)}  \ottsym{)} \\
      \ottkw{Own} \, \ottsym{(}  \ottnt{H}  \ottsym{,}  \ottnt{R'}  \ottsym{,}  \Gamma''  \ottsym{)} & =\ottnt{O'_{{\mathrm{1}}}}  \ottsym{+}  \ottkw{own} \, \ottsym{(}  \ottnt{H}  \ottsym{,}  \ottnt{R'}  \ottsym{(}  \mathit{y}  \ottsym{)}  \ottsym{,}  \Gamma''  \ottsym{(}  \mathit{y}  \ottsym{)}  \ottsym{)}  \ottsym{+}  \ottkw{own} \, \ottsym{(}  \ottnt{H}  \ottsym{,}  \ottnt{R'}  \ottsym{(}  \mathit{x'}  \ottsym{)}  \ottsym{,}  \Gamma''  \ottsym{(}  \mathit{x'}  \ottsym{)}  \ottsym{)} \\
      \ottnt{O'_{{\mathrm{0}}}} &  =  \Sigma _{ \mathit{z} \in   \DOM( \Gamma )   \setminus   \set{ \mathit{y} }   }\, \ottkw{own} \, \ottsym{(}  \ottnt{H}  \ottsym{,}  \ottnt{R}  \ottsym{(}  \mathit{z}  \ottsym{)}  \ottsym{,}  \Gamma_{\ottmv{p}}  \ottsym{(}  \mathit{z}  \ottsym{)}  \ottsym{)}  \\
      \ottnt{O'_{{\mathrm{1}}}} & =  \Sigma _{ \mathit{z} \in   \DOM( \Gamma'' )   \setminus  \ottsym{\{}  \mathit{y}  \ottsym{,}  \mathit{x'}  \ottsym{\}}  }\, \ottkw{own} \, \ottsym{(}  \ottnt{H}  \ottsym{,}  \ottnt{R'}  \ottsym{(}  \mathit{z'}  \ottsym{)}  \ottsym{,}  \Gamma''  \ottsym{(}  \mathit{z'}  \ottsym{)}  \ottsym{)} 
    \end{align*}
    By \Cref{lem:heapop}, $\ottnt{O'_{{\mathrm{0}}}}  \ottsym{=}  \ottnt{O'_{{\mathrm{1}}}}$ holds. Then, it suffices to show that
    $\ottkw{own} \, \ottsym{(}  \ottnt{H}  \ottsym{,}  \ottnt{R}  \ottsym{(}  \mathit{y}  \ottsym{)}  \ottsym{,}  \Gamma_{\ottmv{p}}  \ottsym{(}  \mathit{y}  \ottsym{)}  \ottsym{)}  \ottsym{=}  \ottkw{own} \, \ottsym{(}  \ottnt{H}  \ottsym{,}  \ottnt{R'}  \ottsym{(}  \mathit{y}  \ottsym{)}  \ottsym{,}  \Gamma''  \ottsym{(}  \mathit{y}  \ottsym{)}  \ottsym{)}  \ottsym{+}  \ottkw{own} \, \ottsym{(}  \ottnt{H}  \ottsym{,}  \ottnt{R'}  \ottsym{(}  \mathit{x'}  \ottsym{)}  \ottsym{,}  \Gamma''  \ottsym{(}  \mathit{x'}  \ottsym{)}  \ottsym{)}$.
    
    As $\ottnt{R'}  \ottsym{(}  \mathit{x'}  \ottsym{)}  \ottsym{=}  \ottnt{H}  \ottsym{(}  \ottnt{R'}  \ottsym{(}  \mathit{y}  \ottsym{)}  \ottsym{)}  \ottsym{=}  \ottnt{H}  \ottsym{(}  \ottnt{R}  \ottsym{(}  \mathit{y}  \ottsym{)}  \ottsym{)}$ and from the definition of $\Gamma''$, we have:
    \begin{align*}
       \ottkw{own} \, \ottsym{(}  \ottnt{H}  \ottsym{,}  \ottnt{R'}  \ottsym{(}  \mathit{x'}  \ottsym{)}  \ottsym{,}  \Gamma''  \ottsym{(}  \mathit{x'}  \ottsym{)}  \ottsym{)}  & =  \ottkw{own} \, \ottsym{(}  \ottnt{H}  \ottsym{,}  \ottnt{H}  \ottsym{(}  \ottnt{R}  \ottsym{(}  \mathit{y}  \ottsym{)}  \ottsym{)}  \ottsym{,}  \tau_{{\mathrm{2}}}  \ottsym{)}  \\
       \ottkw{own} \, \ottsym{(}  \ottnt{H}  \ottsym{,}  \ottnt{R'}  \ottsym{(}  \mathit{y}  \ottsym{)}  \ottsym{,}  \Gamma''  \ottsym{(}  \mathit{y}  \ottsym{)}  \ottsym{)}  & =  \ottsym{\{}  \ottmv{a}  \mapsto  r  \ottsym{\}}  \ottsym{+}  \ottkw{own} \, \ottsym{(}  \ottnt{H}  \ottsym{,}  \ottnt{H}  \ottsym{(}  \ottnt{R}  \ottsym{(}  \mathit{y}  \ottsym{)}  \ottsym{)}  \ottsym{,}  \tau_{{\mathrm{1}}}  \ottsym{)} 
    \end{align*}
    From the definition of the ownership function, we have that
    \[
      \ottkw{own} \, \ottsym{(}  \ottnt{H}  \ottsym{,}  \ottnt{R}  \ottsym{(}  \mathit{y}  \ottsym{)}  \ottsym{,}  \Gamma_{\ottmv{p}}  \ottsym{(}  \mathit{y}  \ottsym{)}  \ottsym{)}  \ottsym{=}  \ottsym{\{}  \ottmv{a}  \mapsto  r  \ottsym{\}}  \ottsym{+}  \ottkw{own} \, \ottsym{(}  \ottnt{H}  \ottsym{,}  \ottnt{H}  \ottsym{(}  \ottnt{R}  \ottsym{(}  \mathit{y}  \ottsym{)}  \ottsym{)}  \ottsym{,}  \tau_{{\mathrm{1}}}  \ottsym{+}  \tau_{{\mathrm{2}}}  \ottsym{)}
    \]
    which, by \Cref{lem:ownadd}, is equivalent to:
    \[
      \ottsym{\{}  \ottmv{a}  \mapsto  r  \ottsym{\}}  \ottsym{+}  \ottkw{own} \, \ottsym{(}  \ottnt{H}  \ottsym{,}  \ottnt{H}  \ottsym{(}  \ottnt{R}  \ottsym{(}  \mathit{y}  \ottsym{)}  \ottsym{)}  \ottsym{,}  \tau_{{\mathrm{1}}}  \ottsym{)}  \ottsym{+}  \ottkw{own} \, \ottsym{(}  \ottnt{H}  \ottsym{,}  \ottnt{H}  \ottsym{(}  \ottnt{R}  \ottsym{(}  \mathit{y}  \ottsym{)}  \ottsym{)}  \ottsym{,}  \tau_{{\mathrm{2}}}  \ottsym{)}
    \]
    We therefore have $\ottkw{own} \, \ottsym{(}  \ottnt{H}  \ottsym{,}  \ottnt{R}  \ottsym{(}  \mathit{y}  \ottsym{)}  \ottsym{,}  \Gamma  \ottsym{(}  \mathit{y}  \ottsym{)}  \ottsym{)}  \ottsym{=}  \ottkw{own} \, \ottsym{(}  \ottnt{H}  \ottsym{,}  \ottnt{R'}  \ottsym{(}  \mathit{y}  \ottsym{)}  \ottsym{,}  \Gamma''  \ottsym{(}  \mathit{y}  \ottsym{)}  \ottsym{)}  \ottsym{+}  \ottkw{own} \, \ottsym{(}  \ottnt{H}  \ottsym{,}  \ottnt{R'}  \ottsym{(}  \mathit{x'}  \ottsym{)}  \ottsym{,}  \Gamma''  \ottsym{(}  \mathit{x'}  \ottsym{)}  \ottsym{)}$, and conclude that $\ottkw{Own} \, \ottsym{(}  \ottnt{H}  \ottsym{,}  \ottnt{R}  \ottsym{,}  \Gamma  \ottsym{)}  \ottsym{=}  \ottkw{Own} \, \ottsym{(}  \ottnt{H}  \ottsym{,}  \ottnt{R'}  \ottsym{,}  \Gamma''  \ottsym{)}$.
  \end{rneqncase} % R-Deref

  \begin{rneqncase}{R-Seq}{
       \vdash_{\mathit{conf} }^D   \tuple{ \ottnt{H} ,  \ottnt{R} ,  \oldvec{F} ,  \ottnt{E}  \ottsym{[}   \mathit{x}  \SEQ  \ottnt{e}   \ottsym{]} }  \\
        \tuple{ \ottnt{H} ,  \ottnt{R} ,  \oldvec{F} ,  \ottnt{E}  \ottsym{[}   \mathit{x}  \SEQ  \ottnt{e}   \ottsym{]} }     \longrightarrow _{ \ottnt{D} }     \tuple{ \ottnt{H} ,  \ottnt{R} ,  \oldvec{F} ,  \ottnt{E}  \ottsym{[}  \ottnt{e}  \ottsym{]} }  \\
    }
    By inversion (see \rn{R-Deref}) we have for some $\Gamma$ that:
    \begin{align*}
      &  \Theta   \mid   \oldvec{\ell}   \mid   \Gamma  \ottsym{[}  \mathit{x}  \ottsym{:}  \tau_{{\mathrm{0}}}  \ottsym{+}  \tau_{{\mathrm{1}}}  \ottsym{]}   \vdash   \mathit{x}  :  \tau_{{\mathrm{0}}}   \produces   \Gamma  \ottsym{[}  \mathit{x}  \hookleftarrow  \tau_{{\mathrm{1}}}  \ottsym{]}  \\
      &  \Theta   \mid   \oldvec{\ell}   \mid   \Gamma  \ottsym{[}  \mathit{x}  \hookleftarrow  \tau_{{\mathrm{1}}}  \ottsym{]}   \vdash   \ottnt{e}  :  \tau'   \produces   \Gamma'  \\
      & \ottkw{Cons} \, \ottsym{(}  \ottnt{H}  \ottsym{,}  \ottnt{R}  \ottsym{,}  \Gamma  \ottsym{)}
    \end{align*}
    We take $\Gamma''  \ottsym{=}  \Gamma  \ottsym{[}  \mathit{x}  \hookleftarrow  \tau_{{\mathrm{1}}}  \ottsym{]}$.
    
    It suffices to show (see \rn{R-Deref})
    that $ \Theta   \mid   \oldvec{\ell}   \mid   \Gamma''   \vdash   \ottnt{e}  :  \tau'   \produces   \Gamma' $,
    and $\ottkw{Cons} \, \ottsym{(}  \ottnt{H}  \ottsym{,}  \ottnt{R}  \ottsym{,}  \Gamma''  \ottsym{)}$.
    The first is immediate from the inversion above, and
    $\ottkw{Cons} \, \ottsym{(}  \ottnt{H}  \ottsym{,}  \ottnt{R}  \ottsym{,}  \Gamma''  \ottsym{)}$ follows from \Cref{lem:ownadd,lem:satadd}.
  \end{rneqncase}

  % FIXED
  \begin{rneqncase}{R-Let}{
       \vdash_{\mathit{conf} }^D   \tuple{ \ottnt{H} ,  \ottnt{R} ,  \oldvec{F} ,  \ottnt{E}  \ottsym{[}   \LET  \mathit{x}  =  \mathit{y}  \IN  \ottnt{e}   \ottsym{]} }  \\
        \tuple{ \ottnt{H} ,  \ottnt{R} ,  \oldvec{F} ,  \ottnt{E}  \ottsym{[}   \LET  \mathit{x}  =  \mathit{y}  \IN  \ottnt{e}   \ottsym{]} }     \longrightarrow _{ \ottnt{D} }     \tuple{ \ottnt{H} ,  \ottnt{R}  \ottsym{\{}  \mathit{x'}  \mapsto  \ottnt{R}  \ottsym{(}  \mathit{y}  \ottsym{)}  \ottsym{\}} ,  \oldvec{F} ,  \ottnt{E}  \ottsym{[}    [  \mathit{x'}  /  \mathit{x}  ]    \ottnt{e}   \ottsym{]} }  \\
       \mathit{x'}  \not\in \DOM( \ottnt{R} )  \andalso \ottnt{R'}  \ottsym{=}  \ottnt{R}  \ottsym{\{}  \mathit{x'}  \mapsto  \ottnt{R}  \ottsym{(}  \mathit{y}  \ottsym{)}  \ottsym{\}}
    }
    By inversion (see \rn{R-Deref}) we have that for some $\Gamma$ that:
    \begin{align*}
      &  \oldvec{\ell}   \vdash _{\wf}  \Gamma \\
      & \Gamma  \ottsym{(}  \mathit{y}  \ottsym{)}  \ottsym{=}  \tau_{{\mathrm{1}}}  \ottsym{+}  \tau_{{\mathrm{2}}} \\
      &  \Theta   \mid   \oldvec{\ell}   \mid   \Gamma  \ottsym{[}  \mathit{y}  \hookleftarrow   \tau_{{\mathrm{1}}}  \wedge_{ \mathit{y} }   \mathit{y}  =_{ \tau_{{\mathrm{1}}} }  \mathit{x}    \ottsym{]}  \ottsym{,}  \mathit{x}  \ottsym{:}  \ottsym{(}   \tau_{{\mathrm{2}}}  \wedge_{ \mathit{x} }   \mathit{x}  =_{ \tau_{{\mathrm{2}}} }  \mathit{y}    \ottsym{)}   \vdash   \ottnt{e}  :  \tau   \produces   \Gamma'  \\
      & \ottkw{Cons} \, \ottsym{(}  \ottnt{H}  \ottsym{,}  \ottnt{R}  \ottsym{,}  \Gamma  \ottsym{)} \andalso  \mathit{x}  \not\in \DOM( \Gamma' ) 
    \end{align*}
    We give $\Gamma''  \ottsym{=}  \Gamma  \ottsym{[}  \mathit{y}  \hookleftarrow   \tau_{{\mathrm{1}}}  \wedge_{ \mathit{y} }   \mathit{y}  =_{ \tau_{{\mathrm{1}}} }  \mathit{x'}    \ottsym{]}  \ottsym{,}  \mathit{x'}  \ottsym{:}  \ottsym{(}   \tau_{{\mathrm{2}}}  \wedge_{ \mathit{x'} }   \mathit{x'}  =_{ \tau_{{\mathrm{2}}} }  \mathit{y}    \ottsym{)}$.
    
    It suffices to show (see \rn{R-Deref})
    that $ \Theta   \mid   \oldvec{\ell}   \mid   \Gamma''   \vdash     [  \mathit{x'}  /  \mathit{x}  ]    \ottnt{e}   :  \tau   \produces   \Gamma' $
    and $\ottkw{Cons} \, \ottsym{(}  \ottnt{H}  \ottsym{,}  \ottnt{R}  \ottsym{\{}  \mathit{x'}  \mapsto  \ottnt{R}  \ottsym{(}  \mathit{y}  \ottsym{)}  \ottsym{\}}  \ottsym{,}  \Gamma''  \ottsym{)}$.
    The first is easy to obtain using reasoning as in the \rn{R-Deref} case.
    It therefore remains to show $\ottkw{Cons} \, \ottsym{(}  \ottnt{H}  \ottsym{,}  \ottnt{R'}  \ottsym{,}  \Gamma''  \ottsym{)}$.
    
    To show that the output environment is consistent, we must show that $ \ottkw{SATv} ( \ottnt{H} , \ottnt{R'} , \ottnt{R'}  \ottsym{(}  \mathit{x'}  \ottsym{)} ,  \tau_{{\mathrm{2}}}  \wedge_{ \mathit{x'} }  \mathit{x'} \, \ottsym{=} \, \mathit{y}  ) $ and $ \ottkw{SATv} ( \ottnt{H} , \ottnt{R'} , \ottnt{R'}  \ottsym{(}  \mathit{y}  \ottsym{)} ,  \tau_{{\mathrm{1}}}  \wedge_{ \mathit{y} }  \mathit{y} \, \ottsym{=} \, \mathit{x'}  ) $.
    By reasoning similar to that in $\rn{R-Deref}$, it suffices to show that $ \ottkw{SATv} ( \ottnt{H} , \ottnt{R'} , \ottnt{R'}  \ottsym{(}  \mathit{x'}  \ottsym{)} , \tau_{{\mathrm{2}}} ) $ and $ \ottkw{SATv} ( \ottnt{H} , \ottnt{R'} , \ottnt{R'}  \ottsym{(}  \mathit{y}  \ottsym{)} , \tau_{{\mathrm{1}}} ) $.
    We know that $\ottkw{Cons} \, \ottsym{(}  \ottnt{H}  \ottsym{,}  \ottnt{R}  \ottsym{,}  \Gamma  \ottsym{)}$, from which me have $\ottkw{SAT} \, \ottsym{(}  \ottnt{H}  \ottsym{,}  \ottnt{R}  \ottsym{,}  \Gamma  \ottsym{)}$, in particular $ \mathit{y}  \in \DOM( \ottnt{R} ) $ and $ \ottkw{SATv} ( \ottnt{H} , \ottnt{R} , \ottnt{R}  \ottsym{(}  \mathit{y}  \ottsym{)} , \Gamma  \ottsym{(}  \mathit{y}  \ottsym{)} ) $.
    As $ \ottnt{R}  \sqsubseteq  \ottnt{R'} $ and $ \oldvec{\ell}   \vdash _{\wf}  \Gamma $, from \Cref{lem:satadd,lem:r-valid-subst,lem:register}, we obtain from $ \ottkw{SATv} ( \ottnt{H} , \ottnt{R} , \ottnt{v} , \tau_{{\mathrm{1}}}  \ottsym{+}  \tau_{{\mathrm{2}}} ) $ that
    $ \ottkw{SATv} ( \ottnt{H} , \ottnt{R'} , \ottnt{v} , \tau_{{\mathrm{1}}} ) $ and $ \ottkw{SATv} ( \ottnt{H} , \ottnt{R'} , \ottnt{v} , \tau_{{\mathrm{2}}} ) $ where $v = R(y)$.
    We then have $ \ottkw{SATv} ( \ottnt{H} , \ottnt{R'} , \ottnt{R'}  \ottsym{(}  \mathit{x'}  \ottsym{)} , \tau_{{\mathrm{2}}} ) $ and $ \ottkw{SATv} ( \ottnt{H} , \ottnt{R'} , \ottnt{R'}  \ottsym{(}  \mathit{y}  \ottsym{)} , \tau_{{\mathrm{1}}} ) $ are satisfied.
    
    We must also show that the ownership invariant is preserved.
    Then, it's to show $\forall \,  \ottmv{a}  \in \DOM( \ottnt{H} )   \ottsym{.}  \ottkw{Own} \, \ottsym{(}  \ottnt{H}  \ottsym{,}  \ottnt{R'}  \ottsym{,}  \Gamma''  \ottsym{)}  \ottsym{(}  \ottmv{a}  \ottsym{)}  \le  \ottsym{1}$. Define $\ottnt{O'_{{\mathrm{0}}}}$ and $\ottnt{O'_{{\mathrm{1}}}}$ as follows:
    \begin{align*}
    \ottkw{Own} \, \ottsym{(}  \ottnt{H}  \ottsym{,}  \ottnt{R}  \ottsym{,}  \Gamma  \ottsym{)} & = \ottnt{O'_{{\mathrm{0}}}}  \ottsym{+}  \ottkw{own} \, \ottsym{(}  \ottnt{H}  \ottsym{,}  \ottnt{R}  \ottsym{(}  \mathit{y}  \ottsym{)}  \ottsym{,}  \Gamma  \ottsym{(}  \mathit{y}  \ottsym{)}  \ottsym{)} \\ 
    \ottkw{Own} \, \ottsym{(}  \ottnt{H}  \ottsym{,}  \ottnt{R'}  \ottsym{,}  \Gamma''  \ottsym{)} & = \ottnt{O'_{{\mathrm{1}}}}  \ottsym{+}  \ottkw{own} \, \ottsym{(}  \ottnt{H}  \ottsym{,}  \ottnt{R'}  \ottsym{(}  \mathit{y}  \ottsym{)}  \ottsym{,}  \Gamma''  \ottsym{(}  \mathit{y}  \ottsym{)}  \ottsym{)}  \ottsym{+}  \ottkw{own} \, \ottsym{(}  \ottnt{H}  \ottsym{,}  \ottnt{R'}  \ottsym{(}  \mathit{x'}  \ottsym{)}  \ottsym{,}  \Gamma''  \ottsym{(}  \mathit{x'}  \ottsym{)}  \ottsym{)} \\
    \ottnt{O'_{{\mathrm{0}}}} & =  \Sigma _{ \mathit{z} \in   \DOM( \Gamma )   \setminus   \set{ \mathit{y} }   }\, \ottkw{own} \, \ottsym{(}  \ottnt{H}  \ottsym{,}  \ottnt{R}  \ottsym{(}  \mathit{z}  \ottsym{)}  \ottsym{,}  \Gamma  \ottsym{(}  \mathit{z}  \ottsym{)}  \ottsym{)}  \\
    \ottnt{O'_{{\mathrm{1}}}} & =  \Sigma _{ \mathit{z} \in   \DOM( \Gamma'' )   \setminus  \ottsym{\{}  \mathit{y}  \ottsym{,}  \mathit{x'}  \ottsym{\}}  }\, \ottkw{own} \, \ottsym{(}  \ottnt{H}  \ottsym{,}  \ottnt{R'}  \ottsym{(}  \mathit{z'}  \ottsym{)}  \ottsym{,}  \Gamma''  \ottsym{(}  \mathit{z'}  \ottsym{)}  \ottsym{)} 
    \end{align*}
    By \Cref{lem:heapop}, $\ottnt{O'_{{\mathrm{0}}}}  \ottsym{=}  \ottnt{O'_{{\mathrm{1}}}}$ holds.
    That $\ottkw{own} \, \ottsym{(}  \ottnt{H}  \ottsym{,}  \ottnt{R'}  \ottsym{(}  \mathit{x'}  \ottsym{)}  \ottsym{,}  \tau_{{\mathrm{2}}}  \ottsym{)}  \ottsym{+}  \ottkw{own} \, \ottsym{(}  \ottnt{H}  \ottsym{,}  \ottnt{R'}  \ottsym{(}  \mathit{y}  \ottsym{)}  \ottsym{,}  \tau_{{\mathrm{1}}}  \ottsym{)}  \ottsym{=}  \ottkw{own} \, \ottsym{(}  \ottnt{H}  \ottsym{,}  \ottnt{R}  \ottsym{(}  \mathit{y}  \ottsym{)}  \ottsym{,}  \tau_{{\mathrm{1}}}  \ottsym{+}  \tau_{{\mathrm{2}}}  \ottsym{)}$ follows immediately
    from \Cref{lem:ownadd} and the condition $\ottnt{R}  \ottsym{(}  \mathit{y}  \ottsym{)}  \ottsym{=}  \ottnt{R'}  \ottsym{(}  \mathit{x'}  \ottsym{)}  \ottsym{=}  \ottnt{R'}  \ottsym{(}  \mathit{y}  \ottsym{)}$.
    We therefore conclude that $\ottkw{Own} \, \ottsym{(}  \ottnt{H}  \ottsym{,}  \ottnt{R}  \ottsym{,}  \Gamma  \ottsym{)}  \ottsym{=}  \ottkw{Own} \, \ottsym{(}  \ottnt{H}  \ottsym{,}  \ottnt{R'}  \ottsym{,}  \Gamma''  \ottsym{)}$.
  \end{rneqncase} % R-Let
 
  \begin{rneqncase}{R-LetInt}{ % FIXED
       \vdash_{\mathit{conf} }^D   \tuple{ \ottnt{H} ,  \ottnt{R} ,  \oldvec{F} ,  \ottnt{E}  \ottsym{[}   \LET  \mathit{x}  =  n  \IN  \ottnt{e}   \ottsym{]} }  \\
        \tuple{ \ottnt{H} ,  \ottnt{R} ,  \oldvec{F} ,  \ottnt{E}  \ottsym{[}   \LET  \mathit{x}  =  n  \IN  \ottnt{e}   \ottsym{]} }     \longrightarrow _{ \ottnt{D} }     \tuple{ \ottnt{H} ,  \ottnt{R}  \ottsym{\{}  \mathit{x'}  \mapsto  n  \ottsym{\}} ,  \oldvec{F} ,  \ottnt{E}  \ottsym{[}    [  \mathit{x'}  /  \mathit{x}  ]    \ottnt{e}   \ottsym{]} }  
    }
    By inversion (see \rn{R-Deref}) we have that, for some $\Gamma$:
    \begin{align*}
      &  \Theta   \mid   \oldvec{\ell}   \mid   \Gamma  \ottsym{,}  \mathit{x}  \ottsym{:}   \set{  \nu  \COL \TINT \mid  \nu \, \ottsym{=} \, n }    \vdash   \ottnt{e}  :  \tau   \produces   \Gamma'  \\
      & \ottkw{Cons} \, \ottsym{(}  \ottnt{H}  \ottsym{,}  \ottnt{R}  \ottsym{,}  \Gamma  \ottsym{)} \andalso  \mathit{x}  \not\in   \DOM( \Gamma' )  
    \end{align*}
    
    We give that $\Gamma''  \ottsym{=}  \Gamma  \ottsym{,}  \mathit{x'}  \ottsym{:}   \set{  \nu  \COL \TINT \mid  \nu \, \ottsym{=} \, n } $, and it thus suffices to show
    that $ \Theta   \mid   \oldvec{\ell}   \mid   \Gamma''   \vdash     [  \mathit{x'}  /  \mathit{x}  ]    \ottnt{e}   :  \tau   \produces   \Gamma' $
    and $\ottkw{Cons} \, \ottsym{(}  \ottnt{H}  \ottsym{,}  \ottnt{R}  \ottsym{\{}  \mathit{x'}  \mapsto  n  \ottsym{\}}  \ottsym{,}  \Gamma''  \ottsym{)}$.
    The first one is easy to obtain using the \Cref{lem:substitution} (see \rn{R-Deref})
    and the latter is trivial by similar reasoning to the \rn{T-Let} and \rn{T-Deref} cases.
  \end{rneqncase} % R-LetInt
  
  \begin{rneqncase}{R-IfTrue}{ % DONE
       \vdash_{\mathit{conf} }^D   \tuple{ \ottnt{H} ,  \ottnt{R} ,  \oldvec{F} ,  \ottnt{E}  \ottsym{[}   \IFZERO  \mathit{y}  \THEN  \ottnt{e_{{\mathrm{1}}}}  \ELSE  \ottnt{e_{{\mathrm{2}}}}   \ottsym{]} }  \\
        \tuple{ \ottnt{H} ,  \ottnt{R} ,  \oldvec{F} ,  \ottnt{E}  \ottsym{[}   \IFZERO  \mathit{y}  \THEN  \ottnt{e_{{\mathrm{1}}}}  \ELSE  \ottnt{e_{{\mathrm{2}}}}   \ottsym{]} }     \longrightarrow _{ \ottnt{D} }     \tuple{ \ottnt{H} ,  \ottnt{R} ,  \oldvec{F} ,  \ottnt{E}  \ottsym{[}  \ottnt{e_{{\mathrm{1}}}}  \ottsym{]} }  
    }
    By inversion (see \rn{R-Deref}) we have that for some $\Gamma$:
    \begin{align*}
      & \Gamma  \ottsym{(}  \mathit{x}  \ottsym{)}  \ottsym{=}   \set{  \nu  \COL \TINT \mid  \varphi }  \\
      &  \Theta   \mid   \oldvec{\ell}   \mid   \Gamma  \ottsym{[}  \mathit{x}  \hookleftarrow   \set{  \nu  \COL \TINT \mid   \varphi  \wedge  \nu \, \ottsym{=} \, \ottsym{0}  }   \ottsym{]}   \vdash   \ottnt{e_{{\mathrm{1}}}}  :  \tau   \produces   \Gamma'  \\
      & \ottkw{Cons} \, \ottsym{(}  \ottnt{H}  \ottsym{,}  \ottnt{R}  \ottsym{,}  \Gamma  \ottsym{)}
    \end{align*}
    We take $\Gamma''  \ottsym{=}  \Gamma  \ottsym{[}  \mathit{x}  \hookleftarrow   \set{  \nu  \COL \TINT \mid   \varphi  \wedge  \nu \, \ottsym{=} \, \ottsym{0}  }   \ottsym{]}$, and want to show that $\ottkw{Cons} \, \ottsym{(}  \ottnt{H}  \ottsym{,}  \ottnt{R}  \ottsym{,}  \Gamma''  \ottsym{)}$ (that $ \Theta   \mid   \oldvec{\ell}   \mid   \Gamma''   \vdash   \ottnt{e_{{\mathrm{1}}}}  :  \tau   \produces   \Gamma' $ is immediate).
    
    By definition, from $\ottkw{Cons} \, \ottsym{(}  \ottnt{H}  \ottsym{,}  \ottnt{R}  \ottsym{,}  \Gamma  \ottsym{)}$ we have $\ottkw{SAT} \, \ottsym{(}  \ottnt{H}  \ottsym{,}  \ottnt{R}  \ottsym{,}  \Gamma  \ottsym{)}$, in particular $ \mathit{x}  \in \DOM( \ottnt{R} ) $, $ \ottnt{R}  \ottsym{(}  \mathit{x}  \ottsym{)}  \in  \mathbb{Z} $ and $\ottsym{[}  \ottnt{R}  \ottsym{]} \, \ottsym{[}  \ottnt{R}  \ottsym{(}  \mathit{x}  \ottsym{)}  \ottsym{/}  \nu  \ottsym{]}  \varphi$.
    The refinement predicates $\varphi$ still holds in the output environment, since nothing changes in the register after transition.
    Also from precondition of $\rn{R-IfTrue}$, we have $\ottnt{R}  \ottsym{(}  \mathit{x}  \ottsym{)} \, \ottsym{=} \, \ottsym{0}$, thus $\mathit{x}$ satisfies the refinement that $\nu \, \ottsym{=} \, \ottsym{0}$.
    Thus $\ottsym{[}  \ottnt{R}  \ottsym{]} \, \ottsym{[}  \ottnt{R}  \ottsym{(}  \mathit{x}  \ottsym{)}  \ottsym{/}  \nu  \ottsym{]}  \ottsym{(}   \varphi  \wedge  \nu \, \ottsym{=} \, \ottsym{0}   \ottsym{)}$ is trivially satisfied.
  \end{rneqncase} %R-IfTrue
  
  \begin{rncase}{R-IfFalse} % FIXED
    Similar to the case for \rn{R-IfTrue}.
  \end{rncase}
  
  \begin{rneqncase}{R-MkRef}{ 
       \vdash_{\mathit{conf} }^D   \tuple{ \ottnt{H} ,  \ottnt{R} ,  \oldvec{F} ,  \ottnt{E}  \ottsym{[}   \LET  \mathit{x}  =   \MKREF  \mathit{y}   \IN  \ottnt{e}   \ottsym{]} }  \\
        \tuple{ \ottnt{H} ,  \ottnt{R} ,  \oldvec{F} ,  \ottnt{E}  \ottsym{[}   \LET  \mathit{x}  =   \MKREF  \mathit{y}   \IN  \ottnt{e}   \ottsym{]} }     \longrightarrow _{ \ottnt{D} }     \tuple{ \ottnt{H'} ,  \ottnt{R'} ,  \oldvec{F} ,  \ottnt{E}  \ottsym{[}    [  \mathit{x'}  /  \mathit{x}  ]    \ottnt{e}   \ottsym{]} }  \\
       \ottmv{a}  \not\in \DOM( \ottnt{H} )  \andalso  \mathit{x'}  \not\in \DOM( \ottnt{R} )  \\
      \ottnt{H'}  \ottsym{=}  \ottnt{H}  \ottsym{\{}  \ottmv{a}  \mapsto  \ottnt{R}  \ottsym{(}  \mathit{y}  \ottsym{)}  \ottsym{\}} \andalso \ottnt{R'}  \ottsym{=}  \ottnt{R}  \ottsym{\{}  \mathit{x'}  \mapsto  \ottmv{a}  \ottsym{\}}
    }
    By inversion (see \rn{R-Deref}) we have that for some $\Gamma$:
    \begin{align*}
      &  \oldvec{\ell}   \vdash _{\wf}  \Gamma  \\
      & \Gamma  \ottsym{(}  \mathit{y}  \ottsym{)}  \ottsym{=}  \tau_{{\mathrm{1}}}  \ottsym{+}  \tau_{{\mathrm{2}}} \\
      &  \Theta   \mid   \oldvec{\ell}   \mid   \Gamma  \ottsym{[}  \mathit{y}  \hookleftarrow  \tau_{{\mathrm{1}}}  \ottsym{]}  \ottsym{,}  \mathit{x}  \ottsym{:}   \ottsym{(}   \tau_{{\mathrm{2}}}  \wedge_{ \mathit{x} }  \mathit{x} \, \ottsym{=} \, \mathit{y}   \ottsym{)}  \TREF^{ \ottsym{1} }    \vdash   \ottnt{e}  :  \tau   \produces   \Gamma'  \\
      & \ottkw{Cons} \, \ottsym{(}  \ottnt{H}  \ottsym{,}  \ottnt{R}  \ottsym{,}  \Gamma  \ottsym{)} \andalso  \mathit{x}  \not\in   \DOM( \Gamma' )  
    \end{align*}
    We give $\Gamma''  \ottsym{=}  \Gamma  \ottsym{[}  \mathit{y}  \hookleftarrow  \tau_{{\mathrm{1}}}  \ottsym{]}  \ottsym{,}  \mathit{x'}  \ottsym{:}   \ottsym{(}   \tau_{{\mathrm{2}}}  \wedge_{ \mathit{x'} }  \mathit{x'} \, \ottsym{=} \, \mathit{y}   \ottsym{)}  \TREF^{ \ottsym{1} } $,
    and must show that $ \Theta   \mid   \oldvec{\ell}   \mid   \Gamma''   \vdash     [  \mathit{x'}  /  \mathit{x}  ]    \ottnt{e}   :  \tau   \produces   \Gamma' $ and
    $\ottkw{Cons} \, \ottsym{(}  \ottnt{H'}  \ottsym{,}  \ottnt{R'}  \ottsym{,}  \Gamma''  \ottsym{)}$. The first follows
    from \Cref{lem:substitution} and the reasoning found in \rn{R-Deref},
    and the second from the assumed well-formedness of $\tau_{{\mathrm{1}}}  \ottsym{+}  \tau_{{\mathrm{2}}}$.
    
    It remains to show $\ottkw{Cons} \, \ottsym{(}  \ottnt{H'}  \ottsym{,}  \ottnt{R'}  \ottsym{,}  \Gamma''  \ottsym{)}$.
    To show that the output environment is consistent, we must show that $ \ottkw{SATv} ( \ottnt{H'} , \ottnt{R'} , \ottnt{R'}  \ottsym{(}  \mathit{x'}  \ottsym{)} ,  \ottsym{(}   \tau_{{\mathrm{2}}}  \wedge_{ \mathit{x'} }  \mathit{x'} \, \ottsym{=} \, \mathit{y}   \ottsym{)}  \TREF^{ \ottsym{1} }  ) $ and $ \ottkw{SATv} ( \ottnt{H'} , \ottnt{R'} , \ottnt{R'}  \ottsym{(}  \mathit{y}  \ottsym{)} , \tau_{{\mathrm{1}}} ) $.
    By reasoning similar to that in $\rn{R-Deref}$, it suffices to show that $ \ottkw{SATv} ( \ottnt{H'} , \ottnt{R'} , \ottnt{R'}  \ottsym{(}  \mathit{x'}  \ottsym{)} ,  \tau_{{\mathrm{2}}}  \TREF^{ \ottsym{1} }  ) $ and $ \ottkw{SATv} ( \ottnt{H'} , \ottnt{R'} , \ottnt{R'}  \ottsym{(}  \mathit{y}  \ottsym{)} , \tau_{{\mathrm{1}}} ) $.
    We know that $\ottkw{Cons} \, \ottsym{(}  \ottnt{H}  \ottsym{,}  \ottnt{R}  \ottsym{,}  \Gamma  \ottsym{)}$, from which we have $\ottkw{SAT} \, \ottsym{(}  \ottnt{H}  \ottsym{,}  \ottnt{R}  \ottsym{,}  \Gamma  \ottsym{)}$, in particular $ \mathit{y}  \in \DOM( \ottnt{R} ) $ and $ \ottkw{SATv} ( \ottnt{H} , \ottnt{R} , \ottnt{R}  \ottsym{(}  \mathit{y}  \ottsym{)} , \Gamma  \ottsym{(}  \mathit{y}  \ottsym{)} )  $.
    As $ \ottnt{R}  \sqsubseteq  \ottnt{R'} $ and $ \oldvec{\ell}   \vdash _{\wf}  \Gamma $, from \Cref{lem:register,lem:r-valid-subst}, we have $ \ottkw{SATv} ( \ottnt{H} , \ottnt{R} , \ottnt{R}  \ottsym{(}  \mathit{y}  \ottsym{)} , \tau_{{\mathrm{1}}}  \ottsym{+}  \tau_{{\mathrm{2}}} ) $ implies $ \ottkw{SATv} ( \ottnt{H} , \ottnt{R'} , \ottnt{R'}  \ottsym{(}  \mathit{y}  \ottsym{)} , \tau_{{\mathrm{1}}}  \ottsym{+}  \tau_{{\mathrm{2}}} ) $.
    By \Cref{lem:newaddheap}, we then have $ \ottkw{SATv} ( \ottnt{H'} , \ottnt{R'} , \ottnt{R}  \ottsym{(}  \mathit{y}  \ottsym{)} , \tau_{{\mathrm{1}}}  \ottsym{+}  \tau_{{\mathrm{2}}} ) $.
    Then by \Cref{lem:satadd}, we have $ \ottkw{SATv} ( \ottnt{H'} , \ottnt{R'} , \ottnt{v} , \tau_{{\mathrm{1}}} ) $ and $ \ottkw{SATv} ( \ottnt{H'} , \ottnt{R'} , \ottnt{v} , \tau_{{\mathrm{2}}} ) $ where $\ottnt{v} \, \ottsym{=} \, \ottnt{R}  \ottsym{(}  \mathit{y}  \ottsym{)}$.
    We then have $ \ottkw{SATv} ( \ottnt{H'} , \ottnt{R'} , \ottnt{R'}  \ottsym{(}  \mathit{x'}  \ottsym{)} ,  \tau_{{\mathrm{2}}}  \TREF^{ \ottsym{1} }  ) $ and $ \ottkw{SATv} ( \ottnt{H'} , \ottnt{R'} , \ottnt{R'}  \ottsym{(}  \mathit{y}  \ottsym{)} , \tau_{{\mathrm{1}}} ) $ are satisfied.
    
    We must also show that the ownership invariant is preserved.
    Then, it's to show $\forall \,  \ottmv{a'}  \in \DOM( \ottnt{H} )   \ottsym{.}  \ottkw{Own} \, \ottsym{(}  \ottnt{H'}  \ottsym{,}  \ottnt{R'}  \ottsym{,}  \Gamma''  \ottsym{)}  \ottsym{(}  \ottmv{a'}  \ottsym{)}  \le  \ottsym{1}$. From $\ottkw{Cons} \, \ottsym{(}  \ottnt{H}  \ottsym{,}  \ottnt{R}  \ottsym{,}  \Gamma  \ottsym{)}$ and \Cref{lem:ownaddheap,lem:ownadd} we have:
    \begin{align*}
      \ottkw{Own} \, \ottsym{(}  \ottnt{H'}  \ottsym{,}  \ottnt{R'}  \ottsym{,}  \Gamma''  \ottsym{)} & =  \Sigma _{ \mathit{z} \in  \DOM( \Gamma'' )  }\, \ottkw{own} \, \ottsym{(}  \ottnt{H'}  \ottsym{,}  \ottnt{R'}  \ottsym{(}  \mathit{z}  \ottsym{)}  \ottsym{,}  \Gamma''  \ottsym{(}  \mathit{z}  \ottsym{)}  \ottsym{)}  \\
             & =  \Sigma _{ \mathit{z} \in  \DOM( \Gamma )  }\, \ottkw{own} \, \ottsym{(}  \ottnt{H}  \ottsym{,}  \ottnt{R}  \ottsym{(}  \mathit{z}  \ottsym{)}  \ottsym{,}  \Gamma  \ottsym{(}  \mathit{z}  \ottsym{)}  \ottsym{)}   \ottsym{+}  \ottsym{\{}  \ottmv{a}  \mapsto  \ottsym{1}  \ottsym{\}} \\
             & = \ottkw{Own} \, \ottsym{(}  \ottnt{H}  \ottsym{,}  \ottnt{R}  \ottsym{,}  \Gamma  \ottsym{)}  \ottsym{+}  \ottsym{\{}  \ottmv{a}  \mapsto  \ottsym{1}  \ottsym{\}}
    \end{align*}

    Since $ \ottmv{a}  \not\in \DOM( \ottnt{H} ) $ and $\forall \,  \ottmv{a'}  \in \DOM( \ottnt{H} )   \ottsym{.}  \ottkw{Own} \, \ottsym{(}  \ottnt{H}  \ottsym{,}  \ottnt{R}  \ottsym{,}  \Gamma  \ottsym{)}  \ottsym{(}  \ottmv{a'}  \ottsym{)}  \le  \ottsym{1}$,
    we have $\forall \,  \ottmv{a'}  \in \DOM( \ottnt{H} )   \ottsym{.}  \ottkw{Own} \, \ottsym{(}  \ottnt{H'}  \ottsym{,}  \ottnt{R'}  \ottsym{,}  \Gamma''  \ottsym{)}  \ottsym{(}  \ottmv{a'}  \ottsym{)}  \le  \ottsym{1}$.

    % We want to show that $\forall \,  \ottmv{a}  \in \DOM( \ottnt{H'} )   \ottsym{.}  \ottkw{Own} \, \ottsym{(}  \ottnt{H'}  \ottsym{,}  \ottnt{R'}  \ottsym{,}  \Gamma''  \ottsym{)}  \ottsym{(}  \ottmv{a}  \ottsym{)}  \le  \ottsym{1}$.
    % Suppose $\ottmv{a} \, \neq \, \ottmv{a'}$ (that is $ \ottmv{a}  \in \DOM( \ottnt{H} ) $): we have $\ottkw{Own} \, \ottsym{(}  \ottnt{H}  \ottsym{,}  \ottnt{R}  \ottsym{,}  \Gamma  \ottsym{)}  \ottsym{(}  \ottmv{a}  \ottsym{)}  \ottsym{=}  \ottkw{Own} \, \ottsym{(}  \ottnt{H'}  \ottsym{,}  \ottnt{R'}  \ottsym{,}  \Gamma''  \ottsym{)}  \ottsym{(}  \ottmv{a}  \ottsym{)}$ and thus $\ottkw{Own} \, \ottsym{(}  \ottnt{H'}  \ottsym{,}  \ottnt{R'}  \ottsym{,}  \Gamma''  \ottsym{)}  \ottsym{(}  \ottmv{a}  \ottsym{)}  \le  \ottsym{1}$ by $\ottkw{Cons} \, \ottsym{(}  \ottnt{H}  \ottsym{,}  \ottnt{R}  \ottsym{,}  \Gamma  \ottsym{)}.$
    % Next, suppose $a=a'$; we have $\ottkw{Own} \, \ottsym{(}  \ottnt{H'}  \ottsym{,}  \ottnt{R'}  \ottsym{,}  \Gamma''  \ottsym{)}  \ottsym{(}  \ottmv{a}  \ottsym{)}  \ottsym{=}  \ottsym{1}$ only if $ \ottmv{a'}  \not\in \DOM( \ottkw{Own} \, \ottsym{(}  \ottnt{H}  \ottsym{,}  \ottnt{R}  \ottsym{,}  \Gamma  \ottsym{)} ) $;
    % this requirement immediately holds from $ \ottmv{a'}  \not\in \DOM( \ottnt{H} ) $ by the definition of
    % $\ottkw{own} \, \ottsym{(}  \ottnt{H}  \ottsym{,}  \ottnt{v}  \ottsym{,}  \tau  \ottsym{)}$. \JT{We have been a little sloppy with our definition of + on
    %   these ownership functions so the domains of these functions is a little unclear.}
  \end{rneqncase} % R-MkRef

  \begin{rneqncase}{R-Assign}{ % FIXED
       \vdash_{\mathit{conf} }^D   \tuple{ \ottnt{H} ,  \ottnt{R} ,  \oldvec{F} ,  \ottnt{E}  \ottsym{[}   \mathit{y}  \WRITE  \mathit{x}  \SEQ  \ottnt{e}   \ottsym{]} }  \\
        \tuple{ \ottnt{H} ,  \ottnt{R} ,  \oldvec{F} ,  \ottnt{E}  \ottsym{[}   \mathit{y}  \WRITE  \mathit{x}  \SEQ  \ottnt{e}   \ottsym{]} }     \longrightarrow _{ \ottnt{D} }     \tuple{ \ottnt{H'} ,  \ottnt{R'} ,  \oldvec{F} ,  \ottnt{E}  \ottsym{[}  \ottnt{e}  \ottsym{]} }   \\
      \ottmv{a} \, \ottsym{=} \, \ottnt{R}  \ottsym{(}  \mathit{y}  \ottsym{)} \andalso \ottnt{H'}  \ottsym{=}  \ottnt{H}  \ottsym{\{}  \ottmv{a}  \hookleftarrow  \ottnt{R}  \ottsym{(}  \mathit{x}  \ottsym{)}  \ottsym{\}} \andalso \ottnt{R'}  \ottsym{=}  \ottnt{R}
    }
    By inversion (see the \rn{R-Deref} case) we have that
    \begin{align*}
      &  \Theta   \mid   \oldvec{\ell}   \mid   \Gamma  \ottsym{[}  \mathit{x}  \ottsym{:}  \tau_{{\mathrm{1}}}  \ottsym{+}  \tau_{{\mathrm{2}}}  \ottsym{]}  \ottsym{[}  \mathit{y}  \ottsym{:}   \tau'  \TREF^{ \ottsym{1} }   \ottsym{]}   \vdash    \mathit{y}  \WRITE  \mathit{x}  \SEQ  \ottnt{e}   :  \tau   \produces   \Gamma'  \\
      &  \Theta   \mid   \oldvec{\ell}   \mid   \Gamma  \ottsym{[}  \mathit{x}  \hookleftarrow  \tau_{{\mathrm{1}}}  \ottsym{]}  \ottsym{[}  \mathit{y}  \hookleftarrow   \ottsym{(}   \tau_{{\mathrm{2}}}  \wedge_{ \mathit{y} }   \mathit{y}  =_{ \tau_{{\mathrm{2}}} }  \mathit{x}    \ottsym{)}  \TREF^{ \ottsym{1} }   \ottsym{]}   \vdash   \ottnt{e}  :  \tau   \produces   \Gamma'  \\
      & \ottkw{Cons} \, \ottsym{(}  \ottnt{H}  \ottsym{,}  \ottnt{R}  \ottsym{,}  \Gamma  \ottsym{)}
    \end{align*}
    We give $\Gamma''  \ottsym{=}  \Gamma  \ottsym{[}  \mathit{x}  \hookleftarrow  \tau_{{\mathrm{1}}}  \ottsym{]}  \ottsym{[}  \mathit{y}  \hookleftarrow   \ottsym{(}   \tau_{{\mathrm{2}}}  \wedge_{ \mathit{y} }   \mathit{y}  =_{ \tau_{{\mathrm{2}}} }  \mathit{x}    \ottsym{)}  \TREF^{ \ottsym{1} }   \ottsym{]}$. That
    $ \Theta   \mid   \oldvec{\ell}   \mid   \Gamma''   \vdash   \ottnt{e}  :  \tau   \produces   \Gamma' $ is immediate.
    
    We must therefore show that $\ottkw{Cons} \, \ottsym{(}  \ottnt{H'}  \ottsym{,}  \ottnt{R'}  \ottsym{,}  \Gamma''  \ottsym{)}$.
    To show that the output environment is consistent, we must show that $ \ottkw{SATv} ( \ottnt{H'} , \ottnt{R} , \ottnt{R}  \ottsym{(}  \mathit{y}  \ottsym{)} ,  \ottsym{(}   \tau_{{\mathrm{2}}}  \wedge_{ \mathit{y} }  \mathit{y} \, \ottsym{=} \, \mathit{x}   \ottsym{)}  \TREF^{ \ottsym{1} }  ) $ and $ \ottkw{SATv} ( \ottnt{H'} , \ottnt{R} , \ottnt{R}  \ottsym{(}  \mathit{x}  \ottsym{)} , \tau_{{\mathrm{1}}} ) $.
    By reasoning similar to that in $\rn{R-Deref}$, it suffices to show that $ \ottkw{SATv} ( \ottnt{H'} , \ottnt{R} , \ottnt{R}  \ottsym{(}  \mathit{y}  \ottsym{)} ,  \tau_{{\mathrm{2}}}  \TREF^{ \ottsym{1} }  ) $ and $ \ottkw{SATv} ( \ottnt{H'} , \ottnt{R} , \ottnt{R}  \ottsym{(}  \mathit{x}  \ottsym{)} , \tau_{{\mathrm{1}}} ) $.

    From $\ottkw{Cons} \, \ottsym{(}  \ottnt{H}  \ottsym{,}  \ottnt{R}  \ottsym{,}  \Gamma  \ottsym{)}$, we know that $\ottkw{SAT} \, \ottsym{(}  \ottnt{H}  \ottsym{,}  \ottnt{R}  \ottsym{,}  \Gamma  \ottsym{)}$, in particular, $ \ottkw{SATv} ( \ottnt{H} , \ottnt{R} , \ottnt{R}  \ottsym{(}  \mathit{x}  \ottsym{)} , \Gamma  \ottsym{(}  \mathit{x}  \ottsym{)} ) $.
    If we show that $\ottkw{own} \, \ottsym{(}  \ottnt{H}  \ottsym{,}  \ottnt{R}  \ottsym{(}  \mathit{x}  \ottsym{)}  \ottsym{,}  \tau_{{\mathrm{1}}}  \ottsym{+}  \tau_{{\mathrm{2}}}  \ottsym{)}  \ottsym{(}  \ottmv{a}  \ottsym{)}  \ottsym{=}  \ottsym{0}$ and $ \ottnt{H}   \approx _ \ottmv{a}   \ottnt{H'} $,
    then, by \Cref{lem:heapfor0}, we will obtain $ \ottkw{SATv} ( \ottnt{H'} , \ottnt{R} , \ottnt{R}  \ottsym{(}  \mathit{x}  \ottsym{)} , \tau_{{\mathrm{1}}}  \ottsym{+}  \tau_{{\mathrm{2}}} ) $,
    from which, by \Cref{lem:satadd},
    $ \ottkw{SATv} ( \ottnt{H'} , \ottnt{R} , \ottnt{R}  \ottsym{(}  \mathit{x}  \ottsym{)} , \tau_{{\mathrm{1}}} ) $ and $ \ottkw{SATv} ( \ottnt{H'} , \ottnt{R} , \ottnt{R}  \ottsym{(}  \mathit{x}  \ottsym{)} , \tau_{{\mathrm{2}}} ) $ follow.
    We then have $ \ottkw{SATv} ( \ottnt{H'} , \ottnt{R} , \ottnt{R}  \ottsym{(}  \mathit{y}  \ottsym{)} ,  \tau_{{\mathrm{2}}}  \TREF^{ \ottsym{1} }  ) $ and $ \ottkw{SATv} ( \ottnt{H'} , \ottnt{R} , \ottnt{R}  \ottsym{(}  \mathit{x}  \ottsym{)} , \tau_{{\mathrm{1}}} ) $ as $\ottnt{H'}  \ottsym{(}  \ottnt{R}  \ottsym{(}  \mathit{y}  \ottsym{)}  \ottsym{)} \, \ottsym{=} \, \ottnt{R}  \ottsym{(}  \mathit{x}  \ottsym{)}$.
    (That any other variables $\mathit{z}$ is consistent will follow from $\ottkw{own} \, \ottsym{(}  \ottnt{H}  \ottsym{,}  \ottnt{R}  \ottsym{(}  \mathit{z}  \ottsym{)}  \ottsym{,}  \Gamma  \ottsym{(}  \mathit{z}  \ottsym{)}  \ottsym{)}  \ottsym{(}  \ottmv{a}  \ottsym{)}  \ottsym{=}  \ottsym{0}$
    as proved below and \cref{lem:heapfor0}.)

    To show $\ottkw{own} \, \ottsym{(}  \ottnt{H}  \ottsym{,}  \ottnt{R}  \ottsym{(}  \mathit{x}  \ottsym{)}  \ottsym{,}  \tau_{{\mathrm{1}}}  \ottsym{+}  \tau_{{\mathrm{2}}}  \ottsym{)}  \ottsym{(}  \ottmv{a}  \ottsym{)}  \ottsym{=}  \ottsym{0}$, 
    we define $\ottnt{O'_{{\mathrm{0}}}}, \ottnt{O''_{{\mathrm{0}}}}, \ottnt{O'_{{\mathrm{1}}}}$ and $\ottnt{O''_{{\mathrm{1}}}}$ as below:
    \begin{align*}
      \ottkw{Own} \, \ottsym{(}  \ottnt{H}  \ottsym{,}  \ottnt{R}  \ottsym{,}  \Gamma  \ottsym{)} & = \ottnt{O'_{{\mathrm{0}}}}  \ottsym{+}  \ottnt{O''_{{\mathrm{0}}}} \\
      \ottkw{Own} \, \ottsym{(}  \ottnt{H'}  \ottsym{,}  \ottnt{R}  \ottsym{,}  \Gamma''  \ottsym{)} & = \ottnt{O'_{{\mathrm{1}}}}  \ottsym{+}  \ottnt{O''_{{\mathrm{1}}}} \\
      \ottnt{O'_{{\mathrm{0}}}} & =  \Sigma _{ \mathit{z} \in   \DOM( \Gamma )   \setminus  \ottsym{\{}  \mathit{y}  \ottsym{,}  \mathit{x}  \ottsym{\}}  }\, \ottkw{own} \, \ottsym{(}  \ottnt{H}  \ottsym{,}  \ottnt{R}  \ottsym{(}  \mathit{z}  \ottsym{)}  \ottsym{,}  \Gamma  \ottsym{(}  \mathit{z}  \ottsym{)}  \ottsym{)}  \\
      \ottnt{O''_{{\mathrm{0}}}} &= \ottkw{own} \, \ottsym{(}  \ottnt{H}  \ottsym{,}  \ottnt{R}  \ottsym{(}  \mathit{y}  \ottsym{)}  \ottsym{,}  \Gamma  \ottsym{(}  \mathit{y}  \ottsym{)}  \ottsym{)}  \ottsym{+}  \ottkw{own} \, \ottsym{(}  \ottnt{H}  \ottsym{,}  \ottnt{R}  \ottsym{(}  \mathit{x}  \ottsym{)}  \ottsym{,}  \Gamma  \ottsym{(}  \mathit{x}  \ottsym{)}  \ottsym{)} \\
      \ottnt{O'_{{\mathrm{1}}}} & =  \Sigma _{ \mathit{z} \in   \DOM( \Gamma'' )   \setminus  \ottsym{\{}  \mathit{y}  \ottsym{,}  \mathit{x}  \ottsym{\}}  }\, \ottkw{own} \, \ottsym{(}  \ottnt{H'}  \ottsym{,}  \ottnt{R}  \ottsym{(}  \mathit{z}  \ottsym{)}  \ottsym{,}  \Gamma''  \ottsym{(}  \mathit{z}  \ottsym{)}  \ottsym{)}  \\
      \ottnt{O''_{{\mathrm{1}}}} &= \ottkw{own} \, \ottsym{(}  \ottnt{H'}  \ottsym{,}  \ottnt{R}  \ottsym{(}  \mathit{y}  \ottsym{)}  \ottsym{,}  \Gamma''  \ottsym{(}  \mathit{y}  \ottsym{)}  \ottsym{)}  \ottsym{+}  \ottkw{own} \, \ottsym{(}  \ottnt{H'}  \ottsym{,}  \ottnt{R}  \ottsym{(}  \mathit{x}  \ottsym{)}  \ottsym{,}  \Gamma''  \ottsym{(}  \mathit{x}  \ottsym{)}  \ottsym{)}
    \end{align*}
    By the definition of the ownership function, $\Gamma  \ottsym{(}  \mathit{y}  \ottsym{)}  \ottsym{=}   \tau'  \TREF^{ \ottsym{1} } $ and $\Gamma  \ottsym{(}  \mathit{x}  \ottsym{)}  \ottsym{=}  \tau_{{\mathrm{1}}}  \ottsym{+}  \tau_{{\mathrm{2}}}$, we have:
    \begin{align*}
      \ottnt{O''_{{\mathrm{0}}}} & = \ottkw{own} \, \ottsym{(}  \ottnt{H}  \ottsym{,}  \ottnt{H}  \ottsym{(}  \ottnt{R}  \ottsym{(}  \mathit{y}  \ottsym{)}  \ottsym{)}  \ottsym{,}  \tau'  \ottsym{)}  \ottsym{+}  \ottsym{\{}  \ottmv{a}  \mapsto  \ottsym{1}  \ottsym{\}}  \ottsym{+}  \ottkw{own} \, \ottsym{(}  \ottnt{H}  \ottsym{,}  \ottnt{R}  \ottsym{(}  \mathit{x}  \ottsym{)}  \ottsym{,}  \tau_{{\mathrm{1}}}  \ottsym{+}  \tau_{{\mathrm{2}}}  \ottsym{)} \\
      \ottnt{O''_{{\mathrm{1}}}} & = \ottkw{own} \, \ottsym{(}  \ottnt{H'}  \ottsym{,}  \ottnt{H'}  \ottsym{(}  \ottnt{R}  \ottsym{(}  \mathit{y}  \ottsym{)}  \ottsym{)}  \ottsym{,}  \tau_{{\mathrm{2}}}  \ottsym{)}  \ottsym{+}  \ottsym{\{}  \ottmv{a}  \mapsto  \ottsym{1}  \ottsym{\}}  \ottsym{+}  \ottkw{own} \, \ottsym{(}  \ottnt{H'}  \ottsym{,}  \ottnt{R}  \ottsym{(}  \mathit{x}  \ottsym{)}  \ottsym{,}  \tau_{{\mathrm{1}}}  \ottsym{)}
    \end{align*}
    As $\ottkw{Own} \, \ottsym{(}  \ottnt{H}  \ottsym{,}  \ottnt{R}  \ottsym{,}  \Gamma  \ottsym{)}  \ottsym{(}  \ottmv{a}  \ottsym{)}  \le  \ottsym{1}$ (from $\ottkw{Cons} \, \ottsym{(}  \ottnt{H}  \ottsym{,}  \ottnt{R}  \ottsym{,}  \Gamma  \ottsym{)}$) and from
    \begin{align*}
      \ottkw{Own} \, \ottsym{(}  \ottnt{H}  \ottsym{,}  \ottnt{R}  \ottsym{,}  \Gamma  \ottsym{)}  \ottsym{(}  \ottmv{a}  \ottsym{)} & = \ottnt{O'_{{\mathrm{0}}}}  \ottsym{(}  \ottmv{a}  \ottsym{)}  \ottsym{+}  \ottnt{O''_{{\mathrm{0}}}}  \ottsym{(}  \ottmv{a}  \ottsym{)} \\
                           & = \ottnt{O'_{{\mathrm{0}}}}  \ottsym{(}  \ottmv{a}  \ottsym{)}  \ottsym{+}  \ottkw{own} \, \ottsym{(}  \ottnt{H}  \ottsym{,}  \ottnt{H}  \ottsym{(}  \ottnt{R}  \ottsym{(}  \mathit{y}  \ottsym{)}  \ottsym{)}  \ottsym{,}  \tau'  \ottsym{)}  \ottsym{(}  \ottmv{a}  \ottsym{)}  \ottsym{+}  \ottsym{1}  \ottsym{+}  \ottkw{own} \, \ottsym{(}  \ottnt{H}  \ottsym{,}  \ottnt{R}  \ottsym{(}  \mathit{x}  \ottsym{)}  \ottsym{,}  \tau_{{\mathrm{1}}}  \ottsym{+}  \tau_{{\mathrm{2}}}  \ottsym{)}  \ottsym{(}  \ottmv{a}  \ottsym{)} \\
                           & = 1
    \end{align*}
    we have that:
    \begin{align*}
      \ottkw{own} \, \ottsym{(}  \ottnt{H}  \ottsym{,}  \ottnt{H}  \ottsym{(}  \ottnt{R}  \ottsym{(}  \mathit{y}  \ottsym{)}  \ottsym{)}  \ottsym{,}  \tau'  \ottsym{)}  \ottsym{(}  \ottmv{a}  \ottsym{)} & = \ottkw{own} \, \ottsym{(}  \ottnt{H}  \ottsym{,}  \ottnt{R}  \ottsym{(}  \mathit{x}  \ottsym{)}  \ottsym{,}  \tau_{{\mathrm{1}}}  \ottsym{+}  \tau_{{\mathrm{2}}}  \ottsym{)}  \ottsym{(}  \ottmv{a}  \ottsym{)} \\
                               & = \ottnt{O'_{{\mathrm{0}}}}  \ottsym{(}  \ottmv{a}  \ottsym{)}  \ottsym{=}  \ottsym{0}
    \end{align*}

    We now show that $ \ottnt{H}   \approx _ \ottmv{a}   \ottnt{H'} $. The first two conditions are clear, so it
    remains to show that, for any $n$, $ \ottnt{H} \vdash   \ottmv{a}  \Downarrow  n $ iff $ \ottnt{H'} \vdash   \ottmv{a}  \Downarrow  n $.
    From \Cref{lem:sat-implies-shape-cons}, we have $ \ottnt{H} \vdash   \ottmv{a}  \Downarrow   |  \tau'  \TREF^{ \ottsym{1} }  |  $, and
    a proof by contradiction gives that $ |  \tau'  \TREF^{ \ottsym{1} }  | $ is the only such $n$
    for which $ \ottnt{H} \vdash   \ottmv{a}  \Downarrow  n $. We now argue the forward case for the bi-implication,
    the backwards case follows similar reasoning.

    Given $ \ottnt{H} \vdash   \ottmv{a}  \Downarrow   |  \tau'  \TREF^{ \ottsym{1} }  |  $, we  must show $ \ottnt{H}  \ottsym{\{}  \ottmv{a}  \hookleftarrow  \ottnt{R}  \ottsym{(}  \mathit{x}  \ottsym{)}  \ottsym{\}} \vdash   \ottmv{a}  \Downarrow   |  \tau_{{\mathrm{2}}}  \TREF^{ \ottsym{1} }  |  $, for which it suffices to show
    $ \ottnt{H}  \ottsym{\{}  \ottmv{a}  \hookleftarrow  \ottnt{R}  \ottsym{(}  \mathit{x}  \ottsym{)}  \ottsym{\}} \vdash   \ottnt{R}  \ottsym{(}  \mathit{x}  \ottsym{)}  \Downarrow   | \tau_{{\mathrm{2}}} |  $. 
    From our requirement that $\tau'$ and $\tau_{{\mathrm{2}}}$
    (and therefore $\tau_{{\mathrm{1}}}  \ottsym{+}  \tau_{{\mathrm{2}}}$) have similar shapes, we have $ | \tau' |   \ottsym{=}   | \tau_{{\mathrm{2}}} |   \ottsym{=}   | \tau_{{\mathrm{1}}}  \ottsym{+}  \tau_{{\mathrm{2}}} | $.
    By inverting the well-typing of the input configuration, we must have $ \ottkw{SATv} ( \ottnt{H} , \ottnt{R} , \ottnt{R}  \ottsym{(}  \mathit{x}  \ottsym{)} , \tau_{{\mathrm{1}}}  \ottsym{+}  \tau_{{\mathrm{2}}} ) $,
    thus by \Cref{lem:sat-implies-shape-cons} we must have $ \ottnt{H} \vdash   \ottnt{R}  \ottsym{(}  \mathit{x}  \ottsym{)}  \Downarrow   | \tau_{{\mathrm{2}}} |  $.
    As $ | \tau_{{\mathrm{2}}} |  \, \ottsym{=} \,  | \tau' |  <  |  \tau'  \TREF^{ \ottsym{1} }  | $, $\ottmv{a}$ cannot be reachable from $\ottnt{R}  \ottsym{(}  \mathit{x}  \ottsym{)}$ in $\ottnt{H}$
    (otherwise we would have $\ottmv{a}$ reaches an integer along multiple heap paths of differing lengths,
    a clear contradiction).
    Then the value of $\ottmv{a}$ in $\ottnt{H}$ is irrelevant to the derivation of $ \ottnt{H} \vdash   \ottnt{R}  \ottsym{(}  \mathit{x}  \ottsym{)}  \Downarrow   | \tau_{{\mathrm{2}}} |  $,
    whereby $ \ottnt{H}  \ottsym{\{}  \ottmv{a}  \hookleftarrow  \ottnt{R}  \ottsym{(}  \mathit{x}  \ottsym{)}  \ottsym{\}} \vdash   \ottnt{R}  \ottsym{(}  \mathit{x}  \ottsym{)}  \Downarrow   | \tau_{{\mathrm{2}}} |  $ must hold.

    Then, it's to show $\forall \,  \ottmv{a'}  \in \DOM( \ottnt{H'} )   \ottsym{.}  \ottkw{Own} \, \ottsym{(}  \ottnt{H'}  \ottsym{,}  \ottnt{R}  \ottsym{,}  \Gamma''  \ottsym{)}  \ottsym{(}  \ottmv{a'}  \ottsym{)}  \ottsym{=}  \ottsym{(}  \ottnt{O'_{{\mathrm{1}}}}  \ottsym{+}  \ottnt{O''_{{\mathrm{1}}}}  \ottsym{)}  \ottsym{(}  \ottmv{a'}  \ottsym{)}  \ottsym{=}  \ottnt{O'_{{\mathrm{1}}}}  \ottsym{(}  \ottmv{a'}  \ottsym{)}  \ottsym{+}  \ottnt{O''_{{\mathrm{1}}}}  \ottsym{(}  \ottmv{a'}  \ottsym{)} \, \le \, \ottsym{1}$.
    For every $ \mathit{z}  \in    \DOM( \Gamma )   \setminus  \ottsym{\{}  \mathit{y}  \ottsym{,}  \mathit{x}  \ottsym{\}}  $ (and similarly for $\Gamma''$), we have
    $\Gamma  \ottsym{(}  \mathit{z}  \ottsym{)}  \ottsym{=}  \Gamma''  \ottsym{(}  \mathit{z}  \ottsym{)}$. Further, from $\ottnt{O'_{{\mathrm{0}}}}  \ottsym{(}  \ottmv{a}  \ottsym{)}  \ottsym{=}  \ottsym{0}$ above, we must have $\ottkw{own} \, \ottsym{(}  \ottnt{H}  \ottsym{,}  \ottnt{R}  \ottsym{(}  \mathit{z}  \ottsym{)}  \ottsym{,}  \Gamma  \ottsym{(}  \mathit{z}  \ottsym{)}  \ottsym{)}  \ottsym{(}  \ottmv{a}  \ottsym{)}  \ottsym{=}  \ottsym{0}$
    for all such $\mathit{z}$. As $ \ottnt{H}   \approx _ \ottmv{a}   \ottnt{H'} $, by \Cref{lem:heapop}, we have that $\ottnt{O'_{{\mathrm{0}}}}  \ottsym{=}  \ottnt{O'_{{\mathrm{1}}}}$.
    Then, from $\forall \,  \ottmv{a'}  \in \DOM( \ottnt{H} )   \ottsym{.}  \ottkw{Own} \, \ottsym{(}  \ottnt{H}  \ottsym{,}  \ottnt{R}  \ottsym{,}  \Gamma  \ottsym{)}  \ottsym{(}  \ottmv{a'}  \ottsym{)}  \ottsym{=}  \ottnt{O'_{{\mathrm{0}}}}  \ottsym{(}  \ottmv{a'}  \ottsym{)}  \ottsym{+}  \ottnt{O''_{{\mathrm{0}}}}  \ottsym{(}  \ottmv{a'}  \ottsym{)} \, \le \, \ottsym{1}$, it suffices to show
    that $\forall \,  \ottmv{a'}  \in \DOM( \ottnt{H} )   \ottsym{.}  \ottnt{O''_{{\mathrm{1}}}}  \ottsym{(}  \ottmv{a'}  \ottsym{)}  \le  \ottnt{O''_{{\mathrm{0}}}}  \ottsym{(}  \ottmv{a'}  \ottsym{)}$.
       
    We first consider the case for $\ottmv{a}$:
    \begin{align*}
      \ottnt{O''_{{\mathrm{1}}}}  \ottsym{(}  \ottmv{a}  \ottsym{)} & = \ottkw{own} \, \ottsym{(}  \ottnt{H'}  \ottsym{,}  \ottnt{H'}  \ottsym{(}  \ottnt{R}  \ottsym{(}  \mathit{y}  \ottsym{)}  \ottsym{)}  \ottsym{,}  \tau_{{\mathrm{2}}}  \ottsym{)}  \ottsym{(}  \ottmv{a}  \ottsym{)}  \ottsym{+}  \ottkw{own} \, \ottsym{(}  \ottnt{H'}  \ottsym{,}  \ottnt{R}  \ottsym{(}  \mathit{x}  \ottsym{)}  \ottsym{,}  \tau_{{\mathrm{1}}}  \ottsym{)}  \ottsym{(}  \ottmv{a}  \ottsym{)}  \ottsym{+}  \ottsym{1} \\
      \ottnt{O''_{{\mathrm{0}}}}  \ottsym{(}  \ottmv{a}  \ottsym{)} & = \ottkw{own} \, \ottsym{(}  \ottnt{H}  \ottsym{,}  \ottnt{R}  \ottsym{(}  \mathit{x}  \ottsym{)}  \ottsym{,}  \tau_{{\mathrm{1}}}  \ottsym{+}  \tau_{{\mathrm{2}}}  \ottsym{)}  \ottsym{(}  \ottmv{a}  \ottsym{)}  \ottsym{+}  \ottkw{own} \, \ottsym{(}  \ottnt{H}  \ottsym{,}  \ottnt{H}  \ottsym{(}  \ottnt{R}  \ottsym{(}  \mathit{y}  \ottsym{)}  \ottsym{)}  \ottsym{,}  \tau'  \ottsym{)}  \ottsym{(}  \ottmv{a}  \ottsym{)}  \ottsym{+}  \ottsym{1}
    \end{align*}
    From above, we have $\ottkw{own} \, \ottsym{(}  \ottnt{H}  \ottsym{,}  \ottnt{R}  \ottsym{(}  \mathit{x}  \ottsym{)}  \ottsym{,}  \tau_{{\mathrm{2}}}  \ottsym{+}  \tau_{{\mathrm{1}}}  \ottsym{)}  \ottsym{(}  \ottmv{a}  \ottsym{)}  \ottsym{=}  \ottkw{own} \, \ottsym{(}  \ottnt{H}  \ottsym{,}  \ottnt{H}  \ottsym{(}  \ottnt{R}  \ottsym{(}  \mathit{y}  \ottsym{)}  \ottsym{)}  \ottsym{,}  \tau'  \ottsym{)}  \ottsym{(}  \ottmv{a}  \ottsym{)}  \ottsym{=}  \ottsym{0}$.
    By \Cref{lem:heapop} and $ \ottnt{H}   \approx _ \ottmv{a}   \ottnt{H'} $, we have $\ottkw{own} \, \ottsym{(}  \ottnt{H}  \ottsym{,}  \ottnt{R}  \ottsym{(}  \mathit{x}  \ottsym{)}  \ottsym{,}  \tau_{{\mathrm{2}}}  \ottsym{+}  \tau_{{\mathrm{1}}}  \ottsym{)}  \ottsym{=}  \ottkw{own} \, \ottsym{(}  \ottnt{H'}  \ottsym{,}  \ottnt{R}  \ottsym{(}  \mathit{x}  \ottsym{)}  \ottsym{,}  \tau_{{\mathrm{2}}}  \ottsym{+}  \tau_{{\mathrm{1}}}  \ottsym{)}$.
    Also by \Cref{lem:ownadd}, we have $\ottkw{own} \, \ottsym{(}  \ottnt{H'}  \ottsym{,}  \ottnt{R}  \ottsym{(}  \mathit{x}  \ottsym{)}  \ottsym{,}  \tau_{{\mathrm{2}}}  \ottsym{+}  \tau_{{\mathrm{1}}}  \ottsym{)}  \ottsym{=}  \ottkw{own} \, \ottsym{(}  \ottnt{H'}  \ottsym{,}  \ottnt{R}  \ottsym{(}  \mathit{x}  \ottsym{)}  \ottsym{,}  \tau_{{\mathrm{1}}}  \ottsym{)}  \ottsym{+}  \ottkw{own} \, \ottsym{(}  \ottnt{H'}  \ottsym{,}  \ottnt{R}  \ottsym{(}  \mathit{x}  \ottsym{)}  \ottsym{,}  \tau_{{\mathrm{2}}}  \ottsym{)}$.
    From $\ottnt{H'}  \ottsym{(}  \ottnt{R}  \ottsym{(}  \mathit{y}  \ottsym{)}  \ottsym{)} \, \ottsym{=} \, \ottnt{R}  \ottsym{(}  \mathit{x}  \ottsym{)}$, we therefore have $\ottkw{own} \, \ottsym{(}  \ottnt{H'}  \ottsym{,}  \ottnt{H'}  \ottsym{(}  \ottnt{R}  \ottsym{(}  \mathit{y}  \ottsym{)}  \ottsym{)}  \ottsym{,}  \tau_{{\mathrm{2}}}  \ottsym{)}  \ottsym{(}  \ottmv{a}  \ottsym{)}  \ottsym{=}  \ottkw{own} \, \ottsym{(}  \ottnt{H'}  \ottsym{,}  \ottnt{R}  \ottsym{(}  \mathit{x}  \ottsym{)}  \ottsym{,}  \tau_{{\mathrm{1}}}  \ottsym{)}  \ottsym{(}  \ottmv{a}  \ottsym{)}  \ottsym{=}  \ottsym{0}$,
    and thus:
    \[
      \ottnt{O''_{{\mathrm{1}}}}  \ottsym{(}  \ottmv{a}  \ottsym{)}  \ottsym{=}  \ottkw{own} \, \ottsym{(}  \ottnt{H'}  \ottsym{,}  \ottnt{H'}  \ottsym{(}  \ottnt{R}  \ottsym{(}  \mathit{y}  \ottsym{)}  \ottsym{)}  \ottsym{,}  \tau_{{\mathrm{2}}}  \ottsym{)}  \ottsym{(}  \ottmv{a}  \ottsym{)}  \ottsym{+}  \ottkw{own} \, \ottsym{(}  \ottnt{H'}  \ottsym{,}  \ottnt{R}  \ottsym{(}  \mathit{x}  \ottsym{)}  \ottsym{,}  \tau_{{\mathrm{1}}}  \ottsym{)}  \ottsym{(}  \ottmv{a}  \ottsym{)}  \ottsym{+}  \ottsym{1}  \ottsym{=}  \ottsym{1}  \ottsym{=}  \ottnt{O''_{{\mathrm{0}}}}  \ottsym{(}  \ottmv{a}  \ottsym{)}
    \]
    
    Next, consider some $\ottmv{a} \, \neq \, \ottmv{a'}$;
    \begin{align*}
      \ottnt{O''_{{\mathrm{1}}}}  \ottsym{(}  \ottmv{a'}  \ottsym{)} & = \ottkw{own} \, \ottsym{(}  \ottnt{H'}  \ottsym{,}  \ottnt{H'}  \ottsym{(}  \ottnt{R}  \ottsym{(}  \mathit{y}  \ottsym{)}  \ottsym{)}  \ottsym{,}  \tau_{{\mathrm{2}}}  \ottsym{)}  \ottsym{(}  \ottmv{a'}  \ottsym{)}  \ottsym{+}  \ottkw{own} \, \ottsym{(}  \ottnt{H'}  \ottsym{,}  \ottnt{R}  \ottsym{(}  \mathit{x}  \ottsym{)}  \ottsym{,}  \tau_{{\mathrm{1}}}  \ottsym{)}  \ottsym{(}  \ottmv{a'}  \ottsym{)} \\
      \ottnt{O''_{{\mathrm{0}}}}  \ottsym{(}  \ottmv{a'}  \ottsym{)} & = \ottkw{own} \, \ottsym{(}  \ottnt{H}  \ottsym{,}  \ottnt{R}  \ottsym{(}  \mathit{x}  \ottsym{)}  \ottsym{,}  \tau_{{\mathrm{2}}}  \ottsym{+}  \tau_{{\mathrm{1}}}  \ottsym{)}  \ottsym{(}  \ottmv{a'}  \ottsym{)}  \ottsym{+}  \ottkw{own} \, \ottsym{(}  \ottnt{H}  \ottsym{,}  \ottnt{H}  \ottsym{(}  \ottnt{R}  \ottsym{(}  \mathit{y}  \ottsym{)}  \ottsym{)}  \ottsym{,}  \tau'  \ottsym{)}  \ottsym{(}  \ottmv{a'}  \ottsym{)}
    \end{align*}
    By reasoning similar to the case for $a = a'$, we have $\ottnt{O''_{{\mathrm{1}}}}  \ottsym{(}  \ottmv{a'}  \ottsym{)}  \le  \ottkw{own} \, \ottsym{(}  \ottnt{H}  \ottsym{,}  \ottnt{R}  \ottsym{(}  \mathit{x}  \ottsym{)}  \ottsym{,}  \tau_{{\mathrm{2}}}  \ottsym{+}  \tau_{{\mathrm{1}}}  \ottsym{)}  \ottsym{(}  \ottmv{a'}  \ottsym{)}  \le  \ottnt{O''_{{\mathrm{0}}}}  \ottsym{(}  \ottmv{a'}  \ottsym{)}$.
    We therefore conclude that $\forall \,  \ottmv{a'}  \in \DOM( \ottnt{H'} )   \ottsym{.}  \ottkw{Own} \, \ottsym{(}  \ottnt{H'}  \ottsym{,}  \ottnt{R}  \ottsym{,}  \Gamma''  \ottsym{)}  \ottsym{(}  \ottmv{a'}  \ottsym{)}  \le  \ottsym{1}$.
  \end{rneqncase} % R-Assign
  
  \begin{rneqncase}{R-Alias}{
     \vdash_{\mathit{conf} }^D   \tuple{ \ottnt{H} ,  \ottnt{R} ,  \oldvec{F} ,  \ottnt{E}  \ottsym{[}   \ALIAS( \mathit{x}  =  \mathit{y} ) \SEQ  \ottnt{e}   \ottsym{]} }  \\
      \tuple{ \ottnt{H} ,  \ottnt{R} ,  \oldvec{F} ,  \ottnt{E}  \ottsym{[}   \ALIAS( \mathit{x}  =  \mathit{y} ) \SEQ  \ottnt{e}   \ottsym{]} }     \longrightarrow _{ \ottnt{D} }     \tuple{ \ottnt{H} ,  \ottnt{R} ,  \oldvec{F} ,  \ottnt{E}  \ottsym{[}  \ottnt{e}  \ottsym{]} }  \\
    \ottnt{R}  \ottsym{(}  \mathit{x}  \ottsym{)} \, \ottsym{=} \, \ottnt{R}  \ottsym{(}  \mathit{y}  \ottsym{)}
    }
    By inversion (see \rn{R-Deref}) we have for some $\Gamma$ that:
    \begin{align*}
      &  \Theta   \mid   \oldvec{\ell}   \mid   \Gamma  \ottsym{[}  \mathit{x}  \ottsym{:}   \tau_{{\mathrm{1}}}  \TREF^{ r_{{\mathrm{1}}} }   \ottsym{]}  \ottsym{[}  \mathit{y}  \ottsym{:}   \tau_{{\mathrm{2}}}  \TREF^{ r_{{\mathrm{2}}} }   \ottsym{]}   \vdash    \ALIAS( \mathit{x}  =  \mathit{y} ) \SEQ  \ottnt{e}   :  \tau   \produces   \Gamma'  \\
      &  \Theta   \mid   \oldvec{\ell}   \mid   \Gamma  \ottsym{[}  \mathit{x}  \hookleftarrow   \tau'_{{\mathrm{1}}}  \TREF^{ r'_{{\mathrm{1}}} }   \ottsym{]}  \ottsym{[}  \mathit{y}  \hookleftarrow   \tau'_{{\mathrm{2}}}  \TREF^{ r'_{{\mathrm{2}}} }   \ottsym{]}   \vdash   \ottnt{e}  :  \tau   \produces   \Gamma'  \\
      &   \tau_{{\mathrm{1}}}  \TREF^{ r_{{\mathrm{1}}} }   \ottsym{+}  \tau_{{\mathrm{2}}}  \TREF^{ r_{{\mathrm{2}}} }   \approx    \tau'_{{\mathrm{1}}}  \TREF^{ r'_{{\mathrm{1}}} }   \ottsym{+}  \tau'_{{\mathrm{2}}}  \TREF^{ r'_{{\mathrm{2}}} }  \\
      & \ottkw{Cons} \, \ottsym{(}  \ottnt{H}  \ottsym{,}  \ottnt{R}  \ottsym{,}  \Gamma  \ottsym{)}
    \end{align*}
    We give $\Gamma''  \ottsym{=}  \Gamma  \ottsym{[}  \mathit{x}  \hookleftarrow   \tau'_{{\mathrm{1}}}  \TREF^{ r'_{{\mathrm{1}}} }   \ottsym{]}  \ottsym{[}  \mathit{y}  \hookleftarrow   \tau'_{{\mathrm{2}}}  \TREF^{ r'_{{\mathrm{2}}} }   \ottsym{]}$, and must show
    $ \Theta   \mid   \oldvec{\ell}   \mid   \Gamma''   \vdash   \ottnt{e}  :  \tau   \produces   \Gamma' $ and $\ottkw{Cons} \, \ottsym{(}  \ottnt{H}  \ottsym{,}  \ottnt{R}  \ottsym{,}  \Gamma''  \ottsym{)}$.
    The first is immediate.
    
    To show $\ottkw{Cons} \, \ottsym{(}  \ottnt{H}  \ottsym{,}  \ottnt{R}  \ottsym{,}  \Gamma''  \ottsym{)}$ we first define:
    \begin{align*}
      \tau_{\ottmv{p}\,{\mathrm{1}}} = &  \tau_{{\mathrm{1}}}  \TREF^{ r_{{\mathrm{1}}} }  \\
      \tau_{\ottmv{p}\,{\mathrm{2}}} = &  \tau_{{\mathrm{2}}}  \TREF^{ r_{{\mathrm{2}}} }  \\
      \tau_{\ottmv{q}\,{\mathrm{1}}} = &  \tau'_{{\mathrm{1}}}  \TREF^{ r'_{{\mathrm{1}}} }  \\
      \tau_{\ottmv{q}\,{\mathrm{2}}} = &  \tau'_{{\mathrm{2}}}  \TREF^{ r'_{{\mathrm{2}}} }  \\
      \tau_{\ottmv{q}} = & \tau_{\ottmv{q}\,{\mathrm{1}}}  \ottsym{+}  \tau_{\ottmv{q}\,{\mathrm{2}}} \\
      \tau_{\ottmv{p}} = & \tau_{\ottmv{p}\,{\mathrm{1}}}  \ottsym{+}  \tau_{\ottmv{p}\,{\mathrm{2}}}
    \end{align*}
    We thus have $\tau_{\ottmv{q}}  \approx  \tau_{\ottmv{p}}$.
    
    We know that $\ottkw{Cons} \, \ottsym{(}  \ottnt{H}  \ottsym{,}  \ottnt{R}  \ottsym{,}  \Gamma  \ottsym{)}$, from which we have $\ottkw{SAT} \, \ottsym{(}  \ottnt{H}  \ottsym{,}  \ottnt{R}  \ottsym{,}  \Gamma  \ottsym{)}$, in particular
    $ \ottkw{SATv} ( \ottnt{H} , \ottnt{R} , \ottnt{R}  \ottsym{(}  \mathit{y}  \ottsym{)} ,  \Gamma  \ottsym{(}  \mathit{y}  \ottsym{)}  \ottsym{=}  \tau_{{\mathrm{2}}}  \TREF^{ r_{{\mathrm{2}}} }  ) $ and $ \ottkw{SATv} ( \ottnt{H} , \ottnt{R} , \ottnt{R}  \ottsym{(}  \mathit{x}  \ottsym{)} ,  \Gamma  \ottsym{(}  \mathit{x}  \ottsym{)}  \ottsym{=}  \tau_{{\mathrm{1}}}  \TREF^{ r_{{\mathrm{1}}} }  ) $.
    From $\tau_{\ottmv{p}\,{\mathrm{1}}}  \ottsym{+}  \tau_{\ottmv{p}\,{\mathrm{2}}}  \ottsym{=}  \tau_{\ottmv{p}}$ and \Cref{lem:satadd}, we have
    $ \ottkw{SATv} ( \ottnt{H} , \ottnt{R} , \ottnt{v} , \tau_{\ottmv{p}\,{\mathrm{1}}} ) $ and $ \ottkw{SATv} ( \ottnt{H} , \ottnt{R} , \ottnt{v} , \tau_{\ottmv{p}\,{\mathrm{2}}} ) $ imply $ \ottkw{SATv} ( \ottnt{H} , \ottnt{R} , \ottnt{v} , \tau_{\ottmv{p}} ) $, where $\ottnt{v}  \ottsym{=}  \ottnt{H}  \ottsym{(}  \ottnt{R}  \ottsym{(}  \mathit{y}  \ottsym{)}  \ottsym{)}  \ottsym{=}  \ottnt{H}  \ottsym{(}  \ottnt{R}  \ottsym{(}  \mathit{x}  \ottsym{)}  \ottsym{)}$.
    From $\tau_{\ottmv{q}}  \approx  \tau_{\ottmv{p}}$ and \Cref{lem:sattosat}, we have that $ \ottkw{SATv} ( \ottnt{H} , \ottnt{R} , \ottnt{v} , \tau_{\ottmv{p}} ) $ implies $ \ottkw{SATv} ( \ottnt{H} , \ottnt{R} , \ottnt{v} , \tau_{\ottmv{q}} ) $.
    From \Cref{lem:satadd} we also have that $ \ottkw{SATv} ( \ottnt{H} , \ottnt{R} , \ottnt{v} , \tau_{\ottmv{q}} ) $ implies $ \ottkw{SATv} ( \ottnt{H} , \ottnt{R} , \ottnt{v} , \tau_{\ottmv{q}\,{\mathrm{1}}} ) $ and $ \ottkw{SATv} ( \ottnt{H} , \ottnt{R} , \ottnt{v} , \tau_{\ottmv{q}\,{\mathrm{2}}} ) $, where
    again $\ottnt{v}  \ottsym{=}  \ottnt{H}  \ottsym{(}  \ottnt{R}  \ottsym{(}  \mathit{y}  \ottsym{)}  \ottsym{)}  \ottsym{=}  \ottnt{H}  \ottsym{(}  \ottnt{R}  \ottsym{(}  \mathit{x}  \ottsym{)}  \ottsym{)}$.

    Then from the reasoning above, the refinements of $\tau_{\ottmv{q}\,{\mathrm{1}}}$ and $\tau_{\ottmv{q}\,{\mathrm{2}}}$ are valid and $\ottkw{Cons} \, \ottsym{(}  \ottnt{H}  \ottsym{,}  \ottnt{R}  \ottsym{,}  \Gamma''  \ottsym{)}$ holds.

    Then, it's to show $\forall \,  \ottmv{a}  \in \DOM( \ottnt{H} )   \ottsym{.}  \ottkw{Own} \, \ottsym{(}  \ottnt{H}  \ottsym{,}  \ottnt{R}  \ottsym{,}  \Gamma''  \ottsym{)}  \ottsym{(}  \ottmv{a}  \ottsym{)}  \le  \ottsym{1}$.
    To prove that $\ottkw{Own} \, \ottsym{(}  \ottnt{H}  \ottsym{,}  \ottnt{R}  \ottsym{,}  \Gamma  \ottsym{)}  \ottsym{=}  \ottkw{Own} \, \ottsym{(}  \ottnt{H}  \ottsym{,}  \ottnt{R}  \ottsym{,}  \Gamma''  \ottsym{)}$ follows from:
    \begin{align*}
      &\ottkw{own} \, \ottsym{(}  \ottnt{H}  \ottsym{,}  \ottnt{R}  \ottsym{(}  \mathit{x}  \ottsym{)}  \ottsym{,}   \tau_{{\mathrm{1}}}  \TREF^{ r_{{\mathrm{1}}} }   \ottsym{)}  \ottsym{+}  \ottkw{own} \, \ottsym{(}  \ottnt{H}  \ottsym{,}  \ottnt{R}  \ottsym{(}  \mathit{y}  \ottsym{)}  \ottsym{,}   \tau_{{\mathrm{2}}}  \TREF^{ r_{{\mathrm{2}}} }   \ottsym{)} =  \\
      &\,\,\,\,\ottkw{own} \, \ottsym{(}  \ottnt{H}  \ottsym{,}  \ottnt{R}  \ottsym{(}  \mathit{x}  \ottsym{)}  \ottsym{,}   \tau'_{{\mathrm{1}}}  \TREF^{ r'_{{\mathrm{1}}} }   \ottsym{)}  \ottsym{+}  \ottkw{own} \, \ottsym{(}  \ottnt{H}  \ottsym{,}  \ottnt{R}  \ottsym{(}  \mathit{y}  \ottsym{)}  \ottsym{,}   \tau'_{{\mathrm{2}}}  \TREF^{ r'_{{\mathrm{2}}} }   \ottsym{)}
    \end{align*}
    which follows immediately from the conditions $  \tau_{{\mathrm{1}}}  \TREF^{ r_{{\mathrm{1}}} }   \ottsym{+}  \tau_{{\mathrm{2}}}  \TREF^{ r_{{\mathrm{2}}} }   \approx    \tau'_{{\mathrm{1}}}  \TREF^{ r'_{{\mathrm{1}}} }   \ottsym{+}  \tau'_{{\mathrm{2}}}  \TREF^{ r'_{{\mathrm{2}}} } $, $\ottnt{R}  \ottsym{(}  \mathit{x}  \ottsym{)} \, \ottsym{=} \, \ottnt{R}  \ottsym{(}  \mathit{y}  \ottsym{)}$, and \Cref{lem:ownadd,lem:ownequiv-preserv}.
  \end{rneqncase} % R-Alias

  \begin{rncase}{R-AliasPtr}
    By reasoning similar to the \rn{R-Alias} case.
  \end{rncase}
    
  
  \begin{rncase}{R-AliasFail,R-AliasPtrFail} % DONE
    The result configuration $ \mathbf{AliasFail} $ is trivially well-typed.
  \end{rncase}
  
  \begin{rneqncase}{R-Assert}{ % DONE
     \vdash_{\mathit{conf} }^D   \tuple{ \ottnt{H} ,  \ottnt{R} ,  \oldvec{F} ,  \ottnt{E}  \ottsym{[}   \ASSERT( \varphi ) \SEQ  \ottnt{e}   \ottsym{]} }  \andalso \Gamma  \models  \ottsym{[}  \ottnt{R}  \ottsym{]} \, \varphi\\
      \tuple{ \ottnt{H} ,  \ottnt{R} ,  \oldvec{F} ,  \ottnt{E}  \ottsym{[}   \ASSERT( \varphi ) \SEQ  \ottnt{e}   \ottsym{]} }     \longrightarrow _{ \ottnt{D} }     \tuple{ \ottnt{H} ,  \ottnt{R} ,  \oldvec{F} ,  \ottnt{E}  \ottsym{[}  \ottnt{e}  \ottsym{]} }  \\
    }
    By inversion (see \rn{R-Deref}) we can obtain $ \Theta   \mid   \oldvec{\ell}   \mid   \Gamma   \vdash    \ASSERT( \varphi ) \SEQ  \ottnt{e}   :  \tau   \produces   \Gamma' 
$ and $ \Theta   \mid   \oldvec{\ell}   \mid   \Gamma   \vdash   \ottnt{e}  :  \tau   \produces   \Gamma' $, and
    the result follows immediately by taking $\Gamma''  \ottsym{=}  \Gamma$.
  \end{rneqncase} % R-Assert
  
  \begin{rneqncase}{R-AssertFail}{ % DONE
       \vdash_{\mathit{conf} }^D   \tuple{ \ottnt{H} ,  \ottnt{R} ,  \oldvec{F} ,  \ottnt{E}  \ottsym{[}   \ASSERT( \varphi ) \SEQ  \ottnt{e}   \ottsym{]} }   \\
        \tuple{ \ottnt{H} ,  \ottnt{R} ,  \oldvec{F} ,  \ottnt{E}  \ottsym{[}   \ASSERT( \varphi ) \SEQ  \ottnt{e}   \ottsym{]} }     \longrightarrow _{ \ottnt{D} }     \mathbf{AssertFail}  \\
       \Theta   \mid   \oldvec{\ell}   \mid   \Gamma   \vdash    \ASSERT( \varphi ) \SEQ  \ottnt{e}   :  \tau   \produces   \Gamma' 
    }
    By inversion (see the \rn{R-Deref} case) we have that
    $\Gamma  \models  \varphi$, i.e., $\models   \sem{ \Gamma }   \implies  \varphi$,
    for some $\Gamma$ such that $\ottkw{Cons} \, \ottsym{(}  \ottnt{H}  \ottsym{,}  \ottnt{R}  \ottsym{,}  \Gamma  \ottsym{)}$.
    From \Cref{lem:sat-implies-gamma} we therefore have $\models  \ottsym{[}  \ottnt{R}  \ottsym{]} \,  \sem{ \Gamma } $.
    From the precondition of \rn{R-AssertFail} we have that
    $\not\models  \ottsym{[}  \ottnt{R}  \ottsym{]} \, \varphi$. But from $\models   \sem{ \Gamma }   \implies  \varphi$ and $\models  \ottsym{[}  \ottnt{R}  \ottsym{]} \,  \sem{ \Gamma } $
    we can conclude that $\models  \ottsym{[}  \ottnt{R}  \ottsym{]} \, \varphi$, yielding a contradiction.
    We therefore conclude that this case is impossible.
  \end{rneqncase} % R-AssertFail
  
  \begin{rneqncase}{R-Call}{
       \vdash_{\mathit{conf} }^D   \tuple{ \ottnt{H} ,  \ottnt{R} ,  \oldvec{F} ,  \ottnt{E}  \ottsym{[}   \LET  \mathit{x}  =   \mathit{f} ^ \ell (  \mathit{y_{{\mathrm{1}}}} ,\ldots, \mathit{y_{\ottmv{n}}}  )   \IN  \ottnt{e'}   \ottsym{]} }  \\
       \mathit{f}  \mapsto  \ottsym{(}  \mathit{x_{{\mathrm{1}}}}  \ottsym{,} \, .. \, \ottsym{,}  \mathit{x_{\ottmv{n}}}  \ottsym{)}  \ottnt{e}  \in  \ottnt{D}  \\
       \begin{array}{l}  \tuple{ \ottnt{H} ,  \ottnt{R} ,  \oldvec{F} ,  \ottnt{E}  \ottsym{[}   \LET  \mathit{x}  =   \mathit{f} ^ \ell (  \mathit{y_{{\mathrm{1}}}} ,\ldots, \mathit{y_{\ottmv{n}}}  )   \IN  \ottnt{e'}   \ottsym{]} }   \\ \quad   \longrightarrow _{ \ottnt{D} }     \tuple{ \ottnt{H} ,  \ottnt{R} ,   \ottnt{E} [\LET  \mathit{x}  =   \HOLE^ \ell   \IN  \ottnt{e'}  ]   \ottsym{:}  \oldvec{F} ,     [  \mathit{y_{{\mathrm{1}}}}  /  \mathit{x_{{\mathrm{1}}}}  ]  \cdots  [  \mathit{y_{\ottmv{n}}}  /  \mathit{x_{\ottmv{n}}}  ]     \ottnt{e}  }  \end{array}  \\
    }
    We must show that
    $ \vdash_{\mathit{conf} }^D   \tuple{ \ottnt{H} ,  \ottnt{R} ,   \ottnt{E} [\LET  \mathit{x}  =   \HOLE^ \ell   \IN  \ottnt{e'}  ]   \ottsym{:}  \oldvec{F} ,     [  \mathit{y_{{\mathrm{1}}}}  /  \mathit{x_{{\mathrm{1}}}}  ]  \cdots  [  \mathit{y_{\ottmv{n}}}  /  \mathit{x_{\ottmv{n}}}  ]     \ottnt{e}  }  $
    for some $\Gamma''$.
    
    By inversion on the configuration typing, we have that, for some $\Gamma$:
    \[
       \Theta   \mid   \oldvec{\ell}   \mid   \Gamma   \vdash   \ottnt{E}  \ottsym{[}   \LET  \mathit{x}  =   \mathit{f} ^ \ell (  \mathit{y_{{\mathrm{1}}}} ,\ldots, \mathit{y_{\ottmv{n}}}  )   \IN  \ottnt{e'}   \ottsym{]}  :  \tau_{\ottmv{n}}   \produces   \Gamma_{\ottmv{n}}  .
    \]
    By \Cref{lem:stack_var}, we then have for some $\tau$, and $\Gamma'$ that:
    \begin{align*}
      &  \Theta   \mid   \oldvec{\ell}   \mid   \Gamma   \vdash    \LET  \mathit{x}  =   \mathit{f} ^ \ell (  \mathit{y_{{\mathrm{1}}}} ,\ldots, \mathit{y_{\ottmv{n}}}  )   \IN  \ottnt{e'}   :  \tau   \produces   \Gamma'  \\
      & \Theta  \mid  \HOLE  \ottsym{:}  \tau  \produces  \Gamma'  \mid  \oldvec{\ell}  \vdash_{\mathit{ectx} }  \ottnt{E}  \ottsym{:}  \tau_{\ottmv{n}}  \produces  \Gamma_{\ottmv{n}}
    \end{align*}
    Taking $\tau_{{\mathrm{1}}}  \ottsym{=}  \tau, \Gamma_{{\mathrm{1}}}  \ottsym{=}  \Gamma', \Gamma_{{\mathrm{0}}}  \ottsym{=}  \Gamma, \Gamma_{{\mathrm{2}}}  \ottsym{=}  \Gamma_{\ottmv{n}}, \tau_{{\mathrm{2}}}  \ottsym{=}  \tau_{\ottmv{n}}$,
    by \Cref{lem:callfunc} we have, for some $\tau''', \Gamma'''$:
    \begin{align*}
      &  \Theta   \mid   \ell  \ottsym{:}  \oldvec{\ell}   \mid   \Gamma   \vdash    \sigma_{x}   \ottnt{e}   :  \tau'''   \produces   \Gamma'''  \\
      & \Theta  \mid  \HOLE  \ottsym{:}  \tau'''  \produces  \Gamma'''  \mid  \oldvec{\ell}  \vdash_{\mathit{ectx} }   \ottnt{E} [\LET  \mathit{x}  =   \HOLE^ \ell   \IN  \ottnt{e'}  ]   \ottsym{:}  \tau_{\ottmv{n}}  \produces  \Gamma_{\ottmv{n}}
    \end{align*}
    where:
    \begin{align*}
      \sigma_{x} & =   [  \mathit{y_{{\mathrm{1}}}}  /  \mathit{x_{{\mathrm{1}}}}  ]  \cdots  [  \mathit{y_{\ottmv{n}}}  /  \mathit{x_{\ottmv{n}}}  ]   \\
      \Theta  \ottsym{(}  \mathit{f}  \ottsym{)} & =  \forall  \lambda .\tuple{ \mathit{x_{{\mathrm{1}}}} \COL \tau_{\ottmv{i}} ,\dots, \mathit{x_{\ottmv{n}}} \COL \tau_{\ottmv{n}} }\ra\tuple{ \mathit{x_{{\mathrm{1}}}} \COL \tau'_{{\mathrm{1}}} ,\dots, \mathit{x_{\ottmv{n}}} \COL \tau'_{\ottmv{n}}  \mid  \tau_{\ottmv{p}} } 
    \end{align*}
    We therefore take $\Gamma''  \ottsym{=}  \Gamma$.
     
    We must also prove that $\forall i\in\set{1..n+1}.\Theta  \mid  \HOLE  \ottsym{:}  \tau_{\ottmv{i}}  \produces  \Gamma_{\ottmv{i}}  \mid  \oldvec{\ell}_{{\ottmv{i}-1}}  \vdash_{\mathit{ectx} }  \ottnt{E'_{{\ottmv{i}-1}}}  \ottsym{:}  \tau_{{\ottmv{i}-1}}  \produces  \Gamma_{{\ottmv{i}-1}}$  where $\ottnt{E'_{\ottmv{n}}}  \ottsym{=}  \ottnt{E}  \ottsym{[}   \LET  \mathit{x}  =   \HOLE^ \ell   \IN  \ottnt{e'}   \ottsym{]}$ and $\ottnt{E'_{\ottmv{i}}}  \ottsym{=}  \ottnt{E_{\ottmv{i}}} (0 \leq i < n)$,
    which can be divided into proving
    $\forall i\in\set{1..n}.\Theta  \mid  \HOLE  \ottsym{:}  \tau_{\ottmv{i}}  \produces  \Gamma_{\ottmv{i}}  \mid  \oldvec{\ell}_{{\ottmv{i}-1}}  \vdash_{\mathit{ectx} }  \ottnt{E'_{{\ottmv{i}-1}}}  \ottsym{:}  \tau_{{\ottmv{i}-1}}  \produces  \Gamma_{{\ottmv{i}-1}}$ and $\Theta  \mid  \HOLE  \ottsym{:}   \tau_{n+1}   \produces   \tenv_{n+1}   \mid  \oldvec{\ell}_{\ottmv{n}}  \vdash_{\mathit{ectx} }  \ottnt{E'_{\ottmv{n}}}  \ottsym{:}  \tau_{\ottmv{n}}  \produces  \Gamma_{\ottmv{n}}$. The first follows by inversion on $ \vdash_{\mathit{conf} }^D   \tuple{ \ottnt{H} ,  \ottnt{R} ,  \oldvec{F} ,  \ottnt{E}  \ottsym{[}   \LET  \mathit{x}  =   \mathit{f} ^ \ell (  \mathit{y_{{\mathrm{1}}}} ,\ldots, \mathit{y_{\ottmv{n}}}  )   \IN  \ottnt{e'}   \ottsym{]} }  $.
    To show the latter, we define $ \tenv_{n+1}   \ottsym{=}  \Gamma'''$ and $ \tau_{n+1}   \ottsym{=}  \tau'''$, whereby the
    well-typing holds from the result of applying \Cref{lem:callfunc} above.
    
    Finally, $\ottkw{Cons} \, \ottsym{(}  \ottnt{H}  \ottsym{,}  \ottnt{R}  \ottsym{,}  \Gamma''  \ottsym{)}$ follows immediately from $\ottkw{Cons} \, \ottsym{(}  \ottnt{H}  \ottsym{,}  \ottnt{R}  \ottsym{,}  \Gamma  \ottsym{)}$ and $\Gamma''  \ottsym{=}  \Gamma$.
  \end{rneqncase}
\end{proof}

%%% Local Variables:
%%% mode: latex
%%% TeX-master: t
%%% End:
