\section{Experiments}
\label{sec:eval}

We now present the results of preliminary experiments performed with the implementation
described in \Cref{sec:infr}. The goal of these experiments was to answer the following questions: \begin{inparaenum}[i)]
\item is the type system (and extensions of \Cref{sec:infr}) expressive enough to type and verify non-trivial programs? and
\item is type inference feasible?
\end{inparaenum}\looseness=-1

To answer these questions, we evaluated our prototype implementation
on two sets of benchmarks.\footnote{Our experiments and the \name source code are available
  at \url{https://www.fos.kuis.kyoto-u.ac.jp/projects/consort/}.} 
The first set is adapted from JayHorn \cite{kahsai2017quantified,kahsai2016jayhorn},
a verification tool for Java.
This test suite contains a combination of 82 safe and unsafe programs written in
Java. We chose this benchmark suite as, like \name, JayHorn
is concerned with the automated verification of
programs in a language with mutable, aliased memory cells.
Further, although some of their benchmark programs tested Java
specific features, most could be adapted
into our low-level language.
The tests we could adapt provide a comparison with existing state-of-the-art
verification techniques.
A detailed breakdown of the adapted benchmark suite can be found in \Cref{tab:breakdown}.

\begin{table}[t]
  \caption{Description of benchmark suite adapted from JayHorn. \textbf{Java} are programs
    that test Java-specific features. \textbf{Inc} are tests that cannot be handled by \name, e.g.,
    null checking, etc. \textbf{Bug} includes a ``safe'' program
    we discovered was actually incorrect.}
  \begin{center}
    \begin{tabular}{cccccc}\toprule
\textbf{Set} & \textbf{Orig.} & \textbf{Adapted} & \textbf{Java} & \textbf{Inc} & \textbf{Bug} \\ \midrule
Safe & 41 & 32 & 6 & 2 & 1
\\
Unsafe & 41 & 26 & 13 & 2 & 0
\end{tabular}

  \end{center}
  \label{tab:breakdown}
\end{table}

\begin{remark}
  The original JayHorn paper includes two additional benchmark sets, Mine Pump and CBMC.
  Both our tool and recent JayHorn versions time out on the Mine Pump benchmark. Further,
  the CBMC tests were either subsumed by our own test programs, tested Java specific
  features, or tested program synthesis functionality. We therefore omitted both of these
  benchmarks from our evaluation.
\end{remark}

The second benchmark set consists of data structure
implementations and microbenchmarks written directly in our low-level
imperative language. We developed this suite to
test the expressive power of our type system and inference.
The programs included in this suite are:
\begin{itemize}
\item \textbf{Array-List} Implementation of an unbounded list backed by an array.
\item \textbf{Sorted-List} Implementation of a mutable, sorted list maintained with an in-place insertion sort algorithm.
\item \textbf{Shuffle} Multiple live references are used to mutate the same location in program memory as in \Cref{exmp:shuffle-example}.
\item \textbf{Mut-List} Implementation of general linked lists with a clear operation.
\item \textbf{Array-Inv} A program which allocates a length $n$ array and writes the value $i$ at every index $i$.
\item \textbf{Intro2} The motivating program shown in \Cref{fig:hard-loop} in \Cref{sec:intro}.
\end{itemize}
We introduced unsafe mutations to these programs to
check our tool for unsoundness and translated these programs
into Java for further comparison with JayHorn.\looseness=-1

Our benchmarks and JayHorn's require a small number of trivially identified alias annotations.
The adapted JayHorn benchmarks contain a total of \jhaliascount{}
annotations; the most for any individual test was \jhmaxalias.
The number of annotations required for our benchmark suite are shown in column
\textbf{Ann.} of \Cref{tab:jh-results}.

We first ran \name on each program in our benchmark suite and ran version 0.7 of
JayHorn on the corresponding Java version. We recorded the final verification
result for both our tool and JayHorn. We also collected the
end-to-end runtime of \name for each test;
we do not give a performance comparison with
JayHorn given the many differences in target languages.
For the JayHorn suite, we first ran our tool on the adapted version of
each test program and ran JayHorn on the original Java version. We also did not
collect runtime information for this set of experiments because our goal
is a comparison of tool precision, not performance.
All tests were run on a machine with 16 GB RAM and 4 Intel i5 CPUs at 2GHz
and with a timeout of 60 seconds (the same timeout
was used in \cite{kahsai2017quantified}). We used \name's parallel backend (\Cref{sec:infr})
with Z3 version 4.8.4, HoICE version 1.8.1, and Eldarica version 2.0.1 and JayHorn's Eldarica backend.

\begin{table}[t]
  \caption{Comparison of \name to JayHorn on the benchmark set of \cite{kahsai2017quantified} (top) and our custom benchmark suite (bottom). \emph{T/O}
  indicates a time out.}
  \begin{center}
    \begin{tabular}{ccccccc}\toprule
& & \multicolumn{2}{l}{\textbf{\name}} & \multicolumn{3}{c}{\textbf{JayHorn}} \\
\cmidrule(lr){3-4} \cmidrule(lr){5-7}
\textbf{Set} & \textbf{N. Tests} & \emph{Correct} & \emph{T/O} & \emph{Correct} & \emph{T/O} & \emph{Imp.} \\ \midrule
\textbf{Safe} & 32 & 29 & 3 &  24 &  5 & 3\\
\textbf{Unsafe} & 26 & 26 & 0 &  19 &  0 & 7
\end{tabular}

    \begin{tabular}{lcccc|lcccc}\toprule
\textbf{Name} & \textbf{Safe?} & \textbf{Time(s)} & \textbf{Ann} & \textbf{JH} & \textbf{Name} & \textbf{Safe?} & \textbf{Time(s)} & \textbf{Ann} & \textbf{JH} \\ \midrule
\textbf{Array-Inv} & \checkmark & 10.07 & 0 & \text{T/O} &
\textbf{Array-Inv-BUG} & \text{\sffamily X} & 5.29 & 0 & \text{T/O} \\
\textbf{Array-List} & \checkmark & 16.76 & 0 & \text{T/O} &
\textbf{Array-List-BUG} & \text{\sffamily X} & 1.13 & 0 & \text{T/O} \\
\textbf{Intro2} & \checkmark & 0.08 & 0 & \text{T/O} &
\textbf{Intro2-BUG} & \text{\sffamily X} & 0.02 & 0 & \text{T/O} \\
\textbf{Mut-List} & \checkmark & 1.45 & 3 & \text{T/O} &
\textbf{Mut-List-BUG} & \text{\sffamily X} & 0.41 & 3 & \text{T/O} \\
\textbf{Shuffle} & \checkmark & 0.13 & 3 & \checkmark &
\textbf{Shuffle-BUG} & \text{\sffamily X} & 0.07 & 3 & \text{\sffamily X} \\
\textbf{Sorted-List} & \checkmark & 1.90 & 3 & \text{T/O} &
\textbf{Sorted-List-BUG} & \text{\sffamily X} & 1.10 & 3 & \text{T/O} \\
\end{tabular}

  \end{center}
  \label{tab:jh-results}
\end{table}

\subsection{Results}
The results of our experiments are shown in \Cref{tab:jh-results}. On the JayHorn benchmark
suite \name performs competitively with JayHorn, correctly identifying 29 of the 32 safe programs
as such. For all 3 tests on which \name timed out after 60 seconds, JayHorn also timed out
(column \emph{T/O}).
For the unsafe programs, \name correctly identified all programs as unsafe within 60 seconds;
JayHorn answered \textsc{Unknown} for 7 tests (column \emph{Imp.}).

On our own benchmark set, \name correctly verifies all safe versions
of the programs within 60 seconds. For the unsafe variants,
\name was able to quickly and definitively determine these programs unsafe.
JayHorn times out on all tests except for \textbf{Shuffle} and \textbf{ShuffleBUG} (column \textbf{JH}).
We investigated the cause of time outs and discovered that after verification failed
with an unbounded heap model, JayHorn attempts verification on increasingly
larger bounded heaps. In every case, JayHorn exceeded the 60 second timeout
before reaching a preconfigured limit on the heap bound. This result suggests JayHorn struggles in the presence
of per-object invariants and unbounded allocations; 
the only two tests JayHorn successfully analyzed contain just a single object allocation.\looseness=-1

We do not believe
this struggle is indicative of a shortcoming in JayHorn's implementation,
but stems from the fundamental limitations of JayHorn's memory representation.
Like many verification tools (see \Cref{sec:rw}), JayHorn uses a single, unchanging
invariant to for every object allocated at the same syntactic location;
effectively, all objects allocated at the same location
are assumed to alias with one another.
This representation cannot, in general, handle programs with different
invariants for distinct objects that evolve over time.
We hypothesize other tools that adopt a similar approach will exhibit
the same difficulty. 

%%% Local Variables:
%%% mode: latex
%%% TeX-master: t
%%% End:

%  LocalWords:  JayHorn JayHorn's
