\let\oldvec\vec
\documentclass[runningheads]{llncs}
\usepackage{bcprules}\typicallabel{T-Hoge}
\usepackage{bcpproof}
\usepackage{graphicx,xcolor}
\usepackage{amsmath}
\usepackage{amssymb}
\usepackage{stmaryrd}
\usepackage{parcolumns}
\usepackage{cases}
\usepackage{listings}% http://ctan.org/pkg/listings
\usepackage{etoolbox}
\usepackage{microtype}
\usepackage{wrapfig}
\usepackage{paralist}
\usepackage[numbers,sectionbib,sort]{natbib}
\usepackage{cleveref}


\DeclareMathOperator{\SEQ}{;}
\DeclareMathOperator{\smodels}{\approx}
\newcommand\expr{e}
\newcommand\intconst{n}
\newcommand\SKIP{\mathbf{skip}}
\DeclareMathOperator{\BOP}{\mathit{bop}}
\DeclareMathOperator{\LET}{\mathbf{let}}
\DeclareMathOperator{\IN}{\mathbf{in}}
\DeclareMathOperator{\WRITE}{:=}
\newcommand{\DEREF}{*}
\newcommand{\NULL}{\mathbf{null}}
\DeclareMathOperator\IFZERO{\mathbf{ifz}}
\DeclareMathOperator\THEN{\mathbf{then}}
\DeclareMathOperator\ELSE{\mathbf{else}}
\DeclareMathOperator{\COL}{:}
\newcommand\ASSERT{\mathbf{assert}}
\DeclareMathOperator{\RETURN}{\mathbf{return}}
\newcommand\HOLE{[]}
\newcommand\UNIT{\mathbf{()}}
\newcommand\wf{\mathit{WF}}
\newcommand{\consort}{\textsc{ConSORT}\xspace}
\let\name\consort
\newcommand{\RETVAR}{\rho}
\newcommand{\DOM}{\mathit{dom}}
\newcommand{\CVar}{\textbf{CVar}}

\let\imp\lstinline

\newcommand\seq[1]{\overrightarrow{#1}}

\newcommand\decl{d}
\newcommand\prog{P}

\newcommand\pp{\ell}
\newcommand{\pps}{\beta}

\newcommand\tuple[1]{\left\langle{#1}\right\rangle}

\newcommand\stmt{s}

\newcommand\typ{\tau}

\newcommand\TINT{\mathbf{int}}
\DeclareMathOperator\TREF{\mathbf{ref}}

\newcommand\ownership{r}

\newcommand\RAT{\mathbb{Q}}

\newcommand\set[1]{\left\{{#1}\right\}}

\DeclareMathOperator{\p}{\vdash}
\DeclareMathOperator{\ctxtp}{\vdash_{E}}
\DeclareMathOperator{\Cp}{\vdash_{C}}

\DeclareMathOperator{\produces}{\Rightarrow}
\DeclareMathOperator{\MKREF}{\mathbf{mkref}}

\newcommand\tenv{\Gamma}

\newcommand\funenv{\Theta}
\newcommand\funtyp{\sigma}
\DeclareMathOperator{\diff}{\backslash}

\newcommand{\denote}[1]{\llbracket#1\rrbracket}

\DeclareMathOperator{\ra}{\rightarrow}

\DeclareMathOperator{\subt}{\le}

\DeclareMathOperator{\addt}{+}

\newcommand{\fv}{\mathbf{FV}}

\newcommand{\rulesp}{\vspace{0.1cm}}

\newcommand\fcv{\mathbf{FCV}}

\newcommand\ALIAS{\mathbf{alias}}

\newcommand\sem[1]{\left\llbracket{#1}\right\rrbracket}

\newcommand\TRUE{\top}

\newcommand\TUNIT{\mathbf{unit}}

\newif\ifdraftComments
\draftCommentstrue
\def\mkDraftFn#1#2{%
  \expandafter\def\csname #1\endcsname##1{\ifdraftComments\textcolor{#2}{[#1: ##1]}\marginpar[$\longrightarrow$]{$\longleftarrow$}\fi}%
}
\mkDraftFn{JT}{blue}
\mkDraftFn{AI}{red}
\mkDraftFn{KS}{purple}
\mkDraftFn{NK}{cyan}
\mkDraftFn{RS}{teal}
\usepackage{xspace}
\makeatletter
\def\needcite@with[#1]{\ifdraftComments\textcolor{blue}{[citation needed #1]}\else\empty\fi\xspace}
\def\needcite@bare{\ifdraftComments\textcolor{blue}{[citation needed]}\else\empty\fi\xspace}
\def\needcite{\@ifnextchar[{\needcite@with}{\needcite@bare}}
\makeatother

\lstdefinelanguage{Imp}{
  keywords=[0]{ifz,then,else,alias,assert,mkref,let,in,null,ifnull,return},
  morecomment=[l]{//},
  morecomment=[s]{/*}{*/}
}
\definecolor{comment-green}{rgb}{0,0.6,0}
\lstset{
  basicstyle=\ttfamily,
  language=Imp,
  columns=flexible,
  keywordstyle=[0]{\normalfont\bfseries},
  commentstyle=\color{comment-green},
  mathescape,
}

%%% Local Variables:
%%% mode: latex
%%% TeX-master: "main.tex"
%%% End:

% generated by Ott 0.30 from: flow.ott
\newcommand{\ottdrule}[4][]{{\displaystyle\frac{\begin{array}{l}#2\end{array}}{#3}\quad\ottdrulename{#4}}}
\newcommand{\ottusedrule}[1]{\[#1\]}
\newcommand{\ottpremise}[1]{ #1 \\}
\newenvironment{ottdefnblock}[3][]{ \framebox{\mbox{#2}} \quad #3 \\[0pt]}{}
\newenvironment{ottfundefnblock}[3][]{ \framebox{\mbox{#2}} \quad #3 \\[0pt]\begin{displaymath}\begin{array}{l}}{\end{array}\end{displaymath}}
\newcommand{\ottfunclause}[2]{ #1 \equiv #2 \\}
\newcommand{\ottnt}[1]{\mathit{#1}}
\newcommand{\ottmv}[1]{\mathit{#1}}
\newcommand{\ottkw}[1]{\mathbf{#1}}
\newcommand{\ottsym}[1]{#1}
\newcommand{\ottcom}[1]{\text{#1}}
\newcommand{\ottdrulename}[1]{\textsc{#1}}
\newcommand{\ottcomplu}[5]{\overline{#1}^{\,#2\in #3 #4 #5}}
\newcommand{\ottcompu}[3]{\overline{#1}^{\,#2<#3}}
\newcommand{\ottcomp}[2]{\overline{#1}^{\,#2}}
\newcommand{\ottgrammartabular}[1]{\begin{supertabular}{llcllllll}#1\end{supertabular}}
\newcommand{\ottmetavartabular}[1]{\begin{supertabular}{ll}#1\end{supertabular}}
\newcommand{\ottrulehead}[3]{$#1$ & & $#2$ & & & \multicolumn{2}{l}{#3}}
\newcommand{\ottprodline}[6]{& & $#1$ & $#2$ & $#3 #4$ & $#5$ & $#6$}
\newcommand{\ottfirstprodline}[6]{\ottprodline{#1}{#2}{#3}{#4}{#5}{#6}}
\newcommand{\ottlongprodline}[2]{& & $#1$ & \multicolumn{4}{l}{$#2$}}
\newcommand{\ottfirstlongprodline}[2]{\ottlongprodline{#1}{#2}}
\newcommand{\ottbindspecprodline}[6]{\ottprodline{#1}{#2}{#3}{#4}{#5}{#6}}
\newcommand{\ottprodnewline}{\\}
\newcommand{\ottinterrule}{\\[5.0mm]}
\newcommand{\ottafterlastrule}{\\}
\newcommand{\ottmetavars}{
\ottmetavartabular{
 $ \mathit{termvar} ,\, \mathit{x} ,\, \mathit{y} ,\, \mathit{z} $ & \ottcom{term variable} \\
 $ \mathit{c} $ & \ottcom{context var} \\
 $ \mathit{funvar} ,\, \mathit{f} ,\, \mathit{g} $ & \ottcom{function name} \\
 $ \ottmv{index} ,\, \ottmv{i} ,\, \ottmv{j} ,\, \ottmv{n} ,\, \ottmv{p} ,\, \ottmv{q} ,\, \ottmv{k} $ & \ottcom{index} \\
 $ \ottmv{a} $ & \ottcom{addresses} \\
 $ \phi $ & \ottcom{predicate symbol} \\
}}

\newcommand{\otte}{
\ottrulehead{\ottnt{e}}{::=}{\ottcom{exp}}\ottprodnewline
\ottfirstprodline{|}{\mathit{x}}{}{}{}{\ottcom{variable}}\ottprodnewline
\ottprodline{|}{ \LET  \mathit{x}  =  \ottnt{rhs}  \IN  \ottnt{e} }{}{}{}{}\ottprodnewline
\ottprodline{|}{ \IFZERO  \mathit{x}  \THEN  \ottnt{e_{{\mathrm{1}}}}  \ELSE  \ottnt{e_{{\mathrm{2}}}} }{}{}{}{}\ottprodnewline
\ottprodline{|}{ \mathit{x}  \WRITE  \mathit{y}  \SEQ  \ottnt{e} }{}{}{}{}\ottprodnewline
\ottprodline{|}{ \ALIAS( \mathit{x}  =  \mathit{y} ) \SEQ  \ottnt{e} }{}{}{}{}\ottprodnewline
\ottprodline{|}{ \ALIAS( \mathit{x}  = *  \mathit{y} ) \SEQ  \ottnt{e} }{}{}{}{}\ottprodnewline
\ottprodline{|}{ \ASSERT( \varphi ) \SEQ  \ottnt{e} }{}{}{}{}\ottprodnewline
\ottprodline{|}{ \ottnt{e}  \SEQ  \ottnt{e'} }{}{}{}{}\ottprodnewline
\ottprodline{|}{ \sigma_{x}   \ottnt{e} }{}{}{}{}\ottprodnewline
\ottprodline{|}{\ottsym{(}  \ottnt{e}  \ottsym{)}} {\textsf{S}}{}{}{}\ottprodnewline
\ottprodline{|}{F  \ottsym{[}  \mathit{x}  \ottsym{]}} {\textsf{M}}{}{}{}\ottprodnewline
\ottprodline{|}{\ottnt{E}  \ottsym{[}  \ottnt{e}  \ottsym{]}} {\textsf{M}}{}{}{}\ottprodnewline
\ottprodline{|}{\ottnt{e_{{\mathrm{1}}}}  \ottsym{=}  \ottnt{e_{{\mathrm{2}}}}}{}{}{}{}\ottprodnewline
\ottprodline{|}{ \ldots }{}{}{}{}}

\newcommand{\ottreturnXXhole}{
\ottrulehead{\ottnt{return\_hole}}{::=}{}\ottprodnewline
\ottfirstprodline{|}{ \HOLE^ \ell }{}{}{}{}}

\newcommand{\ottrhs}{
\ottrulehead{\ottnt{rhs}}{::=}{}\ottprodnewline
\ottfirstprodline{|}{ \MKREF  \mathit{y} }{}{}{}{}\ottprodnewline
\ottprodline{|}{ *  \mathit{y} }{}{}{}{}\ottprodnewline
\ottprodline{|}{\mathit{x}}{}{}{}{\ottcom{variable}}\ottprodnewline
\ottprodline{|}{\ottnt{v}}{}{}{}{\ottcom{value}}\ottprodnewline
\ottprodline{|}{\ottnt{return\_hole}}{}{}{}{}\ottprodnewline
\ottprodline{|}{ \mathit{f} ^ \ell ( \ottnt{arg\_list} ) }{}{}{}{\ottcom{function call}}}

\newcommand{\ottargXXlist}{
\ottrulehead{\ottnt{arg\_list}}{::=}{}\ottprodnewline
\ottfirstprodline{|}{ \mathit{y_{{\mathrm{1}}}} ,\ldots, \mathit{y_{\ottmv{n}}} }{}{}{}{}\ottprodnewline
\ottprodline{|}{\mathit{y_{{\mathrm{1}}}}  \ottsym{,} \, ... \, \ottsym{,}  \mathit{y_{\ottmv{n}}}}{}{}{}{}}

\newcommand{\ottA}{
\ottrulehead{\lambda}{::=}{}}

\newcommand{\ottv}{
\ottrulehead{\ottnt{v}}{::=}{\ottcom{value}}\ottprodnewline
\ottfirstprodline{|}{n}{}{}{}{\ottcom{natural number}}\ottprodnewline
\ottprodline{|}{\ottsym{0}}{}{}{}{\ottcom{zero}}\ottprodnewline
\ottprodline{|}{\ottsym{1}}{}{}{}{\ottcom{one}}\ottprodnewline
\ottprodline{|}{\ottmv{a}}{}{}{}{}}

\newcommand{\ottfunXXbody}{
\ottrulehead{\ottnt{fun\_body}}{::=}{}\ottprodnewline
\ottfirstprodline{|}{\ottsym{(}  \mathit{x_{{\mathrm{1}}}}  \ottsym{,} \, ... \, \ottsym{,}  \mathit{x_{\ottmv{i}}}  \ottsym{)}  \ottnt{e}}{}{}{}{}}

\newcommand{\ottd}{
\ottrulehead{\ottnt{d}}{::=}{}\ottprodnewline
\ottfirstprodline{|}{\mathit{f}  \mapsto  \ottnt{fun\_body}}{}{}{}{}}

\newcommand{\ottD}{
\ottrulehead{\ottnt{D}}{::=}{}\ottprodnewline
\ottfirstprodline{|}{\ottsym{\{}  \ottnt{d_{{\mathrm{1}}}}  \ottsym{,} \, ... \, \ottsym{,}  \ottnt{d_{\ottmv{i}}}  \ottsym{\}}}{}{}{}{}}

\newcommand{\ottP}{
\ottrulehead{\ottnt{P}}{::=}{\ottcom{programs}}\ottprodnewline
\ottfirstprodline{|}{ \tuple{ \ottnt{D} ,  \ottnt{e} } }{}{}{}{}}

\newcommand{\ottr}{
\ottrulehead{r}{::=}{\ottcom{ownership}}\ottprodnewline
\ottfirstprodline{|}{ [0,1] \in \mathbb{Q} }{}{}{}{}\ottprodnewline
\ottprodline{|}{r_{{\mathrm{1}}}  \ottsym{+}  r_{{\mathrm{2}}}}{}{}{}{}\ottprodnewline
\ottprodline{|}{r_{{\mathrm{1}}}  \ottsym{-}  r_{{\mathrm{2}}}}{}{}{}{}\ottprodnewline
\ottprodline{|}{\ottsym{0}}{}{}{}{}\ottprodnewline
\ottprodline{|}{\ottsym{1}}{}{}{}{}\ottprodnewline
\ottprodline{|}{\ottnt{O}  \ottsym{(}  \ottmv{a}  \ottsym{)}}{}{}{}{}\ottprodnewline
\ottprodline{|}{\ottnt{float}}{}{}{}{}\ottprodnewline
\ottprodline{|}{ r _{ \mathit{x} , n , p } }{}{}{}{}\ottprodnewline
\ottprodline{|}{\ottsym{(}  r  \ottsym{)}}{}{}{}{}}

\newcommand{\otttop}{
\ottrulehead{\top}{::=}{}\ottprodnewline
\ottfirstprodline{|}{\top}{}{}{}{}}

\newcommand{\ottfloat}{
\ottrulehead{\ottnt{float}}{::=}{}\ottprodnewline
\ottfirstprodline{|}{\ottsym{0}  \ottsym{.}  \ottnt{DL}}{}{}{}{}}

\newcommand{\ottDL}{
\ottrulehead{\ottnt{DL}}{::=}{}\ottprodnewline
\ottfirstprodline{|}{\ottnt{digit}}{}{}{}{}\ottprodnewline
\ottprodline{|}{\ottnt{digit} \, \ottnt{DL}}{}{}{}{}}

\newcommand{\ottnn}{
\ottrulehead{n}{::=}{}\ottprodnewline
\ottfirstprodline{|}{\ottnt{DL}}{}{}{}{}\ottprodnewline
\ottprodline{|}{n_{{\mathrm{1}}}  \ottsym{+}  n_{{\mathrm{2}}}}{}{}{}{}\ottprodnewline
\ottprodline{|}{n_{{\mathrm{1}}}  \ottsym{-}  n_{{\mathrm{2}}}}{}{}{}{}\ottprodnewline
\ottprodline{|}{ | \tau | }{}{}{}{}\ottprodnewline
\ottprodline{|}{ | \oldvec{\ell} | }{}{}{}{}\ottprodnewline
\ottprodline{|}{ | \oldvec{F} | }{}{}{}{}}

\newcommand{\ottl}{
\ottrulehead{\ell}{::=}{\ottcom{location label}}\ottprodnewline
\ottfirstprodline{|}{\ottnt{DL}}{}{}{}{}}

\newcommand{\ottdigit}{
\ottrulehead{\ottnt{digit}}{::=}{}\ottprodnewline
\ottfirstprodline{|}{\ottsym{0}}{}{}{}{}\ottprodnewline
\ottprodline{|}{\ottsym{1}}{}{}{}{}\ottprodnewline
\ottprodline{|}{\ottsym{2}}{}{}{}{}\ottprodnewline
\ottprodline{|}{\ottsym{3}}{}{}{}{}\ottprodnewline
\ottprodline{|}{\ottsym{4}}{}{}{}{}\ottprodnewline
\ottprodline{|}{\ottsym{5}}{}{}{}{}\ottprodnewline
\ottprodline{|}{\ottsym{6}}{}{}{}{}\ottprodnewline
\ottprodline{|}{\ottsym{7}}{}{}{}{}\ottprodnewline
\ottprodline{|}{\ottsym{8}}{}{}{}{}\ottprodnewline
\ottprodline{|}{\ottsym{9}}{}{}{}{}}

\newcommand{\ottC}{
\ottrulehead{\mathbf{C}}{::=}{\ottcom{configuration}}\ottprodnewline
\ottfirstprodline{|}{ \tuple{ \ottnt{H} ,  \ottnt{R} ,  \oldvec{F} ,  \ottnt{e} } }{}{}{}{}\ottprodnewline
\ottprodline{|}{ \mathbf{AssertFail} }{}{}{}{}\ottprodnewline
\ottprodline{|}{ \mathbf{AliasFail} }{}{}{}{}}

\newcommand{\ottR}{
\ottrulehead{\ottnt{R}}{::=}{\ottcom{env or register file}}\ottprodnewline
\ottfirstprodline{|}{ \emptyset }{}{}{}{}\ottprodnewline
\ottprodline{|}{\ottnt{R}  \ottsym{\{}  \mathit{x}  \mapsto  v  \ottsym{\}}}{}{}{}{}\ottprodnewline
\ottprodline{|}{\ottnt{R_{{\mathrm{1}}}}  \ottsym{=}  \ottnt{R_{{\mathrm{2}}}}}{}{}{}{}}

\newcommand{\ottH}{
\ottrulehead{\ottnt{H}}{::=}{\ottcom{heap}}\ottprodnewline
\ottfirstprodline{|}{ \emptyset }{}{}{}{}\ottprodnewline
\ottprodline{|}{\ottnt{H}  \ottsym{\{}  \ottmv{a}  \mapsto  v  \ottsym{\}}}{}{}{}{}\ottprodnewline
\ottprodline{|}{\ottnt{H}  \ottsym{\{}  \ottmv{a}  \hookleftarrow  v  \ottsym{\}}}{}{}{}{}\ottprodnewline
\ottprodline{|}{\ottnt{H_{{\mathrm{1}}}}  \ottsym{=}  \ottnt{H_{{\mathrm{2}}}}}{}{}{}{}}

\newcommand{\ottRv}{
\ottrulehead{v}{::=}{\ottcom{runtime value forms}}\ottprodnewline
\ottfirstprodline{|}{\ottmv{a}}{}{}{}{}\ottprodnewline
\ottprodline{|}{\ottnt{R}  \ottsym{(}  \mathit{x}  \ottsym{)}}{}{}{}{}\ottprodnewline
\ottprodline{|}{\ottnt{H}  \ottsym{(}  v  \ottsym{)}}{}{}{}{}\ottprodnewline
\ottprodline{|}{\ottnt{v}}{}{}{}{}\ottprodnewline
\ottprodline{|}{n}{}{}{}{}\ottprodnewline
\ottprodline{|}{\mathcal{M}  \ottsym{(}  \pi  \ottsym{)}}{}{}{}{}\ottprodnewline
\ottprodline{|}{\mathcal{H}  \ottsym{(}  \oldvec{\star}  \ottsym{)}}{}{}{}{}}

\newcommand{\ottE}{
\ottrulehead{\ottnt{E}}{::=}{\ottcom{evaluation context}}\ottprodnewline
\ottfirstprodline{|}{\HOLE}{}{}{}{}\ottprodnewline
\ottprodline{|}{ \ottnt{E} \SEQ \ottnt{e} }{}{}{}{}\ottprodnewline
\ottprodline{|}{\ottsym{(}  \ottnt{E}  \ottsym{)}}{}{}{}{}\ottprodnewline
\ottprodline{|}{\ottnt{E}  \ottsym{=}  \ottnt{E'}}{}{}{}{}}

\newcommand{\ottCE}{
\ottrulehead{F}{::=}{\ottcom{return contexts}}\ottprodnewline
\ottfirstprodline{|}{ \ottnt{E} [\LET  \mathit{x}  =  \ottnt{return\_hole}  \IN  \ottnt{e}  ] }{}{}{}{}}

\newcommand{\ottEs}{
\ottrulehead{\oldvec{F}}{::=}{\ottcom{stack}}\ottprodnewline
\ottfirstprodline{|}{ \cdot }{}{}{}{}\ottprodnewline
\ottprodline{|}{F  \ottsym{:}  \oldvec{F}}{}{}{}{}\ottprodnewline
\ottprodline{|}{\ottsym{Fn-1}  \ottsym{:}  \oldvec{F}}{}{}{}{}}

\newcommand{\ottineq}{
\ottrulehead{\ottnt{ineq}}{::=}{}\ottprodnewline
\ottfirstprodline{|}{ \subseteq }{}{}{}{}\ottprodnewline
\ottprodline{|}{ \not\subseteq }{}{}{}{}\ottprodnewline
\ottprodline{|}{ \overset{?}{\subseteq} }{}{}{}{}}

\newcommand{\ottplhs}{
\ottrulehead{\mathcal{C}}{::=}{}\ottprodnewline
\ottfirstprodline{|}{\mathcal{L}}{}{}{}{}\ottprodnewline
\ottprodline{|}{ \ell  :  \mathcal{C} }{}{}{}{}\ottprodnewline
\ottprodline{|}{\oldvec{\ell}  \ottsym{:}  \lambda}{}{}{}{}}

\newcommand{\ottprhs}{
\ottrulehead{\ottnt{prhs}}{::=}{}\ottprodnewline
\ottfirstprodline{|}{ \oldvec{\ell} }{}{}{}{}}

\newcommand{\ottprefixXXop}{
\ottrulehead{\ottnt{prefix\_op}}{::=}{\ottcom{prefix operator}}\ottprodnewline
\ottfirstprodline{|}{ \subseteq }{}{}{}{}\ottprodnewline
\ottprodline{|}{ \not\subseteq }{}{}{}{}}

\newcommand{\ottCQP}{
\ottrulehead{\mathcal{CP}}{::=}{\ottcom{prefix query}}\ottprodnewline
\ottfirstprodline{|}{ \ottnt{prhs}   \ottnt{prefix\_op}   \mathcal{C} }{}{}{}{}}

\newcommand{\ottll}{
\ottrulehead{\oldvec{\ell}}{::=}{\ottcom{concrete contexts}}\ottprodnewline
\ottfirstprodline{|}{ \epsilon }{}{}{}{}\ottprodnewline
\ottprodline{|}{\ell}{}{}{}{}\ottprodnewline
\ottprodline{|}{\ell  \ottsym{:}  \oldvec{\ell}}{}{}{}{}\ottprodnewline
\ottprodline{|}{ \ell_{{\mathrm{1}}} \ldots \ell_{{\mathrm{2}}} }{}{}{}{}\ottprodnewline
\ottprodline{|}{\ell_{{\mathrm{1}}} \, ... \, \ell_{\ottmv{n}}}{}{}{}{}}

\newcommand{\ottB}{
\ottrulehead{\mathcal{L}}{::=}{\ottcom{type contexts}}\ottprodnewline
\ottfirstprodline{|}{ \epsilon }{}{}{}{}\ottprodnewline
\ottprodline{|}{\ell  \ottsym{:}  \mathcal{L}}{}{}{}{}\ottprodnewline
\ottprodline{|}{\ell_{{\mathrm{1}}}  \ottsym{:} \, ... \, \ottsym{:}  \ell_{\ottmv{n}}  \ottsym{:}  \mathcal{L}}{}{}{}{}\ottprodnewline
\ottprodline{|}{\oldvec{\ell}}{}{}{}{}\ottprodnewline
\ottprodline{|}{\lambda}{}{}{}{}\ottprodnewline
\ottprodline{|}{\ottkw{Trace} \, \ottsym{(}  \oldvec{F}  \ottsym{)}}{}{}{}{}\ottprodnewline
\ottprodline{|}{\sigma_{\alpha} \, \mathcal{L}}{}{}{}{}}

\newcommand{\ottca}{
\ottrulehead{\ottnt{ca}}{::=}{\ottcom{predicate context argument single}}\ottprodnewline
\ottfirstprodline{|}{\mathit{c}}{}{}{}{}\ottprodnewline
\ottprodline{|}{\ell}{}{}{}{}}

\newcommand{\ottcontextXXargs}{
\ottrulehead{\ottnt{context\_args}}{::=}{\ottcom{predicate context arguments}}\ottprodnewline
\ottfirstprodline{|}{\ottnt{ca_{{\mathrm{1}}}}  \ottsym{,}  \ottnt{context\_args}}{}{}{}{}\ottprodnewline
\ottprodline{|}{ \ottnt{ca_{{\mathrm{1}}}} ,\ldots, \ottnt{ca_{\ottmv{n}}} }{}{}{}{}}

\newcommand{\ottph}{
\ottrulehead{\varphi}{::=}{\ottcom{refinement predicates}}\ottprodnewline
\ottfirstprodline{|}{\phi  \ottsym{(}  \widehat{v}_{{\mathrm{1}}}  \ottsym{,} \, .. \, \ottsym{,}  \widehat{v}_{\ottmv{n}}  \ottsym{)}}{}{}{}{\ottcom{predicate in underlying logic}}\ottprodnewline
\ottprodline{|}{ \varphi_{{\mathrm{1}}}  \wedge  \varphi_{{\mathrm{2}}} }{}{}{}{\ottcom{conjunction}}\ottprodnewline
\ottprodline{|}{\widehat{v}_{{\mathrm{1}}} \, \ottnt{rel\_op} \, \widehat{v}_{{\mathrm{2}}}}{}{}{}{\ottcom{equality}}\ottprodnewline
\ottprodline{|}{ \top }{}{}{}{}\ottprodnewline
\ottprodline{|}{\varphi_{{\mathrm{1}}}  \implies  \varphi_{{\mathrm{2}}}}{}{}{}{}\ottprodnewline
\ottprodline{|}{\ottsym{[}  \widehat{v}  \ottsym{/}  \ottnt{pvar}  \ottsym{]}  \varphi}{}{}{}{}\ottprodnewline
\ottprodline{|}{\sigma_{x} \, \varphi}{}{}{}{}\ottprodnewline
\ottprodline{|}{\sigma_{\alpha} \, \varphi}{}{}{}{}\ottprodnewline
\ottprodline{|}{\mathcal{CP}}{}{}{}{}\ottprodnewline
\ottprodline{|}{\ottsym{(}  \varphi  \ottsym{)}}{}{}{}{}\ottprodnewline
\ottprodline{|}{ \mathit{x}  =_{ \tau }  \mathit{y} }{}{}{}{}\ottprodnewline
\ottprodline{|}{ \bot }{}{}{}{}\ottprodnewline
\ottprodline{|}{ \varphi _{ \mathit{x} , n , p_{{\mathrm{1}}} } }{}{}{}{}\ottprodnewline
\ottprodline{|}{ \varphi ( \widehat{v} ;  \ottnt{FVF} ) }{}{}{}{}\ottprodnewline
\ottprodline{|}{ \varphi ( \widehat{v} , \ottnt{context\_args} ; \ottnt{FVF} ) }{}{}{}{}\ottprodnewline
\ottprodline{|}{ \neg  \varphi }{}{}{}{}\ottprodnewline
\ottprodline{|}{\varphi_{{\mathrm{1}}}  \vee  \varphi_{{\mathrm{2}}}}{}{}{}{}\ottprodnewline
\ottprodline{|}{ \varphi_{= n } }{}{}{}{}}

\newcommand{\ottpvar}{
\ottrulehead{\ottnt{pvar}}{::=}{}\ottprodnewline
\ottfirstprodline{|}{\nu}{}{}{}{}\ottprodnewline
\ottprodline{|}{\pi}{}{}{}{}}

\newcommand{\ottAP}{
\ottrulehead{\pi}{::=}{}\ottprodnewline
\ottfirstprodline{|}{\mathit{x} \, \oldvec{\star}}{}{}{}{}\ottprodnewline
\ottprodline{|}{\pi \, \ottnt{AP\_step}}{}{}{}{}\ottprodnewline
\ottprodline{|}{\mathit{x}}{}{}{}{}}

\newcommand{\ottw}{
\ottrulehead{\oldvec{\star}}{::=}{}\ottprodnewline
\ottfirstprodline{|}{ \ottnt{AP\_step_{{\mathrm{1}}}}  \cdots  \ottnt{AP\_step_{{\mathrm{2}}}} }{}{}{}{}\ottprodnewline
\ottprodline{|}{\oldvec{\star} \, \ottnt{AP\_step}}{}{}{}{}\ottprodnewline
\ottprodline{|}{\ottnt{AP\_step} \, \oldvec{\star}}{}{}{}{}\ottprodnewline
\ottprodline{|}{ \epsilon }{}{}{}{}}

\newcommand{\ottAPXXstep}{
\ottrulehead{\ottnt{AP\_step}}{::=}{}\ottprodnewline
\ottfirstprodline{|}{ \star }{}{}{}{}\ottprodnewline
\ottprodline{|}{ \star }{}{}{}{}}

\newcommand{\ottpval}{
\ottrulehead{\widehat{v}}{::=}{}\ottprodnewline
\ottfirstprodline{|}{\ottnt{pvar}}{}{}{}{}\ottprodnewline
\ottprodline{|}{v}{}{}{}{}\ottprodnewline
\ottprodline{|}{\widehat{v}_{{\mathrm{1}}}  \ottsym{+}  \widehat{v}_{{\mathrm{2}}}}{}{}{}{}}

\newcommand{\ottcsub}{
\ottrulehead{\sigma_{\alpha}}{::=}{\ottcom{context sub}}\ottprodnewline
\ottfirstprodline{|}{\ottsym{[}  \mathcal{L}  \ottsym{/}  \lambda  \ottsym{]}} {\textsf{M}}{}{}{}\ottprodnewline
\ottprodline{|}{\sigma_{\alpha}_{{\mathrm{1}}} \, \sigma_{\alpha}_{{\mathrm{2}}}} {\textsf{M}}{}{}{}}

\newcommand{\ottvsub}{
\ottrulehead{\sigma_{x}}{::=}{}\ottprodnewline
\ottfirstprodline{|}{ \sigma_{x}_{{\mathrm{1}}} \cdots \sigma_{x}_{{\mathrm{2}}} }{}{}{}{}\ottprodnewline
\ottprodline{|}{ [  \mathit{x_{{\mathrm{1}}}}  /  \mathit{x_{{\mathrm{2}}}}  ] }{}{}{}{}\ottprodnewline
\ottprodline{|}{\ottsym{[}  \mathit{x_{{\mathrm{1}}}}  \ottsym{/} \, \nu \, \ottsym{]}}{}{}{}{}\ottprodnewline
\ottprodline{|}{\ottsym{[} \, \nu \, \ottsym{/}  \mathit{x_{{\mathrm{1}}}}  \ottsym{]}}{}{}{}{}\ottprodnewline
\ottprodline{|}{\ottsym{[}  \ottnt{R}  \ottsym{]}}{}{}{}{}\ottprodnewline
\ottprodline{|}{\ottsym{[}  \mathit{c_{{\mathrm{1}}}}  \ottsym{/}  \mathit{c_{{\mathrm{2}}}}  \ottsym{]}}{}{}{}{}\ottprodnewline
\ottprodline{|}{\ottsym{[}  \ell  \ottsym{/}  \mathit{c}  \ottsym{]}}{}{}{}{}\ottprodnewline
\ottprodline{|}{\sigma_{x}_{{\mathrm{1}}} \, \sigma_{x}_{{\mathrm{2}}}}{}{}{}{}\ottprodnewline
\ottprodline{|}{ [  \pi  /  \nu  ] }{}{}{}{}\ottprodnewline
\ottprodline{|}{\ottsym{[}  \ottnt{H}  \ottsym{,}  \ottnt{R}  \ottsym{]}}{}{}{}{}}

\newcommand{\ottT}{
\ottrulehead{\tau}{::=}{\ottcom{type}}\ottprodnewline
\ottfirstprodline{|}{ \set{  \nu  \COL \TINT \mid  \varphi } }{}{}{}{\ottcom{refined integer type}}\ottprodnewline
\ottprodline{|}{ \set{  \nu  \COL \TINT \mid \_ } }{}{}{}{\ottcom{refined integer type with dontcare refinement}}\ottprodnewline
\ottprodline{|}{ \tau  \TREF^{ r } }{}{}{}{\ottcom{reference type}}\ottprodnewline
\ottprodline{|}{\ottsym{(}  \tau  \ottsym{)}} {\textsf{S}}{}{}{}\ottprodnewline
\ottprodline{|}{\tau_{{\mathrm{1}}}  \ottsym{+}  \tau_{{\mathrm{2}}}} {\textsf{M}}{}{}{}\ottprodnewline
\ottprodline{|}{\Gamma  \ottsym{(}  \pi  \ottsym{)}}{}{}{}{}\ottprodnewline
\ottprodline{|}{\tau  \ottsym{(}  \oldvec{\star}  \ottsym{)}}{}{}{}{}\ottprodnewline
\ottprodline{|}{\sigma_{\alpha} \, \tau}{}{}{}{}\ottprodnewline
\ottprodline{|}{\sigma_{x} \, \tau} {\textsf{M}}{}{}{}\ottprodnewline
\ottprodline{|}{ \tau_{{\mathrm{1}}}  \wedge_{ \mathit{x} }  \varphi }{}{}{}{}\ottprodnewline
\ottprodline{|}{\tau_{{\mathrm{1}}}  \wedge \, \nu \, \ottsym{=}  \pi}{}{}{}{}\ottprodnewline
\ottprodline{|}{\tau_{{\mathrm{1}}}  \ottsym{=}  \tau_{{\mathrm{2}}}}{}{}{}{}\ottprodnewline
\ottprodline{|}{ \tau_{n+1} }{}{}{}{}\ottprodnewline
\ottprodline{|}{\top}{}{}{}{}\ottprodnewline
\ottprodline{|}{ \sem{ \tau_{S} }_{ \mathit{x} , n , p } }{}{}{}{}\ottprodnewline
\ottprodline{|}{ ( \ottnt{tuple\_bind_{{\mathrm{1}}}} ,\ldots, \ottnt{tuple\_bind_{{\mathrm{2}}}} ) }{}{}{}{}\ottprodnewline
\ottprodline{|}{\ottsym{(}  \ottnt{tuple\_bind_{{\mathrm{1}}}}  \ottsym{,} \, ... \, \ottsym{,}  \ottnt{tuple\_bind_{{\mathrm{2}}}}  \ottsym{)}}{}{}{}{}\ottprodnewline
\ottprodline{|}{ \mu \alpha . \tau }{}{}{}{}}

\newcommand{\otttupleXXbind}{
\ottrulehead{\ottnt{tuple\_bind}}{::=}{}\ottprodnewline
\ottfirstprodline{|}{ \mathit{x} \COL \tau }{}{}{}{}\ottprodnewline
\ottprodline{|}{\tau}{}{}{}{}}

\newcommand{\ottST}{
\ottrulehead{\tau_{S}}{::=}{\ottcom{simple type}}\ottprodnewline
\ottfirstprodline{|}{ \TINT }{}{}{}{}\ottprodnewline
\ottprodline{|}{ \tau_{S}  \TREF }{}{}{}{}\ottprodnewline
\ottprodline{|}{ \llparenthesis  \tau \rrparenthesis }{}{}{}{}\ottprodnewline
\ottprodline{|}{\Gamma_{S}  \ottsym{(}  \mathit{x}  \ottsym{)}}{}{}{}{}\ottprodnewline
\ottprodline{|}{ \mu \alpha . \tau_{S} }{}{}{}{}\ottprodnewline
\ottprodline{|}{\tau_{S}  \ottsym{[} \, \alpha \, \ottsym{]}}{}{}{}{}\ottprodnewline
\ottprodline{|}{\ottsym{[}  \tau_{S}_{{\mathrm{1}}}  \ottsym{/} \, \alpha \, \ottsym{]}  \tau_{S}_{{\mathrm{2}}}}{}{}{}{}\ottprodnewline
\ottprodline{|}{\ottsym{(}  \tau_{S}_{{\mathrm{1}}}  \ottsym{,} \, ... \, \ottsym{,}  \tau_{S}_{{\mathrm{2}}}  \ottsym{)}}{}{}{}{}\ottprodnewline
\ottprodline{|}{\alpha}{}{}{}{}}

\newcommand{\ottGst}{
\ottrulehead{\Gamma_{S}}{::=}{\ottcom{simple type env}}\ottprodnewline
\ottfirstprodline{|}{ \bullet }{}{}{}{}\ottprodnewline
\ottprodline{|}{\Gamma_{S}  \ottsym{,}  \mathit{x}  \ottsym{:}  \tau_{S}}{}{}{}{}\ottprodnewline
\ottprodline{|}{ \llparenthesis  \Gamma  \rrparenthesis }{}{}{}{}\ottprodnewline
\ottprodline{|}{\Gamma_{S}  \ottsym{,} \, \nu \, \ottsym{:}  \tau_{S}}{}{}{}{}}

\newcommand{\otttyargXXlist}{
\ottrulehead{\ottnt{tyarg\_list}}{::=}{}\ottprodnewline
\ottfirstprodline{|}{\mathit{x_{{\mathrm{1}}}}  \ottsym{:}  \tau_{{\mathrm{1}}}  \ottsym{,} \, .. \, \ottsym{,}  \mathit{x_{\ottmv{n}}}  \ottsym{:}  \tau_{\ottmv{n}}}{}{}{}{}}

\newcommand{\ottFT}{
\ottrulehead{\sigma}{::=}{\ottcom{function type}}\ottprodnewline
\ottfirstprodline{|}{ \forall  \lambda .\tuple{ \mathit{x_{{\mathrm{1}}}} \COL \tau_{{\mathrm{1}}} ,\dots, \mathit{x_{\ottmv{n}}} \COL \tau_{\ottmv{n}} }\ra\tuple{ \mathit{y_{{\mathrm{1}}}} \COL \tau'_{{\mathrm{1}}} ,\dots, \mathit{y_{\ottmv{i}}} \COL \tau'_{\ottmv{i}}  \mid  \tau } }{}{}{}{}\ottprodnewline
\ottprodline{|}{ \forall  \lambda .\tuple{ \ottnt{tyarg\_list_{{\mathrm{1}}}} }\ra\tuple{ \ottnt{tyarg\_list_{{\mathrm{2}}}} \mid \tau } }{}{}{}{}\ottprodnewline
\ottprodline{|}{\Theta  \ottsym{(}  \mathit{f}  \ottsym{)}}{}{}{}{}\ottprodnewline
\ottprodline{|}{ \forall  \lambda .\tuple{ \mathit{x_{{\mathrm{1}}}} \COL \tau_{{\mathrm{1}}} ,\dots, \mathit{x_{\ottmv{n}}} \COL \tau_{\ottmv{n}} } \\ & & & \ra\tuple{ \mathit{y_{{\mathrm{1}}}} \COL \tau'_{{\mathrm{1}}} ,\dots, \mathit{y_{\ottmv{i}}} \COL \tau'_{\ottmv{i}}  \mid  \tau } }{}{}{}{}}

\newcommand{\ottTh}{
\ottrulehead{\Theta}{::=}{}\ottprodnewline
\ottfirstprodline{|}{ \bullet }{}{}{}{}\ottprodnewline
\ottprodline{|}{\mathit{f}  \mapsto  \sigma  \ottsym{,}  \Theta}{}{}{}{}}

\newcommand{\ottG}{
\ottrulehead{\Gamma}{::=}{\ottcom{type environment}}\ottprodnewline
\ottfirstprodline{|}{ \bullet }{}{}{}{}\ottprodnewline
\ottprodline{|}{\Gamma_{{\mathrm{1}}}  \ottsym{+}  \Gamma_{{\mathrm{2}}}}{}{}{}{\ottcom{pointwise addition}}\ottprodnewline
\ottprodline{|}{\Gamma  \ottsym{,}  \mathit{x}  \ottsym{:}  \tau}{}{}{}{}\ottprodnewline
\ottprodline{|}{ \Gamma  \setminus  \mathit{x} }{}{}{}{}\ottprodnewline
\ottprodline{|}{ \Gamma , \mathit{x_{{\mathrm{1}}}} \COL \tau_{{\mathrm{1}}} ,\ldots, \mathit{x_{\ottmv{n}}} \COL \tau_{\ottmv{n}} }{}{}{}{}\ottprodnewline
\ottprodline{|}{ \mathit{x_{{\mathrm{1}}}} \COL \tau_{{\mathrm{1}}} ,\ldots, \mathit{x_{\ottmv{n}}} \COL \tau_{\ottmv{n}} }{}{}{}{}\ottprodnewline
\ottprodline{|}{ \mathit{x} \COL \tau }{}{}{}{}\ottprodnewline
\ottprodline{|}{\Gamma  \ottsym{[}  \mathit{x}  \ottsym{:}  \tau  \ottsym{]}}{}{}{}{}\ottprodnewline
\ottprodline{|}{\Gamma  \ottsym{[}  \mathit{x_{{\mathrm{1}}}}  \hookleftarrow  \tau_{{\mathrm{1}}}  \ottsym{]}}{}{}{}{}\ottprodnewline
\ottprodline{|}{ \Gamma  \left[  \mathit{x_{{\mathrm{1}}}} \hookleftarrow \tau_{{\mathrm{1}}}  \right]\cdots[  \mathit{x_{{\mathrm{2}}}} \hookleftarrow \tau_{{\mathrm{2}}} ] }{}{}{}{}\ottprodnewline
\ottprodline{|}{\ottsym{[}  \mathcal{L}  \ottsym{/}  \lambda  \ottsym{]}  \Gamma}{}{}{}{}\ottprodnewline
\ottprodline{|}{\ottsym{(}  \Gamma  \ottsym{)}}{}{}{}{}\ottprodnewline
\ottprodline{|}{\Gamma_{{\mathrm{1}}}  \ottsym{=}  \Gamma_{{\mathrm{2}}}}{}{}{}{}\ottprodnewline
\ottprodline{|}{\sigma_{x} \, \Gamma}{}{}{}{}\ottprodnewline
\ottprodline{|}{ \tenv_{n+1} }{}{}{}{}\ottprodnewline
\ottprodline{|}{ \Gamma ^ p }{}{}{}{}}

\newcommand{\ottrelXXop}{
\ottrulehead{\ottnt{rel\_op}}{::=}{}\ottprodnewline
\ottfirstprodline{|}{\le}{}{}{}{}\ottprodnewline
\ottprodline{|}{\ottsym{<}}{}{}{}{}\ottprodnewline
\ottprodline{|}{\ottsym{=}}{}{}{}{}\ottprodnewline
\ottprodline{|}{\neq}{}{}{}{}\ottprodnewline
\ottprodline{|}{\ge}{}{}{}{}\ottprodnewline
\ottprodline{|}{\ottsym{>}}{}{}{}{}}

\newcommand{\ottpp}{
\ottrulehead{p}{::=}{}\ottprodnewline
\ottfirstprodline{|}{ { \mathit{f} ^{b} } }{}{}{}{}\ottprodnewline
\ottprodline{|}{ { \mathit{f} ^{e} } }{}{}{}{}}

\newcommand{\ottterminals}{
\ottrulehead{\ottnt{terminals}}{::=}{}\ottprodnewline
\ottfirstprodline{|}{ \SEQ }{}{}{}{}\ottprodnewline
\ottprodline{|}{ \longrightarrow }{}{}{}{}\ottprodnewline
\ottprodline{|}{ \longrightarrow^* }{}{}{}{}\ottprodnewline
\ottprodline{|}{ \rightarrow }{}{}{}{}\ottprodnewline
\ottprodline{|}{ \vdash }{}{}{}{}\ottprodnewline
\ottprodline{|}{ \vdash_{\mathit{ectx} } }{}{}{}{}\ottprodnewline
\ottprodline{|}{ \vdash_{\mathit{conf} }^D }{}{}{}{}\ottprodnewline
\ottprodline{|}{ \models }{}{}{}{}\ottprodnewline
\ottprodline{|}{ \in }{}{}{}{}\ottprodnewline
\ottprodline{|}{ \leq }{}{}{}{}\ottprodnewline
\ottprodline{|}{ \implies }{}{}{}{}\ottprodnewline
\ottprodline{|}{ \iff }{}{}{}{}\ottprodnewline
\ottprodline{|}{ \produces }{}{}{}{}\ottprodnewline
\ottprodline{|}{ \mapsto }{}{}{}{}\ottprodnewline
\ottprodline{|}{ \HOLE }{}{}{}{}\ottprodnewline
\ottprodline{|}{ \mid }{}{}{}{}\ottprodnewline
\ottprodline{|}{ \mathbb{Z} }{}{}{}{}\ottprodnewline
\ottprodline{|}{\ottkw{ifz}}{}{}{}{}\ottprodnewline
\ottprodline{|}{\ottkw{SAT}}{}{}{}{}\ottprodnewline
\ottprodline{|}{\ottkw{SATv}}{}{}{}{}\ottprodnewline
\ottprodline{|}{\ottkw{own}}{}{}{}{}\ottprodnewline
\ottprodline{|}{\ottkw{Own}}{}{}{}{}\ottprodnewline
\ottprodline{|}{\ottkw{Cons}}{}{}{}{}\ottprodnewline
\ottprodline{|}{ \mathbf{FCV} }{}{}{}{}\ottprodnewline
\ottprodline{|}{ \hookleftarrow }{}{}{}{}\ottprodnewline
\ottprodline{|}{ \not\models }{}{}{}{}\ottprodnewline
\ottprodline{|}{ \forall }{}{}{}{}\ottprodnewline
\ottprodline{|}{ \wedge }{}{}{}{}\ottprodnewline
\ottprodline{|}{ \vee }{}{}{}{}\ottprodnewline
\ottprodline{|}{ \Sigma }{}{}{}{}\ottprodnewline
\ottprodline{|}{ \le }{}{}{}{}\ottprodnewline
\ottprodline{|}{ \ge }{}{}{}{}\ottprodnewline
\ottprodline{|}{ \neq }{}{}{}{}\ottprodnewline
\ottprodline{|}{\ottsym{=}}{}{}{}{}\ottprodnewline
\ottprodline{|}{\ottsym{<}}{}{}{}{}\ottprodnewline
\ottprodline{|}{\ottsym{>}}{}{}{}{}\ottprodnewline
\ottprodline{|}{ \top }{}{}{}{}\ottprodnewline
\ottprodline{|}{ \approx }{}{}{}{}\ottprodnewline
\ottprodline{|}{ \nu }{}{}{}{}\ottprodnewline
\ottprodline{|}{ \exists }{}{}{}{}\ottprodnewline
\ottprodline{|}{ \mathcal{C} }{}{}{}{}\ottprodnewline
\ottprodline{|}{ \alpha }{}{}{}{}}

\newcommand{\ottreduction}{
\ottrulehead{\ottnt{reduction}}{::=}{}\ottprodnewline
\ottfirstprodline{|}{\longrightarrow}{}{}{}{}\ottprodnewline
\ottprodline{|}{\longrightarrow^*}{}{}{}{}\ottprodnewline
\ottprodline{|}{ \not \ottnt{reduction} }{}{}{}{}}

\newcommand{\ottlogicalXXfrag}{
\ottrulehead{\ottnt{logical\_frag}}{::=}{}\ottprodnewline
\ottfirstprodline{|}{\varphi}{}{}{}{}\ottprodnewline
\ottprodline{|}{ \sem{ \tau }_{ \pi } }{}{}{}{}\ottprodnewline
\ottprodline{|}{ \sem{ \Gamma } }{}{}{}{}\ottprodnewline
\ottprodline{|}{\sigma_{x} \, \ottnt{logical\_frag}}{}{}{}{}\ottprodnewline
\ottprodline{|}{\ottsym{[}  \widehat{v}  \ottsym{/}  \ottnt{pvar}  \ottsym{]}  \ottnt{logical\_frag}}{}{}{}{}\ottprodnewline
\ottprodline{|}{\sigma_{\alpha} \, \ottnt{logical\_frag}}{}{}{}{}\ottprodnewline
\ottprodline{|}{\ottnt{logical\_frag_{{\mathrm{1}}}}  \wedge  \ottnt{logical\_frag_{{\mathrm{2}}}}}{}{}{}{}\ottprodnewline
\ottprodline{|}{ \neg  \ottnt{logical\_frag} }{}{}{}{}\ottprodnewline
\ottprodline{|}{\ottsym{(}  \ottnt{logical\_frag}  \ottsym{)}}{}{}{}{}\ottprodnewline
\ottprodline{|}{\ottnt{logical\_frag_{{\mathrm{1}}}}  \vee  \ottnt{logical\_frag_{{\mathrm{2}}}}}{}{}{}{}\ottprodnewline
\ottprodline{|}{\ottnt{logical\_frag_{{\mathrm{1}}}}  \implies  \ottnt{logical\_frag_{{\mathrm{2}}}}}{}{}{}{}\ottprodnewline
\ottprodline{|}{\ottnt{logical\_frag_{{\mathrm{1}}}}  \iff  \ottnt{logical\_frag_{{\mathrm{2}}}}}{}{}{}{}\ottprodnewline
\ottprodline{|}{ \bigwedge_{  \ottnt{bind}  \in  \ottnt{VS}  }  \ottnt{logical\_frag} }{}{}{}{}}

\newcommand{\ottlogicXXjudgment}{
\ottrulehead{\ottnt{logic\_judgment}}{::=}{}\ottprodnewline
\ottfirstprodline{|}{\models  \ottnt{logical\_frag}}{}{}{}{}\ottprodnewline
\ottprodline{|}{\not\models  \ottnt{logical\_frag}}{}{}{}{}\ottprodnewline
\ottprodline{|}{\Gamma  \models  \ottnt{logical\_frag}}{}{}{}{}}

\newcommand{\ottFVF}{
\ottrulehead{\ottnt{FVF}}{::=}{\ottcom{free variable generators}}\ottprodnewline
\ottfirstprodline{|}{\ottkw{FPV} \, \ottsym{(}  \tau  \ottsym{)}}{}{}{}{}\ottprodnewline
\ottprodline{|}{\ottkw{FPV} \, \ottsym{(}  \varphi  \ottsym{)}}{}{}{}{}\ottprodnewline
\ottprodline{|}{\mathbf{FCV} \, \ottsym{(}  \varphi  \ottsym{)}}{}{}{}{}\ottprodnewline
\ottprodline{|}{\mathbf{FCV} \, \ottsym{(}  \mathcal{C}  \ottsym{)}}{}{}{}{}\ottprodnewline
\ottprodline{|}{\mathbf{FCV} \, \ottsym{(}  \tau  \ottsym{)}}{}{}{}{}\ottprodnewline
\ottprodline{|}{\ottkw{FV} \, \ottsym{(}  \ottnt{e}  \ottsym{)}}{}{}{}{}\ottprodnewline
\ottprodline{|}{ \ottkw{FV} _{ p } }{}{}{}{}\ottprodnewline
\ottprodline{|}{ \mathcal{C} ^ \ottmv{k} }{}{}{}{}\ottprodnewline
\ottprodline{|}{ \ottnt{FVF_{{\mathrm{1}}}}  \cup  \ottnt{FVF_{{\mathrm{2}}}} }{}{}{}{}\ottprodnewline
\ottprodline{|}{\ottsym{\{}  \mathit{x_{{\mathrm{1}}}}  \ottsym{,} \, .. \, \ottsym{,}  \mathit{x_{{\mathrm{2}}}}  \ottsym{\}}}{}{}{}{}}

\newcommand{\ottfunctions}{
\ottrulehead{\ottnt{functions}}{::=}{}\ottprodnewline
\ottfirstprodline{|}{\ottkw{CV} \, \ottsym{(}  \mathcal{L}  \ottsym{)}}{}{}{}{}\ottprodnewline
\ottprodline{|}{\ottkw{Cons} \, \ottsym{(}  \ottnt{H}  \ottsym{,}  \ottnt{R}  \ottsym{,}  \Gamma  \ottsym{)}}{}{}{}{}\ottprodnewline
\ottprodline{|}{\ottkw{SAT} \, \ottsym{(}  \ottnt{H}  \ottsym{,}  \ottnt{R}  \ottsym{,}  \Gamma  \ottsym{)}}{}{}{}{}\ottprodnewline
\ottprodline{|}{ \ottkw{SATv} ( \ottnt{H} , \ottnt{R} , v , \tau ) }{}{}{}{}\ottprodnewline
\ottprodline{|}{\ottnt{O}}{}{}{}{}}

\newcommand{\ottO}{
\ottrulehead{\ottnt{O}}{::=}{\ottcom{ownership map}}\ottprodnewline
\ottfirstprodline{|}{\ottsym{\{}  \ottmv{a}  \mapsto  r  \ottsym{\}}}{}{}{}{}\ottprodnewline
\ottprodline{|}{\ottnt{O_{{\mathrm{1}}}}  \ottsym{+}  \ottnt{O_{{\mathrm{2}}}}}{}{}{}{}\ottprodnewline
\ottprodline{|}{\ottkw{own} \, \ottsym{(}  \ottnt{H}  \ottsym{,}  v  \ottsym{,}  \tau  \ottsym{)}}{}{}{}{}\ottprodnewline
\ottprodline{|}{\ottkw{Own} \, \ottsym{(}  \ottnt{H}  \ottsym{,}  \ottnt{R}  \ottsym{,}  \Gamma  \ottsym{)}}{}{}{}{}\ottprodnewline
\ottprodline{|}{ \emptyset }{}{}{}{}\ottprodnewline
\ottprodline{|}{\ottsym{(}  \ottnt{O}  \ottsym{)}}{}{}{}{}\ottprodnewline
\ottprodline{|}{ \Sigma _{ \mathit{x} \in \ottnt{VS} }\, \ottnt{O} }{}{}{}{}}

\newcommand{\ottstepXXrelation}{
\ottrulehead{\ottnt{step\_relation}}{::=}{\ottcom{step relation}}\ottprodnewline
\ottfirstprodline{|}{ \ottnt{reduction} _{ \ottnt{D} } }{}{}{}{}}

\newcommand{\ottjudgment}{
\ottrulehead{\ottnt{judgment}}{::=}{}\ottprodnewline
\ottfirstprodline{|}{\Gamma  \vdash  \tau_{{\mathrm{1}}}  \leq  \tau_{{\mathrm{2}}}}{}{}{}{}\ottprodnewline
\ottprodline{|}{ \ottnt{R} \vdash _{vs}  \tau }{}{}{}{}\ottprodnewline
\ottprodline{|}{ \mathcal{L}   \vdash _{\wf}  \tau   \produces   \Gamma }{}{}{}{}\ottprodnewline
\ottprodline{|}{\Gamma_{{\mathrm{1}}}  \leq  \Gamma_{{\mathrm{2}}}}{}{}{}{}\ottprodnewline
\ottprodline{|}{\Gamma_{{\mathrm{1}}}  \ottsym{,}  \tau_{{\mathrm{1}}}  \leq  \Gamma_{{\mathrm{2}}}  \ottsym{,}  \tau_{{\mathrm{2}}}}{}{}{}{}\ottprodnewline
\ottprodline{|}{ \Theta   \mid   \mathcal{L}   \mid   \Gamma_{{\mathrm{1}}}   \vdash   \ottnt{e}  :  \tau   \produces   \Gamma_{{\mathrm{2}}} }{}{}{}{}\ottprodnewline
\ottprodline{|}{ \mathbf{C}_{{\mathrm{1}}}   \ottnt{step\_relation}   \mathbf{C}_{{\mathrm{2}}} }{}{}{}{}\ottprodnewline
\ottprodline{|}{ \begin{array}{l} \mathbf{C}_{{\mathrm{1}}}  \\ \quad  \ottnt{step\_relation}   \mathbf{C}_{{\mathrm{2}}} \end{array} }{}{}{}{}\ottprodnewline
\ottprodline{|}{ \begin{array}{r} \mathbf{C}_{{\mathrm{1}}}   \ottnt{step\_relation}  \\  \mathbf{C}_{{\mathrm{2}}} \end{array} }{}{}{}{}\ottprodnewline
\ottprodline{|}{ \mathcal{L}   \vdash _{\wf}  \Gamma }{}{}{}{}\ottprodnewline
\ottprodline{|}{ \mathcal{L}   \mid   \Gamma   \vdash _{\wf}  \tau }{}{}{}{}\ottprodnewline
\ottprodline{|}{ \mathcal{L}   \mid   \Gamma   \vdash _{\wf}  \varphi }{}{}{}{}\ottprodnewline
\ottprodline{|}{\tau_{{\mathrm{1}}}  \ottsym{=}  \tau_{{\mathrm{2}}}}{}{}{}{}\ottprodnewline
\ottprodline{|}{\tau_{S}_{{\mathrm{1}}}  \ottsym{=}  \tau_{S}_{{\mathrm{2}}}}{}{}{}{}\ottprodnewline
\ottprodline{|}{\Theta  \vdash  \ottnt{d}}{}{}{}{}\ottprodnewline
\ottprodline{|}{\vdash  \ottnt{P}}{}{}{}{}\ottprodnewline
\ottprodline{|}{\Theta  \vdash  \ottnt{D}}{}{}{}{}\ottprodnewline
\ottprodline{|}{ \vdash_{\mathit{conf} }^D  \mathbf{C} }{}{}{}{}\ottprodnewline
\ottprodline{|}{\Theta  \mid  \HOLE  \ottsym{:}  \tau  \produces  \Gamma  \mid  \mathcal{L}  \vdash_{\mathit{ectx} }  \ottnt{E}  \ottsym{:}  \tau'  \produces  \Gamma'}{}{}{}{}\ottprodnewline
\ottprodline{|}{\Theta  \mid  \HOLE  \ottsym{:}  \tau  \produces  \Gamma  \mid  \mathcal{L}  \vdash_{\mathit{ectx} }  F  \ottsym{:}  \tau'  \produces  \Gamma'}{}{}{}{}\ottprodnewline
\ottprodline{|}{ \vdash _{\wf}  \Theta }{}{}{}{}\ottprodnewline
\ottprodline{|}{ \vdash _{\wf}  \sigma }{}{}{}{}\ottprodnewline
\ottprodline{|}{ \ottnt{H} \vdash   v  \Downarrow  n }{}{}{}{}\ottprodnewline
\ottprodline{|}{\Gamma  \vdash  \varphi}{}{}{}{}\ottprodnewline
\ottprodline{|}{\Gamma  \vdash  \widehat{v}}{}{}{}{}\ottprodnewline
\ottprodline{|}{ \oldvec{\star}  \Downarrow  \tau }{}{}{}{}}

\newcommand{\ottbind}{
\ottrulehead{\ottnt{bind}}{::=}{\ottcom{mapping function domains}}\ottprodnewline
\ottfirstprodline{|}{\mathit{x}}{}{}{}{}\ottprodnewline
\ottprodline{|}{\lambda}{}{}{}{}\ottprodnewline
\ottprodline{|}{\ottmv{a}}{}{}{}{}\ottprodnewline
\ottprodline{|}{\mathit{f}}{}{}{}{}\ottprodnewline
\ottprodline{|}{\ottnt{v}}{}{}{}{}\ottprodnewline
\ottprodline{|}{\nu}{}{}{}{}}

\newcommand{\ottMF}{
\ottrulehead{\ottnt{MF}}{::=}{\ottcom{mapping function}}\ottprodnewline
\ottfirstprodline{|}{\Gamma}{}{}{}{}\ottprodnewline
\ottprodline{|}{\ottnt{R}}{}{}{}{}\ottprodnewline
\ottprodline{|}{\Gamma_{S}}{}{}{}{}\ottprodnewline
\ottprodline{|}{\ottnt{H}}{}{}{}{}\ottprodnewline
\ottprodline{|}{\Theta}{}{}{}{}\ottprodnewline
\ottprodline{|}{\ottnt{O}}{}{}{}{}\ottprodnewline
\ottprodline{|}{\ottnt{D}}{}{}{}{}}

\newcommand{\ottVS}{
\ottrulehead{\ottnt{VS}}{::=}{\ottcom{sets}}\ottprodnewline
\ottfirstprodline{|}{ \ottnt{VS_{{\mathrm{1}}}}  \cup  \ottnt{VS_{{\mathrm{2}}}} }{}{}{}{}\ottprodnewline
\ottprodline{|}{ \ottnt{VS_{{\mathrm{1}}}}  \cap  \ottnt{VS_{{\mathrm{2}}}} }{}{}{}{}\ottprodnewline
\ottprodline{|}{ \set{ \ottnt{bind} } }{}{}{}{}\ottprodnewline
\ottprodline{|}{ \set{ \ottnt{bind_{{\mathrm{1}}}} ,\ldots, \ottnt{bind_{\ottmv{n}}} } }{}{}{}{}\ottprodnewline
\ottprodline{|}{\ottsym{\{}  \ottnt{bind_{{\mathrm{1}}}}  \ottsym{,} \, ... \, \ottsym{,}  \ottnt{bind_{\ottmv{n}}}  \ottsym{\}}}{}{}{}{}\ottprodnewline
\ottprodline{|}{\ottnt{FVF}}{}{}{}{}\ottprodnewline
\ottprodline{|}{ \DOM( \ottnt{MF} ) }{}{}{}{}\ottprodnewline
\ottprodline{|}{ \ottnt{VS_{{\mathrm{1}}}}  \setminus  \ottnt{VS_{{\mathrm{2}}}} }{}{}{}{}\ottprodnewline
\ottprodline{|}{ \emptyset }{}{}{}{}\ottprodnewline
\ottprodline{|}{ \textbf{Addr} }{}{}{}{}}

\newcommand{\ottVM}{
\ottrulehead{\mathcal{M}}{::=}{}\ottprodnewline
\ottfirstprodline{|}{\ottsym{[}  \ottnt{H}  \ottsym{,}  \ottnt{R}  \ottsym{]}}{}{}{}{}}

\newcommand{\ottHM}{
\ottrulehead{\mathcal{H}}{::=}{}\ottprodnewline
\ottfirstprodline{|}{\ottsym{[}  \ottnt{H}  \ottsym{,}  v  \ottsym{]}}{}{}{}{}}

\newcommand{\ottEqT}{
\ottrulehead{\ottnt{EqT}}{::=}{\ottcom{equatable terms}}\ottprodnewline
\ottfirstprodline{|}{\ottnt{e}}{}{}{}{}\ottprodnewline
\ottprodline{|}{\Gamma_{S}}{}{}{}{}\ottprodnewline
\ottprodline{|}{\Gamma}{}{}{}{}\ottprodnewline
\ottprodline{|}{\mathcal{L}}{}{}{}{}\ottprodnewline
\ottprodline{|}{v}{}{}{}{}\ottprodnewline
\ottprodline{|}{\ottnt{VS}}{}{}{}{}\ottprodnewline
\ottprodline{|}{\ottnt{functions}}{}{}{}{}\ottprodnewline
\ottprodline{|}{\sigma_{\alpha}}{}{}{}{}\ottprodnewline
\ottprodline{|}{\sigma_{x}}{}{}{}{}\ottprodnewline
\ottprodline{|}{\mathbf{C}}{}{}{}{}\ottprodnewline
\ottprodline{|}{\mathit{termvar}}{}{}{}{}\ottprodnewline
\ottprodline{|}{\ottnt{H}}{}{}{}{}\ottprodnewline
\ottprodline{|}{\ottnt{E}}{}{}{}{}\ottprodnewline
\ottprodline{|}{r}{}{}{}{}\ottprodnewline
\ottprodline{|}{\oldvec{F}}{}{}{}{}\ottprodnewline
\ottprodline{|}{\ottnt{D}}{}{}{}{}}

\newcommand{\ottformula}{
\ottrulehead{\ottnt{formula}}{::=}{}\ottprodnewline
\ottfirstprodline{|}{\ottnt{logical\_frag_{{\mathrm{1}}}}  \ottsym{=}  \ottnt{logical\_frag_{{\mathrm{2}}}}}{}{}{}{}\ottprodnewline
\ottprodline{|}{\ottnt{logical\_frag}}{}{}{}{}\ottprodnewline
\ottprodline{|}{\ottnt{judgment}}{}{}{}{}\ottprodnewline
\ottprodline{|}{\ottnt{functions}}{}{}{}{}\ottprodnewline
\ottprodline{|}{\ottnt{logic\_judgment}}{}{}{}{}\ottprodnewline
\ottprodline{|}{ r_{{\mathrm{1}}}   \ottnt{rel\_op}   r_{{\mathrm{2}}} }{}{}{}{}\ottprodnewline
\ottprodline{|}{ r_{{\mathrm{1}}}  &  \ottnt{rel\_op}   r_{{\mathrm{2}}} }{}{}{}{}\ottprodnewline
\ottprodline{|}{\ottnt{formula_{{\mathrm{1}}}}  \iff  \ottnt{formula_{{\mathrm{2}}}}}{}{}{}{}\ottprodnewline
\ottprodline{|}{\ottnt{formula_{{\mathrm{1}}}}  \implies  \ottnt{formula_{{\mathrm{2}}}}}{}{}{}{}\ottprodnewline
\ottprodline{|}{ \ottnt{formula} \,\,(x \in \DOM( \Gamma )) }{}{}{}{}\ottprodnewline
\ottprodline{|}{\Theta  \ottsym{(}  \mathit{f}  \ottsym{)}  \ottsym{=}  \sigma}{}{}{}{}\ottprodnewline
\ottprodline{|}{ \ottnt{bind}  \not\in \DOM( \ottnt{MF} ) }{}{}{}{}\ottprodnewline
\ottprodline{|}{ \ottnt{bind}  \in \DOM( \ottnt{MF} ) }{}{}{}{}\ottprodnewline
\ottprodline{|}{ \ottnt{bind}  \not\in \DOM( \ottnt{MF} ) }{}{}{}{}\ottprodnewline
\ottprodline{|}{ \ottnt{bind}  \in  \ottnt{VS} }{}{}{}{}\ottprodnewline
\ottprodline{|}{ \ottnt{bind}  \not\in  \ottnt{VS} }{}{}{}{}\ottprodnewline
\ottprodline{|}{ \ottnt{VS_{{\mathrm{1}}}}  \subseteq  \ottnt{VS_{{\mathrm{2}}}} }{}{}{}{}\ottprodnewline
\ottprodline{|}{ v  \in  \mathbb{Z} }{}{}{}{}\ottprodnewline
\ottprodline{|}{ \ottnt{d}  \in  \ottnt{D} }{}{}{}{}\ottprodnewline
\ottprodline{|}{\ottnt{formula_{{\mathrm{1}}}} \quad ... \quad \ottnt{formula_{\ottmv{n}}}}{}{}{}{}\ottprodnewline
\ottprodline{|}{\ottnt{EqT_{{\mathrm{1}}}}  \ottsym{=} \, ... \, \ottsym{=}  \ottnt{EqT_{{\mathrm{2}}}}}{}{}{}{}\ottprodnewline
\ottprodline{|}{ \ottnt{EqT_{{\mathrm{1}}}}  & =  \ottnt{EqT_{{\mathrm{2}}}} }{}{}{}{}\ottprodnewline
\ottprodline{|}{r_{{\mathrm{1}}}  \le \, ... \, \le  r_{{\mathrm{2}}}}{}{}{}{}\ottprodnewline
\ottprodline{|}{r_{{\mathrm{1}}}  \ottsym{=} \, ... \, \ottsym{=}  r_{{\mathrm{2}}} \, \ottnt{rel\_op} \, r_{{\mathrm{3}}}}{}{}{}{}\ottprodnewline
\ottprodline{|}{ \ottnt{EqT_{{\mathrm{1}}}}  \neq  \ottnt{EqT_{{\mathrm{2}}}} }{}{}{}{}\ottprodnewline
\ottprodline{|}{\ottnt{formula_{{\mathrm{1}}}}  \wedge  \ottnt{formula_{{\mathrm{2}}}}}{}{}{}{}\ottprodnewline
\ottprodline{|}{\forall \, \ottnt{formula_{{\mathrm{1}}}}  \ottsym{.}  \ottnt{formula_{{\mathrm{2}}}}}{}{}{}{}\ottprodnewline
\ottprodline{|}{\exists \, v  \ottsym{.}  \ottnt{formula}}{}{}{}{}\ottprodnewline
\ottprodline{|}{\ottsym{(}  \ottnt{formula}  \ottsym{)}}{}{}{}{}\ottprodnewline
\ottprodline{|}{\tau_{{\mathrm{1}}}  \approx  \tau_{{\mathrm{2}}}}{}{}{}{}\ottprodnewline
\ottprodline{|}{ \ottnt{H_{{\mathrm{1}}}}   \approx _ \ottmv{a}   \ottnt{H_{{\mathrm{2}}}} }{}{}{}{}\ottprodnewline
\ottprodline{|}{ \ottnt{R_{{\mathrm{1}}}}  \sqsubseteq  \ottnt{R_{{\mathrm{2}}}} }{}{}{}{}}

\newcommand{\ottjudgement}{
\ottrulehead{\ottnt{judgement}}{::=}{}}

\newcommand{\ottuserXXsyntax}{
\ottrulehead{\ottnt{user\_syntax}}{::=}{}\ottprodnewline
\ottfirstprodline{|}{\mathit{termvar}}{}{}{}{}\ottprodnewline
\ottprodline{|}{\mathit{c}}{}{}{}{}\ottprodnewline
\ottprodline{|}{\mathit{funvar}}{}{}{}{}\ottprodnewline
\ottprodline{|}{\ottmv{index}}{}{}{}{}\ottprodnewline
\ottprodline{|}{\ottmv{a}}{}{}{}{}\ottprodnewline
\ottprodline{|}{\phi}{}{}{}{}\ottprodnewline
\ottprodline{|}{\ottnt{e}}{}{}{}{}\ottprodnewline
\ottprodline{|}{\ottnt{return\_hole}}{}{}{}{}\ottprodnewline
\ottprodline{|}{\ottnt{rhs}}{}{}{}{}\ottprodnewline
\ottprodline{|}{\ottnt{arg\_list}}{}{}{}{}\ottprodnewline
\ottprodline{|}{\lambda}{}{}{}{}\ottprodnewline
\ottprodline{|}{\ottnt{v}}{}{}{}{}\ottprodnewline
\ottprodline{|}{\ottnt{fun\_body}}{}{}{}{}\ottprodnewline
\ottprodline{|}{\ottnt{d}}{}{}{}{}\ottprodnewline
\ottprodline{|}{\ottnt{D}}{}{}{}{}\ottprodnewline
\ottprodline{|}{\ottnt{P}}{}{}{}{}\ottprodnewline
\ottprodline{|}{r}{}{}{}{}\ottprodnewline
\ottprodline{|}{\top}{}{}{}{}\ottprodnewline
\ottprodline{|}{\ottnt{float}}{}{}{}{}\ottprodnewline
\ottprodline{|}{\ottnt{DL}}{}{}{}{}\ottprodnewline
\ottprodline{|}{n}{}{}{}{}\ottprodnewline
\ottprodline{|}{\ell}{}{}{}{}\ottprodnewline
\ottprodline{|}{\ottnt{digit}}{}{}{}{}\ottprodnewline
\ottprodline{|}{\mathbf{C}}{}{}{}{}\ottprodnewline
\ottprodline{|}{\ottnt{R}}{}{}{}{}\ottprodnewline
\ottprodline{|}{\ottnt{H}}{}{}{}{}\ottprodnewline
\ottprodline{|}{v}{}{}{}{}\ottprodnewline
\ottprodline{|}{\ottnt{E}}{}{}{}{}\ottprodnewline
\ottprodline{|}{F}{}{}{}{}\ottprodnewline
\ottprodline{|}{\oldvec{F}}{}{}{}{}\ottprodnewline
\ottprodline{|}{\ottnt{ineq}}{}{}{}{}\ottprodnewline
\ottprodline{|}{\mathcal{C}}{}{}{}{}\ottprodnewline
\ottprodline{|}{\ottnt{prhs}}{}{}{}{}\ottprodnewline
\ottprodline{|}{\ottnt{prefix\_op}}{}{}{}{}\ottprodnewline
\ottprodline{|}{\mathcal{CP}}{}{}{}{}\ottprodnewline
\ottprodline{|}{\oldvec{\ell}}{}{}{}{}\ottprodnewline
\ottprodline{|}{\mathcal{L}}{}{}{}{}\ottprodnewline
\ottprodline{|}{\ottnt{ca}}{}{}{}{}\ottprodnewline
\ottprodline{|}{\ottnt{context\_args}}{}{}{}{}\ottprodnewline
\ottprodline{|}{\varphi}{}{}{}{}\ottprodnewline
\ottprodline{|}{\ottnt{pvar}}{}{}{}{}\ottprodnewline
\ottprodline{|}{\pi}{}{}{}{}\ottprodnewline
\ottprodline{|}{\oldvec{\star}}{}{}{}{}\ottprodnewline
\ottprodline{|}{\ottnt{AP\_step}}{}{}{}{}\ottprodnewline
\ottprodline{|}{\widehat{v}}{}{}{}{}\ottprodnewline
\ottprodline{|}{\sigma_{\alpha}}{}{}{}{}\ottprodnewline
\ottprodline{|}{\sigma_{x}}{}{}{}{}\ottprodnewline
\ottprodline{|}{\tau}{}{}{}{}\ottprodnewline
\ottprodline{|}{\ottnt{tuple\_bind}}{}{}{}{}\ottprodnewline
\ottprodline{|}{\tau_{S}}{}{}{}{}\ottprodnewline
\ottprodline{|}{\Gamma_{S}}{}{}{}{}\ottprodnewline
\ottprodline{|}{\ottnt{tyarg\_list}}{}{}{}{}\ottprodnewline
\ottprodline{|}{\sigma}{}{}{}{}\ottprodnewline
\ottprodline{|}{\Theta}{}{}{}{}\ottprodnewline
\ottprodline{|}{\Gamma}{}{}{}{}\ottprodnewline
\ottprodline{|}{\ottnt{rel\_op}}{}{}{}{}\ottprodnewline
\ottprodline{|}{p}{}{}{}{}\ottprodnewline
\ottprodline{|}{\ottnt{terminals}}{}{}{}{}\ottprodnewline
\ottprodline{|}{\ottnt{reduction}}{}{}{}{}\ottprodnewline
\ottprodline{|}{\ottnt{logical\_frag}}{}{}{}{}\ottprodnewline
\ottprodline{|}{\ottnt{logic\_judgment}}{}{}{}{}\ottprodnewline
\ottprodline{|}{\ottnt{FVF}}{}{}{}{}\ottprodnewline
\ottprodline{|}{\ottnt{functions}}{}{}{}{}\ottprodnewline
\ottprodline{|}{\ottnt{O}}{}{}{}{}\ottprodnewline
\ottprodline{|}{\ottnt{step\_relation}}{}{}{}{}\ottprodnewline
\ottprodline{|}{\ottnt{judgment}}{}{}{}{}\ottprodnewline
\ottprodline{|}{\ottnt{bind}}{}{}{}{}\ottprodnewline
\ottprodline{|}{\ottnt{MF}}{}{}{}{}\ottprodnewline
\ottprodline{|}{\ottnt{VS}}{}{}{}{}\ottprodnewline
\ottprodline{|}{\mathcal{M}}{}{}{}{}\ottprodnewline
\ottprodline{|}{\mathcal{H}}{}{}{}{}\ottprodnewline
\ottprodline{|}{\ottnt{EqT}}{}{}{}{}\ottprodnewline
\ottprodline{|}{\ottnt{formula}}{}{}{}{}}

\newcommand{\ottgrammar}{\ottgrammartabular{
\otte\ottinterrule
\ottreturnXXhole\ottinterrule
\ottrhs\ottinterrule
\ottargXXlist\ottinterrule
\ottA\ottinterrule
\ottv\ottinterrule
\ottfunXXbody\ottinterrule
\ottd\ottinterrule
\ottD\ottinterrule
\ottP\ottinterrule
\ottr\ottinterrule
\otttop\ottinterrule
\ottfloat\ottinterrule
\ottDL\ottinterrule
\ottnn\ottinterrule
\ottl\ottinterrule
\ottdigit\ottinterrule
\ottC\ottinterrule
\ottR\ottinterrule
\ottH\ottinterrule
\ottRv\ottinterrule
\ottE\ottinterrule
\ottCE\ottinterrule
\ottEs\ottinterrule
\ottineq\ottinterrule
\ottplhs\ottinterrule
\ottprhs\ottinterrule
\ottprefixXXop\ottinterrule
\ottCQP\ottinterrule
\ottll\ottinterrule
\ottB\ottinterrule
\ottca\ottinterrule
\ottcontextXXargs\ottinterrule
\ottph\ottinterrule
\ottpvar\ottinterrule
\ottAP\ottinterrule
\ottw\ottinterrule
\ottAPXXstep\ottinterrule
\ottpval\ottinterrule
\ottcsub\ottinterrule
\ottvsub\ottinterrule
\ottT\ottinterrule
\otttupleXXbind\ottinterrule
\ottST\ottinterrule
\ottGst\ottinterrule
\otttyargXXlist\ottinterrule
\ottFT\ottinterrule
\ottTh\ottinterrule
\ottG\ottinterrule
\ottrelXXop\ottinterrule
\ottpp\ottinterrule
\ottterminals\ottinterrule
\ottreduction\ottinterrule
\ottlogicalXXfrag\ottinterrule
\ottlogicXXjudgment\ottinterrule
\ottFVF\ottinterrule
\ottfunctions\ottinterrule
\ottO\ottinterrule
\ottstepXXrelation\ottinterrule
\ottjudgment\ottinterrule
\ottbind\ottinterrule
\ottMF\ottinterrule
\ottVS\ottinterrule
\ottVM\ottinterrule
\ottHM\ottinterrule
\ottEqT\ottinterrule
\ottformula\ottinterrule
\ottjudgement\ottinterrule
\ottuserXXsyntax\ottafterlastrule
}}

% defnss
\newcommand{\ottdefnss}{
}

\newcommand{\ottall}{\ottmetavars\\[0pt]
\ottgrammar\\[5.0mm]
\ottdefnss}


\makeatletter
\def\admitted@what[#1]{\textcolor{red}{\textbf{[Admitted: that #1]}}}
\def\admitted@{\textcolor{red}{\textbf{Admitted}}\xspace}
\def\admitted{\@ifnextchar[{\admitted@what}{\admitted@}}
\makeatother
\begin{document}
Types and refinements:
\[
  \begin{array}{rrcl}
      \text{\scriptsize Types} & \tau % & \in & \mathbf{Types} \\
                                         &::=&  \set{  \nu  \COL \TINT \mid  \varphi }  \mid  \tau  \TREF^{ r }  \\
      \text{\scriptsize Ownership} & r & \in & [0,1] \\
      \text{\scriptsize Refinements} & \varphi & ::= & \varphi_{{\mathrm{1}}}  \vee  \varphi_{{\mathrm{2}}} \mid  \neg  \varphi  \mid  \top  \\
                               & & \mid & \phi  \ottsym{(}  \widehat{v}_{{\mathrm{1}}}  \ottsym{,} \, .. \, \ottsym{,}  \widehat{v}_{\ottmv{n}}  \ottsym{)} \\
                               & & \mid & \widehat{v}_{{\mathrm{1}}} \, \ottsym{=} \, \widehat{v}_{{\mathrm{2}}} \\
                                 & & \mid & \mathcal{CP} \\
    \text{\scriptsize Refinement Values} & \widehat{v} & ::= & \pi \mid n \mid \nu \\
    \text{\scriptsize Access Paths} & \pi & ::= & \mathit{x} \, \oldvec{\star} \\
    \text{\scriptsize Function Types} & \sigma & ::= &  \forall  \lambda .\tuple{ \mathit{x_{{\mathrm{1}}}} \COL \tau_{{\mathrm{1}}} ,\dots, \mathit{x_{\ottmv{n}}} \COL \tau_{\ottmv{n}} } \\ & & & \ra\tuple{ \mathit{x_{{\mathrm{1}}}} \COL \tau'_{{\mathrm{1}}} ,\dots, \mathit{x_{\ottmv{n}}} \COL \tau'_{\ottmv{n}}  \mid  \tau }  \\
    \text{\scriptsize Context Variables} & \lambda & \in & \CVar \\
    \text{\scriptsize Concrete Context} & \oldvec{\ell} & ::= & \ell  \ottsym{:}  \oldvec{\ell} \mid  \epsilon  \\
    \text{\scriptsize Pred. Context} & \mathcal{C} & ::= &  \ell  :  \mathcal{C}  \mid \lambda \mid  \epsilon  \\
    \text{\scriptsize Context Query} & \mathcal{CP} & ::= &   \oldvec{\ell}     \subseteq    \mathcal{C}  \\
    \text{\scriptsize Typing Context} & \mathcal{L} & ::= & \lambda \mid \oldvec{\ell} \\
  \end{array}
\]

An access path denotes a path through memory by a root variable and a potentially empty sequence of references $\oldvec{\star}$.
The empty sequence is denoted $ \epsilon $. We abbreviate $\mathit{x} \,  \epsilon $ as $\mathit{x}$.

Well-formedness:
\infrule[WF-Env]{
  \forall \,  \mathit{x}  \in \DOM( \Gamma )   \ottsym{.}   \mathcal{L}   \mid   \Gamma   \vdash _{\wf}  \Gamma  \ottsym{(}  \mathit{x}  \ottsym{)} 
}{
   \mathcal{L}   \vdash _{\wf}  \Gamma 
}
\infrule[WF-Int]{
   \mathcal{L}   \mid   \Gamma   \vdash _{\wf}  \varphi 
}{
   \mathcal{L}   \mid   \Gamma   \vdash _{\wf}   \set{  \nu  \COL \TINT \mid  \varphi }   
}
\infrule[WF-Ref]{
   \mathcal{L}   \mid   \Gamma   \vdash _{\wf}  \tau 
}{
   \mathcal{L}   \mid   \Gamma   \vdash _{\wf}   \tau  \TREF^{ r }  
}
\infrule[WF-Phi]{
  \Gamma  \vdash  \varphi \\
  \mathbf{FCV} \, \ottsym{(}  \varphi  \ottsym{)} \subseteq \ottkw{CV} \, \ottsym{(}  \mathcal{L}  \ottsym{)}
}{
   \mathcal{L}   \mid   \Gamma   \vdash _{\wf}  \varphi 
}
\infrule[WF-Result]{
   \mathcal{L}   \mid   \Gamma   \vdash _{\wf}  \tau  \andalso
   \mathcal{L}   \vdash _{\wf}  \Gamma 
}{
   \mathcal{L}   \vdash _{\wf}  \tau   \produces   \Gamma 
}
\infrule[WF-FunType]{
   \lambda   \vdash _{\wf}   \mathit{x_{{\mathrm{1}}}} \COL \tau_{{\mathrm{1}}} ,\ldots, \mathit{x_{\ottmv{n}}} \COL \tau_{\ottmv{n}}   \\  \lambda   \vdash _{\wf}  \tau   \produces    \mathit{x_{{\mathrm{1}}}} \COL \tau'_{{\mathrm{1}}} ,\ldots, \mathit{x_{\ottmv{n}}} \COL \tau'_{\ottmv{n}}  
}{
   \vdash _{\wf}   \forall  \lambda .\tuple{ \mathit{x_{{\mathrm{1}}}} \COL \tau_{{\mathrm{1}}} ,\dots, \mathit{x_{\ottmv{n}}} \COL \tau_{\ottmv{n}} }\ra\tuple{ \mathit{x_{{\mathrm{1}}}} \COL \tau'_{{\mathrm{1}}} ,\dots, \mathit{x_{\ottmv{n}}} \COL \tau'_{\ottmv{n}}  \mid  \tau }  
}
\infrule[WF-FunEnv]{
  \forall \,  \mathit{f}  \in \DOM( \Theta )   \ottsym{.}   \vdash _{\wf}  \Theta  \ottsym{(}  \mathit{f}  \ottsym{)} 
}{
   \vdash _{\wf}  \Theta 
}


Well-typed predicates:

\begin{center}
\bcprulessavespacetrue
\infax[Pr-Top]{ \Gamma  \vdash   \top  }
\infax[Pr-Top]{ \Gamma  \vdash  \mathcal{CP} }
\infrule[Pr-Not]{
  \Gamma  \vdash  \varphi
}{ \Gamma  \vdash   \neg  \varphi  }
\infrule[Pr-Or]{
  \Gamma  \vdash  \varphi_{{\mathrm{1}}} \andalso \Gamma  \vdash  \varphi_{{\mathrm{2}}}
}{ \Gamma  \vdash  \varphi_{{\mathrm{1}}}  \vee  \varphi_{{\mathrm{2}}} }

\infrule[Pr-Eq]{
  \Gamma  \vdash  \widehat{v}_{{\mathrm{1}}} \andalso \Gamma  \vdash  \widehat{v}_{{\mathrm{2}}}
}{ \Gamma  \vdash  \widehat{v}_{{\mathrm{1}}} \, \ottsym{=} \, \widehat{v}_{{\mathrm{2}}} }
\infrule[Pr-App]{
  \Gamma  \vdash  \widehat{v}_{{\mathrm{1}}} \andalso \cdots \andalso \Gamma  \vdash  \widehat{v}_{\ottmv{n}}
}{ \Gamma  \vdash  \phi  \ottsym{(}  \widehat{v}_{{\mathrm{1}}}  \ottsym{,} \, .. \, \ottsym{,}  \widehat{v}_{\ottmv{n}}  \ottsym{)}}
\bcprulessavespacefalse
\end{center}

Well-typed predicate values
\begin{center}
  \bcprulessavespacetrue
  \infax[Pv-Int]{ \Gamma  \vdash  n }
  \infax[Pv-Nu]{ \Gamma  \vdash  \nu }
  \infrule[Pv-AP]{
     \oldvec{\star}  \Downarrow  \Gamma  \ottsym{(}  \mathit{x}  \ottsym{)} 
  }{ \Gamma  \vdash  \mathit{x} \, \oldvec{\star} }

  \infax[AP-Eps]{   \epsilon   \Downarrow   \set{  \nu  \COL \TINT \mid \_ }   }
  \infrule[AP-Cons]{
     \oldvec{\star}  \Downarrow  \tau  \andalso r > 0
  }{
      \star  \, \oldvec{\star}  \Downarrow   \tau  \TREF^{ r }  
  }
  \bcprulessavespacefalse
\end{center}

The addition operator is defined as in the ESOP 2020 paper.

We assume that $\mathit{x} \, \oldvec{\star}$ is a valid variable in the underlying logic; it can be lifted to one using consistent substitution.

The denotation operation is defined as:

\begin{align*}
   \sem{ \Gamma  \ottsym{,}  \mathit{x}  \ottsym{:}  \tau }  & =  \sem{ \tau }_{ \mathit{x} }   \wedge   \sem{ \Gamma }  \\
   \sem{  \bullet  }  & =  \top  \\
   \sem{  \set{  \nu  \COL \TINT \mid  \varphi }  }_{ \pi }  & =  [  \pi  /  \nu  ]  \, \varphi \\
   \sem{  \tau  \TREF^{ r }  }_{ \mathit{x} \, \oldvec{\star} }  & =  \sem{ \tau }_{ \mathit{x} \, \oldvec{\star} \,  \star  }  
\end{align*}

We define a new strengthening operation $\tau  \wedge \, \nu \, \ottsym{=}  \pi$ as:

\begin{align*} 
   \set{  \nu  \COL \TINT \mid  \varphi }   \wedge \, \nu \, \ottsym{=}  \pi & \triangleq  \set{  \nu  \COL \TINT \mid   \varphi  \wedge  \nu \, \ottsym{=} \, \pi  }  \\
   \tau  \TREF^{ \ottsym{0} }   \wedge \, \nu \, \ottsym{=}  \pi & \triangleq  \tau  \TREF^{ \ottsym{0} }  \\
   \tau  \TREF^{ r }   \wedge \, \nu \, \ottsym{=}  \pi & \triangleq  \ottsym{(}  \tau  \wedge \, \nu \, \ottsym{=}  \pi \,  \star   \ottsym{)}  \TREF^{ r }  \text{ if ($r > 0$)}
\end{align*}

We now describe the type rules for the extended type system.
We omit the rules for \rn{T-Assert, T-Seq, T-If, T-LetInt, T-Var, T-Alias, T-AliasPtr, T-Sub} as they are unchanged.

\infrule[T-Assign]{
  (\text{The shapes of $\tau'$ and $\tau_{{\mathrm{2}}}$ are similar}) \\
   \mathcal{L}   \vdash _{\wf}   \Gamma  \setminus  \mathit{y}   \\
   \Theta   \mid   \mathcal{L}   \mid   \Gamma  \ottsym{[}  \mathit{x}  \hookleftarrow  \tau_{{\mathrm{1}}}  \wedge \, \nu \, \ottsym{=}  \mathit{y} \,  \star   \ottsym{]}  \ottsym{[}  \mathit{y}  \ottsym{:}   \ottsym{(}  \tau_{{\mathrm{2}}}  \wedge \, \nu \, \ottsym{=}  \mathit{x}  \ottsym{)}  \TREF^{ \ottsym{1} }   \ottsym{]}   \vdash   \ottnt{e}  :  \tau   \produces   \Gamma' 
}{  \Theta   \mid   \mathcal{L}   \mid   \Gamma  \ottsym{[}  \mathit{x}  \ottsym{:}  \tau_{{\mathrm{1}}}  \ottsym{+}  \tau_{{\mathrm{2}}}  \ottsym{]}  \ottsym{[}  \mathit{y}  \ottsym{:}   \tau'  \TREF^{ \ottsym{1} }   \ottsym{]}   \vdash    \mathit{y}  \WRITE  \mathit{x}  \SEQ  \ottnt{e}   :  \tau   \produces   \Gamma'  }

\infrule[T-Let]{
   \Theta   \mid   \mathcal{L}   \mid   \Gamma  \ottsym{[}  \mathit{x}  \hookleftarrow  \tau_{{\mathrm{1}}}  \wedge \, \nu \, \ottsym{=}  \mathit{y}  \ottsym{]}  \ottsym{,}  \mathit{y}  \ottsym{:}  \tau_{{\mathrm{2}}}  \wedge \, \nu \, \ottsym{=}  \mathit{x}   \vdash   \ottnt{e}  :  \tau   \produces   \Gamma'  \andalso  \mathit{x}  \not\in   \DOM( \Gamma )  
}{ \Theta   \mid   \mathcal{L}   \mid   \Gamma  \ottsym{[}  \mathit{x}  \ottsym{:}  \tau_{{\mathrm{1}}}  \ottsym{+}  \tau_{{\mathrm{2}}}  \ottsym{]}   \vdash    \LET  \mathit{x}  =  \mathit{y}  \IN  \ottnt{e}   :  \tau   \produces   \Gamma' }

\infrule[T-Deref]{
  \tau'_{{\mathrm{1}}} = \begin{cases}
    \tau_{{\mathrm{1}}}  \wedge \, \nu \, \ottsym{=}  \mathit{x} &  r   \ottsym{>}   \ottsym{0}  \\
    \tau_{{\mathrm{1}}} & r  \ottsym{=}  \ottsym{0}
  \end{cases}  \\
  \tau'_{{\mathrm{2}}} = \begin{cases}
    \tau_{{\mathrm{2}}}  \wedge \, \nu \, \ottsym{=}  \mathit{y} \,  \star  &  r   \ottsym{>}   \ottsym{0}  \\
    \tau_{{\mathrm{2}}} & r  \ottsym{=}  \ottsym{0}
  \end{cases} \\
   \Theta   \mid   \mathcal{L}   \mid   \Gamma  \ottsym{[}  \mathit{y}  \hookleftarrow   \tau'_{{\mathrm{1}}}  \TREF^{ r }   \ottsym{]}  \ottsym{,}  \mathit{x}  \ottsym{:}  \tau'_{{\mathrm{2}}}   \vdash   \ottnt{e}  :  \tau   \produces   \Gamma'  \\
   \mathit{x}  \not\in   \DOM( \Gamma' )  
}{
   \Theta   \mid   \mathcal{L}   \mid   \Gamma  \ottsym{[}  \mathit{y}  \ottsym{:}   \ottsym{(}  \tau_{{\mathrm{1}}}  \ottsym{+}  \tau_{{\mathrm{2}}}  \ottsym{)}  \TREF^{ r }   \ottsym{]}   \vdash    \LET  \mathit{x}  =   *  \mathit{y}   \IN  \ottnt{e}   :  \tau   \produces   \Gamma' 
}

\infrule[T-MkRef]{
   \Theta   \mid   \mathcal{L}   \mid   \Gamma  \ottsym{[}  \mathit{y}  \hookleftarrow  \tau_{{\mathrm{1}}}  \wedge \, \nu \, \ottsym{=}  \mathit{x} \,  \star   \ottsym{]}  \ottsym{,}  \mathit{x}  \ottsym{:}   \ottsym{(}  \tau_{{\mathrm{2}}}  \wedge \, \nu \, \ottsym{=}  \mathit{y}  \ottsym{)}  \TREF^{ \ottsym{1} }    \vdash   \ottnt{e}  :  \tau   \produces   \Gamma'  \\
   \mathit{x}  \not\in   \DOM( \Gamma' )  
}{  \Theta   \mid   \mathcal{L}   \mid   \Gamma  \ottsym{[}  \mathit{y}  \ottsym{:}  \tau_{{\mathrm{1}}}  \ottsym{+}  \tau_{{\mathrm{2}}}  \ottsym{]}   \vdash    \LET  \mathit{x}  =   \MKREF  \mathit{y}   \IN  \ottnt{e}   :  \tau   \produces   \Gamma'  }

\infrule[T-Call]{
  \sigma_{\alpha}  \ottsym{=}  \ottsym{[}  \ell  \ottsym{:}  \mathcal{L}  \ottsym{/}  \lambda  \ottsym{]} \andalso \sigma_{x}  \ottsym{=}    [  \mathit{y_{{\mathrm{1}}}}  /  \mathit{x_{{\mathrm{1}}}}  ]  \cdots  [  \mathit{y_{\ottmv{n}}}  /  \mathit{x_{\ottmv{n}}}  ]   \\
  \Theta  \ottsym{(}  \mathit{f}  \ottsym{)}  \ottsym{=}   \forall  \lambda .\tuple{ \mathit{x_{{\mathrm{1}}}} \COL \tau_{{\mathrm{1}}} ,\dots, \mathit{x_{\ottmv{n}}} \COL \tau_{\ottmv{n}} }\ra\tuple{ \mathit{x_{{\mathrm{1}}}} \COL \tau'_{{\mathrm{1}}} ,\dots, \mathit{x_{\ottmv{n}}} \COL \tau'_{\ottmv{n}}  \mid  \tau }  \\
  \Gamma_{{\mathrm{1}}}  \ottsym{(}  \mathit{y_{\ottmv{i}}}  \ottsym{)}  \ottsym{=}  \tau''_{\ottmv{i}}  \ottsym{+}  \sigma_{\alpha} \, \sigma_{x} \, \tau_{\ottmv{i}} \\
  \Gamma_{{\mathrm{2}}}  \ottsym{[}  \mathit{y_{\ottmv{i}}}  \hookleftarrow  \tau''_{\ottmv{i}}  \ottsym{]} \andalso  \mathcal{L}   \vdash _{\wf}  \Gamma_{{\mathrm{2}}}  \\
  \Gamma_{{\mathrm{3}}}  \ottsym{=}  \Gamma_{{\mathrm{1}}}  \ottsym{[}  \mathit{y_{\ottmv{i}}}  \hookleftarrow  \tau''_{\ottmv{i}}  \ottsym{+}  \sigma_{\alpha} \, \sigma_{x} \, \tau'_{\ottmv{i}}  \ottsym{]}  \ottsym{,}  \mathit{z}  \ottsym{:}  \sigma_{\alpha} \, \sigma_{x} \, \tau \\
   \Theta   \mid   \mathcal{L}   \mid   \Gamma_{{\mathrm{3}}}   \vdash   \ottnt{e}  :  \tau'   \produces   \Gamma_{{\mathrm{4}}}  \andalso  \mathit{z}  \not\in   \DOM( \Gamma_{{\mathrm{4}}} )  
}{
   \Theta   \mid   \mathcal{L}   \mid   \Gamma_{{\mathrm{1}}}   \vdash    \LET  \mathit{z}  =   \mathit{f} ^ \ell (  \mathit{y_{{\mathrm{1}}}} ,\ldots, \mathit{y_{\ottmv{n}}}  )   \IN  \ottnt{e}   :  \tau'   \produces   \Gamma_{{\mathrm{4}}} 
}

\infrule[T-Frame]{
   \mathcal{L}   \vdash _{\wf}  \Gamma_{\ottmv{p}}  \\
   \Theta   \mid   \mathcal{L}   \mid   \Gamma   \vdash   \ottnt{e}  :  \tau   \produces   \Gamma' 
}{
   \Theta   \mid   \mathcal{L}   \mid   \Gamma  \ottsym{+}  \Gamma_{\ottmv{p}}   \vdash   \ottnt{e}  :  \tau   \produces   \Gamma'  \ottsym{+}  \Gamma_{\ottmv{p}} 
}
  

\section{Proofs}

Define the partial type lookup operation $\Gamma  \ottsym{(}  \pi  \ottsym{)}$ as:

\begin{align*}
  \Gamma  \ottsym{(}  \mathit{x} \, \oldvec{\star}  \ottsym{)} & = \Gamma  \ottsym{(}  \mathit{x}  \ottsym{)}  \ottsym{(}  \oldvec{\star}  \ottsym{)} \\
  \tau  \ottsym{(}   \epsilon   \ottsym{)} & = \tau \\
  \ottsym{(}   \tau'  \TREF^{ r }   \ottsym{)}  \ottsym{(}   \star  \, \oldvec{\star}  \ottsym{)} & = \tau'  \ottsym{(}  \oldvec{\star}  \ottsym{)}
\end{align*}

\JT{defining this traversal as a map operation on $\tau$ is gross}

$\ottsym{[}  \ottnt{H}  \ottsym{,}  \ottnt{v}  \ottsym{]}$ is the partial function from $\oldvec{\star}$ to values $v$
defined by

\begin{align*}
  \ottsym{[}  \ottnt{H}  \ottsym{,}  \ottnt{v}  \ottsym{]}  \ottsym{(}   \epsilon   \ottsym{)} \, \ottsym{=} \, \ottnt{v} && \ottsym{[}  \ottnt{H}  \ottsym{,}  \ottnt{v}  \ottsym{]}  \ottsym{(}   \star  \, \oldvec{\star}  \ottsym{)} = \begin{cases}
    H(a) & \text{if } \ottsym{[}  \ottnt{H}  \ottsym{,}  \ottnt{v}  \ottsym{]}  \ottsym{(}  \oldvec{\star}  \ottsym{)} \, \ottsym{=} \, \ottmv{a}  \wedge   \ottmv{a}  \in   \DOM( \ottnt{H} )   \\
    \mathit{undef} & o.w.
  \end{cases}
\end{align*}

$\ottsym{[}  \ottnt{H}  \ottsym{,}  \ottnt{R}  \ottsym{]}$ is the partial map from $\pi$ to values $v$ defined by
$\ottsym{[}  \ottnt{H}  \ottsym{,}  \ottnt{R}  \ottsym{]}  \ottsym{(}  \mathit{x} \, \oldvec{\star}  \ottsym{)} \, \ottsym{=} \, \ottsym{[}  \ottnt{H}  \ottsym{,}  \ottnt{R}  \ottsym{(}  \mathit{x}  \ottsym{)}  \ottsym{]}  \ottsym{(}  \oldvec{\star}  \ottsym{)}$

\begin{lemma}\label{lemma:framed-update-pred}
  If $ \mathcal{L}   \mid    \Gamma  \setminus  \mathit{x}    \vdash _{\wf}  \varphi $ and for all $y \neq x$ we have
  $\ottkw{own} \, \ottsym{(}  \ottnt{H}  \ottsym{,}  \ottnt{R}  \ottsym{(}  \mathit{y}  \ottsym{)}  \ottsym{,}  \Gamma  \ottsym{(}  \mathit{y}  \ottsym{)}  \ottsym{)}  \ottsym{(}  \ottmv{a}  \ottsym{)}  \ottsym{=}  \ottsym{0}$, then $\ottsym{[}  \ottnt{H}  \ottsym{,}  \ottnt{R}  \ottsym{]} \, \ottsym{[}  n  \ottsym{/}  \nu  \ottsym{]}  \varphi$
  is equivalent to $\ottsym{[}  \ottnt{H}  \ottsym{\{}  \ottmv{a}  \hookleftarrow  v'  \ottsym{\}}  \ottsym{,}  \ottnt{R}  \ottsym{]} \, \ottsym{[}  n  \ottsym{/}  \nu  \ottsym{]}  \varphi$ where $\ottnt{R}  \ottsym{(}  \mathit{x}  \ottsym{)} \, \ottsym{=} \, \ottmv{a}$.
\end{lemma}
\begin{proof}
  Suppose not. Then there must be some access path $\pi$ in $\varphi$ such that
  for some prefix of the path (called $\pi'$) we have $\ottsym{[}  \ottnt{H}  \ottsym{,}  \ottnt{R}  \ottsym{]}  \ottsym{(}  \pi'  \ottsym{)} \, \ottsym{=} \, \ottmv{a}$. From
  $ \mathcal{L}   \mid    \Gamma  \setminus  \mathit{x}    \vdash _{\wf}  \varphi $ we must have that $\pi'$ cannot be rooted in $\mathit{x}$,
  and must therefore be rooted in some other variable $\mathit{z}$, whereby
  $\ottkw{own} \, \ottsym{(}  \ottnt{H}  \ottsym{,}  \ottnt{R}  \ottsym{(}  \mathit{z}  \ottsym{)}  \ottsym{,}  \Gamma  \ottsym{(}  \mathit{z}  \ottsym{)}  \ottsym{)}  \ottsym{(}  \ottmv{a}  \ottsym{)}  \ottsym{=}  \ottsym{0}$. But we must then have $\Gamma  \ottsym{(}  \pi'  \ottsym{)}  \ottsym{=}   \tau'  \TREF^{ \ottsym{0} } $, which
  contradicts our assumption that $ \mathcal{L}   \mid    \Gamma  \setminus  \mathit{x}    \vdash _{\wf}  \varphi $.
\end{proof}

\begin{lemma}
  For any $\mathit{x}, \ottmv{a}, \ottnt{R}, \ottnt{H}$, $\Gamma$, and $n$ such that
  $\ottnt{R}  \ottsym{(}  \mathit{x}  \ottsym{)} \, \ottsym{=} \, \ottmv{a}$, $ \ottnt{H} \vdash   v'  \Downarrow  n $, $ \ottnt{H} \vdash   \ottnt{H}  \ottsym{(}  \ottmv{a}  \ottsym{)}  \Downarrow  n $ and where for all
  $y \neq x$ we have $\ottkw{own} \, \ottsym{(}  \ottnt{H}  \ottsym{,}  \ottnt{R}  \ottsym{(}  \mathit{y}  \ottsym{)}  \ottsym{,}  \Gamma  \ottsym{(}  \mathit{y}  \ottsym{)}  \ottsym{)}  \ottsym{(}  \ottmv{a}  \ottsym{)}  \ottsym{=}  \ottsym{0}$:
  \begin{enumerate}
  \item $ \ottnt{H}  \ottsym{\{}  \ottmv{a}  \hookleftarrow  v'  \ottsym{\}} \vdash   v'  \Downarrow  n $
  \item If $ \mathcal{L}   \mid    \Gamma  \setminus  \mathit{x}    \vdash _{\wf}  \tau $, $\ottkw{own} \, \ottsym{(}  \ottnt{H}  \ottsym{,}  v  \ottsym{,}  \tau  \ottsym{)}  \ottsym{(}  \ottmv{a}  \ottsym{)}  \ottsym{=}  \ottsym{0}$, and $ \ottkw{SATv} ( \ottnt{H} , \ottnt{R} , v , \tau ) $
    then $ \ottkw{SATv} ( \ottnt{H}  \ottsym{\{}  \ottmv{a}  \hookleftarrow  \ottnt{v'}  \ottsym{\}} , \ottnt{R} , v , \tau ) $.
  \item If $ \mathcal{L}   \vdash _{\wf}   \Gamma  \setminus  \mathit{x}  $, and $ \ottkw{SATv} ( \ottnt{H} , \ottnt{R} , v , \Gamma  \ottsym{(}  \mathit{z}  \ottsym{)} ) $, then $ \ottkw{SATv} ( \ottnt{H}  \ottsym{\{}  \ottmv{a}  \hookleftarrow  v  \ottsym{\}} , \ottnt{R} , v , \Gamma  \ottsym{(}  \mathit{z}  \ottsym{)} ) $
  \end{enumerate}
\end{lemma}
\begin{proof}\leavevmode
  \begin{enumerate}
  \item From $ \ottnt{H} \vdash   v'  \Downarrow  n $ and $ \ottnt{H} \vdash   \ottnt{H}  \ottsym{(}  \ottmv{a}  \ottsym{)}  \Downarrow  n $, we must have that for
    any possible sequence $\oldvec{\star}$, $\ottsym{[}  \ottnt{H}  \ottsym{,}  v'  \ottsym{]}  \ottsym{(}  \oldvec{\star}  \ottsym{)} \, \neq \, \ottmv{a}$ (if we did, then
    we would have that $v$ reaches an integer along paths of different lengths, a clear
    contradiction). Then the value of $\ottmv{a}$ in $\ottnt{H}$ is irrelevant to the derivation
    of $ \ottnt{H} \vdash   v'  \Downarrow  n $, giving $ \ottnt{H}  \ottsym{\{}  \ottmv{a}  \hookleftarrow  v'  \ottsym{\}} \vdash   v'  \Downarrow  n $.
  \item By induction on the shape of $\tau$. In the base case where $\tau  \ottsym{=}   \set{  \nu  \COL \TINT \mid  \varphi } $, from $ \mathcal{L}   \mid    \Gamma  \setminus  \mathit{x}    \vdash _{\wf}  \tau $
    we have $ \mathcal{L}   \mid    \Gamma  \setminus  \mathit{x}    \vdash _{\wf}  \varphi $, where by from \Cref{lemma:framed-update-pred} we have $\ottsym{[}  \ottnt{H}  \ottsym{,}  \ottnt{R}  \ottsym{]} \, \ottsym{[}  n  \ottsym{/}  \nu  \ottsym{]}  \varphi$ is equivalent to
    $\ottsym{[}  \ottnt{H}  \ottsym{\{}  \ottmv{a}  \hookleftarrow  v'  \ottsym{\}}  \ottsym{,}  \ottnt{R}  \ottsym{]} \, \ottsym{[}  n  \ottsym{/}  \nu  \ottsym{]}  \varphi$ whereby the result holds by assumption.

    In the inductive step, we have $\tau  \ottsym{=}   \tau'  \TREF^{ r } $, and $v \, \ottsym{=} \, \ottmv{a'}$.
    Suppose $\ottmv{a} \, \ottsym{=} \, \ottmv{a'}$: from $\ottkw{own} \, \ottsym{(}  \ottnt{H}  \ottsym{,}  v  \ottsym{,}  \tau  \ottsym{)}  \ottsym{(}  \ottmv{a}  \ottsym{)}  \ottsym{=}  \ottsym{0}$ we must then have $r  \ottsym{=}  \ottsym{0}$,
    whereby $\tau'  \ottsym{=}  \top_{\ottmv{n}}$. From \Cref{lem:top-type-path-sat}, item 1 above,
    \JT{the lemma that any shape consistent values satisfy the top type}, we have
    $ \ottkw{SATv} ( \ottnt{H}  \ottsym{\{}  \ottmv{a}  \hookleftarrow  v'  \ottsym{\}} , \ottnt{R} , \ottmv{a} , \tau ) $.

    Otherwise $\ottmv{a} \, \neq \, \ottmv{a'}$ in which case the result holds by inversion on
    $ \mathcal{L}   \mid    \Gamma  \setminus  \mathit{x}    \vdash _{\wf}  \tau $, $\ottkw{own} \, \ottsym{(}  \ottnt{H}  \ottsym{,}  v  \ottsym{,}   \tau'  \TREF^{ r }   \ottsym{)}  \ottsym{(}  \ottmv{a}  \ottsym{)}  \ottsym{=}  \ottsym{0}$ and the inductive hypothesis.

  \item Immediate result of item 2.
  \end{enumerate}
\end{proof}

\begin{lemma}[Preservation]
  For any $\ottnt{e}$ where $ \Theta   \mid   \mathcal{L}   \mid   \Gamma_{{\mathrm{1}}}   \vdash   \ottnt{e}  :  \tau   \produces   \Gamma_{{\mathrm{2}}} $,
  for all $\ottnt{e'}$ and $\ottnt{E}$ such that
  $\Theta  \mid  \HOLE  \ottsym{:}  \tau  \produces  \Gamma_{{\mathrm{2}}}  \mid  \mathcal{L}  \vdash_{\mathit{ectx} }  \ottnt{E}  \ottsym{:}  \tau'  \produces  \Gamma_{{\mathrm{3}}}$ if
  $  \tuple{ \ottnt{H} ,  \ottnt{R} ,  \oldvec{F} ,  \ottnt{E}  \ottsym{[}  \ottnt{e}  \ottsym{]} }     \longrightarrow _{ \ottnt{D} }     \tuple{ \ottnt{H} ,  \ottnt{R} ,  \oldvec{F} ,  \ottnt{E}  \ottsym{[}  \ottnt{e'}  \ottsym{]} }  $, $ \mathcal{L}   \vdash _{\wf}  \Gamma_{\ottmv{p}} $ and
  $\ottkw{Cons} \, \ottsym{(}  \ottnt{H}  \ottsym{,}  \ottnt{R}  \ottsym{,}  \Gamma_{{\mathrm{1}}}  \ottsym{+}  \Gamma_{\ottmv{p}}  \ottsym{)}$ then there exists some $\Gamma_{{\mathrm{4}}}$ such that
  \begin{enumerate}
  \item $\ottkw{Cons} \, \ottsym{(}  \ottnt{H'}  \ottsym{,}  \ottnt{R'}  \ottsym{,}  \Gamma_{{\mathrm{4}}}  \ottsym{+}  \Gamma_{\ottmv{p}}  \ottsym{)}$
  \item $ \Theta   \mid   \mathcal{L}   \mid   \Gamma_{{\mathrm{4}}}   \vdash   \ottnt{e'}  :  \tau   \produces   \Gamma_{{\mathrm{2}}} $
  \end{enumerate}
\end{lemma}
\begin{proof}
  By induction on the derivation of $ \Theta   \mid   \mathcal{L}   \mid   \Gamma_{{\mathrm{1}}}   \vdash   \ottnt{e}  :  \tau   \produces   \Gamma_{{\mathrm{2}}} $.
  \begin{rncase}{T-Sub}
    By the inductive hypothesis, that $\ottkw{Own}$ is anti-monotone w.r.t the subtyping
    relation \admitted{}, and the preservation of $\ottkw{Cons}$ by subtyping
    \admitted{} (\Cref{lem:subtyping-preserves-cons}).
  \end{rncase}
  \begin{rncase}{T-Frame}
    Then we have that $ \Theta   \mid   \mathcal{L}   \mid   \Gamma'_{{\mathrm{1}}}   \vdash   \ottnt{e}  :  \tau   \produces   \Gamma'_{{\mathrm{2}}} $ where
    $\Gamma_{{\mathrm{1}}}  \ottsym{=}  \Gamma'_{{\mathrm{1}}}  \ottsym{+}  \Gamma''_{{\mathrm{1}}}$ and $\Gamma_{{\mathrm{2}}}  \ottsym{=}  \Gamma'_{{\mathrm{2}}}  \ottsym{+}  \Gamma'_{{\mathrm{1}}}$ and where $ \mathcal{L}   \vdash _{\wf}  \Gamma'_{{\mathrm{1}}} $.
    We have $\ottkw{Cons} \, \ottsym{(}  \ottnt{H}  \ottsym{,}  \ottnt{R}  \ottsym{,}  \Gamma'_{{\mathrm{1}}}  \ottsym{+}  \Gamma''_{{\mathrm{1}}}  \ottsym{+}  \Gamma_{\ottmv{p}}  \ottsym{)}$. We must then have that $ \mathcal{L}   \vdash _{\wf}  \Gamma'_{{\mathrm{1}}}  \ottsym{+}  \Gamma_{\ottmv{p}} $
    by \admitted[WF is closed under +].
    Taking $\Gamma_{\ottmv{p}}$ in the inductive hypothesis to be $\Gamma_{\ottmv{p}}  \ottsym{+}  \Gamma'_{{\mathrm{1}}}$, we
    then have that $\ottkw{Cons} \, \ottsym{(}  \ottnt{H'}  \ottsym{,}  \ottnt{R'}  \ottsym{,}  \Gamma'_{{\mathrm{4}}}  \ottsym{+}  \Gamma_{\ottmv{p}}  \ottsym{+}  \Gamma'_{{\mathrm{1}}}  \ottsym{)}$ and
    $ \Theta   \mid   \mathcal{L}   \mid   \Gamma'_{{\mathrm{4}}}   \vdash   \ottnt{e'}  :  \tau   \produces   \Gamma'_{{\mathrm{2}}} $. We take $\Gamma_{{\mathrm{4}}}  \ottsym{=}  \Gamma'_{{\mathrm{4}}}  \ottsym{+}  \Gamma'_{{\mathrm{1}}}$. Then, by an
    application of \rn{T-Frame} we have $ \Theta   \mid   \mathcal{L}   \mid   \Gamma_{{\mathrm{4}}}   \vdash   \ottnt{e'}  :  \tau   \produces   \Gamma_{{\mathrm{2}}} $, and
    we then have $\ottkw{Cons} \, \ottsym{(}  \ottnt{H'}  \ottsym{,}  \ottnt{R'}  \ottsym{,}  \Gamma_{{\mathrm{4}}}  \ottsym{+}  \Gamma_{\ottmv{p}}  \ottsym{)}$ from $\ottkw{Cons} \, \ottsym{(}  \ottnt{H'}  \ottsym{,}  \ottnt{R'}  \ottsym{,}  \Gamma'_{{\mathrm{4}}}  \ottsym{+}  \Gamma'_{{\mathrm{1}}}  \ottsym{+}  \Gamma_{\ottmv{p}}  \ottsym{)}$.
  \end{rncase}

  \begin{rneqncase}{T-Assign}{
      \ottnt{e}  \ottsym{=}   \mathit{y}  \WRITE  \mathit{x}  \SEQ  \ottnt{e''}  & \Gamma_{{\mathrm{1}}}  \ottsym{(}  \mathit{x}  \ottsym{)}  \ottsym{=}  \tau_{{\mathrm{1}}}  \ottsym{+}  \tau_{{\mathrm{2}}} & \Gamma_{{\mathrm{1}}}  \ottsym{(}  \mathit{y}  \ottsym{)}  \ottsym{=}   \tau'  \TREF^{ \ottsym{1} }  \\
       | \tau' |  \, \ottsym{=} \,  | \tau_{{\mathrm{1}}}  \ottsym{+}  \tau_{{\mathrm{2}}} |  &  \mathcal{L}   \vdash _{\wf}   \Gamma_{{\mathrm{1}}}  \setminus  \mathit{y}   \\
      \multicolumn{4}{l}{ \Theta   \mid   \mathcal{L}   \mid   \Gamma  \ottsym{[}  \mathit{x}  \hookleftarrow  \tau_{{\mathrm{1}}}  \wedge \, \nu \, \ottsym{=}  \mathit{y} \,  \star   \ottsym{]}  \ottsym{[}  \mathit{y}  \hookleftarrow   \ottsym{(}  \tau_{{\mathrm{2}}}  \wedge \, \nu \, \ottsym{=}  \mathit{x}  \ottsym{)}  \TREF^{ \ottsym{1} }   \ottsym{]}   \vdash   \ottnt{e'}  :  \tau   \produces   \Gamma_{{\mathrm{2}}} } \\
      \ottmv{a} \, \ottsym{=} \, \ottnt{R}  \ottsym{(}  \mathit{y}  \ottsym{)} & \ottnt{H'}  \ottsym{=}  \ottnt{H}  \ottsym{\{}  \ottmv{a}  \hookleftarrow  \ottnt{R}  \ottsym{(}  \mathit{x}  \ottsym{)}  \ottsym{\}} & \ottnt{R}  \ottsym{=}  \ottnt{R'} & \ottnt{e''}  \ottsym{=}  \ottnt{e'}
    }
    From $\ottkw{Cons} \, \ottsym{(}  \ottnt{H}  \ottsym{,}  \ottnt{R}  \ottsym{,}  \Gamma_{{\mathrm{1}}}  \ottsym{+}  \Gamma_{\ottmv{p}}  \ottsym{)}$ and from $\Gamma_{{\mathrm{1}}}  \ottsym{(}  \mathit{y}  \ottsym{)}  \ottsym{=}   \tau'  \TREF^{ \ottsym{1} } $ we must have that
    for any variable $ \mathit{z}  \in   \DOM( \Gamma_{{\mathrm{1}}} )  , z \neq y$ that
    $\ottkw{own} \, \ottsym{(}  \ottnt{H}  \ottsym{,}  \ottnt{R}  \ottsym{(}  \mathit{z}  \ottsym{)}  \ottsym{,}  \Gamma_{{\mathrm{1}}}  \ottsym{(}  \mathit{z}  \ottsym{)}  \ottsym{)}  \ottsym{(}  \ottmv{a}  \ottsym{)}  \ottsym{=}  \ottsym{0}$ and similarly for all variables in $ \DOM( \Gamma_{\ottmv{p}} ) $.

    We must also have that if $ \mathit{y}  \in \DOM( \Gamma_{\ottmv{p}} ) $ that $\Gamma  \ottsym{(}  \mathit{y}  \ottsym{)}  \ottsym{=}   \top_{\ottmv{n}}  \TREF^{ \ottsym{0} } $. Then by
    \admitted[0 references are not referenced in types], we have $ \mathcal{L}   \vdash _{\wf}   \Gamma_{\ottmv{p}}  \setminus  \mathit{y}  $.

    Next, from \admitted[SATv implies shape consistency], from $ \ottkw{SATv} ( \ottnt{H} , \ottnt{R} , \ottnt{R}  \ottsym{(}  \mathit{y}  \ottsym{)} ,  \tau'  \TREF^{ \ottsym{1} }  ) $ we have that $ \ottnt{H} \vdash   \ottnt{R}  \ottsym{(}  \mathit{y}  \ottsym{)}  \Downarrow   |  \tau'  \TREF^{ \ottsym{1} }  |  $, whereby we have $ \ottnt{H} \vdash   \ottnt{H}  \ottsym{(}  \ottnt{R}  \ottsym{(}  \mathit{y}  \ottsym{)}  \ottsym{)}  \Downarrow   | \tau' |  $. Similarly, from $ \ottkw{SATv} ( \ottnt{H} , \ottnt{R} , \ottnt{R}  \ottsym{(}  \mathit{x}  \ottsym{)} , \tau_{{\mathrm{1}}}  \ottsym{+}  \tau_{{\mathrm{2}}} ) $ we have $ \ottnt{H} \vdash   \ottnt{R}  \ottsym{(}  \mathit{x}  \ottsym{)}  \Downarrow   | \tau_{{\mathrm{1}}}  \ottsym{+}  \tau_{{\mathrm{1}}} |  $.

    From the above, our assumption $ \mathcal{L}   \vdash _{\wf}   \Gamma_{{\mathrm{1}}}  \setminus  \mathit{y}  $ and  \Cref{lemma:framed-update-pred}
    we then have $ \ottkw{SATv} ( \ottnt{H'} , \ottnt{R} , \ottnt{R}  \ottsym{(}  \mathit{z}  \ottsym{)} , \Gamma_{{\mathrm{1}}}  \ottsym{(}  \mathit{z}  \ottsym{)} ) $ for any $\mathit{z} \, \neq \, \mathit{x}$ and
    $\mathit{z} \, \neq \, \mathit{y}$.

    Similarly, we must have that $ \ottkw{SATv} ( \ottnt{H'} , \ottnt{R} , \ottnt{R}  \ottsym{(}  \mathit{z}  \ottsym{)} , \Gamma_{\ottmv{p}}  \ottsym{(}  \mathit{z}  \ottsym{)} ) $ by the reasoning above
    and \Cref{lemma:framed-update-pred}.

    We must also have that $ \ottkw{SATv} ( \ottnt{H'} , \ottnt{R} , \ottnt{R}  \ottsym{(}  \mathit{x}  \ottsym{)} , \tau_{{\mathrm{1}}}  \ottsym{+}  \tau_{{\mathrm{2}}} ) $. From \admitted[SATv distributes
    over +], we therefore
    have $ \ottkw{SATv} ( \ottnt{H'} , \ottnt{R} , \ottnt{R}  \ottsym{(}  \mathit{x}  \ottsym{)} , \tau_{{\mathrm{1}}} ) $ and $ \ottkw{SATv} ( \ottnt{H'} , \ottnt{R} , \ottnt{R}  \ottsym{(}  \mathit{x}  \ottsym{)} , \tau_{{\mathrm{2}}} ) $. It is then immediate
    that $ \ottkw{SATv} ( \ottnt{H'} , \ottnt{R} , \ottnt{R}  \ottsym{(}  \mathit{x}  \ottsym{)} , \tau_{{\mathrm{1}}}  \wedge \, \nu \, \ottsym{=}  \mathit{y} \,  \star  ) $ and $ \ottkw{SATv} ( \ottnt{H'} , \ottnt{R} , \ottnt{R}  \ottsym{(}  \mathit{x}  \ottsym{)} , \tau_{{\mathrm{2}}}  \wedge \, \nu \, \ottsym{=}  \mathit{x} ) $.
    From the latter we then have $ \ottkw{SATv} ( \ottnt{H'} , \ottnt{R} , \ottnt{R}  \ottsym{(}  \mathit{y}  \ottsym{)} ,  \ottsym{(}  \tau_{{\mathrm{2}}}  \wedge \, \nu \, \ottsym{=}  \mathit{x}  \ottsym{)}  \TREF^{ \ottsym{1} }  ) $.

    \JT{The ownership reasoning is entirely similar to the ESOP paper. We can use that
      $\ottkw{Cons} \, \ottsym{(}  \ottnt{H}  \ottsym{,}  \ottnt{R}  \ottsym{,}  \Gamma_{{\mathrm{1}}}  \ottsym{+}  \Gamma_{\ottmv{p}}  \ottsym{)}$ and that ownership is only lost in $\Gamma_{{\mathrm{1}}}$ to re-establish
      ownership consistency for $\Gamma_{{\mathrm{4}}}  \ottsym{+}  \Gamma_{\ottmv{p}}$}
    
    We take $\Gamma_{{\mathrm{4}}}  \ottsym{=}  \Gamma  \ottsym{[}  \mathit{x}  \hookleftarrow  \tau_{{\mathrm{1}}}  \wedge \, \nu \, \ottsym{=}  \mathit{y} \,  \star   \ottsym{]}  \ottsym{[}  \mathit{y}  \hookleftarrow   \ottsym{(}  \tau_{{\mathrm{2}}}  \wedge \, \nu \, \ottsym{=}  \mathit{x}  \ottsym{)}  \TREF^{ \ottsym{1} }   \ottsym{]}$ whereby
    from \admitted[adding SATv envs is Cons]
    and the previous reasoning we have $\ottkw{Cons} \, \ottsym{(}  \ottnt{H'}  \ottsym{,}  \ottnt{R'}  \ottsym{,}  \Gamma_{{\mathrm{4}}}  \ottsym{+}  \Gamma_{\ottmv{p}}  \ottsym{)}$.
    That $ \Theta   \mid   \mathcal{L}   \mid   \Gamma_{{\mathrm{4}}}   \vdash   \ottnt{e'}  :  \tau   \produces   \Gamma_{{\mathrm{2}}} $ is immediate from our assumption.
  \end{rneqncase}
    
\end{proof}


\end{document}
%%% Local Variables:
%%% mode: latex
%%% TeX-master: t
%%% End:
