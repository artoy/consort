\section{Introduction}
\label{sec:intro}
Driven by the increasing power of automated theorem provers and
recent high-profile software failures,
fully automated program verification has seen a surge of interest
in recent years \cite{leino2010dafny,cousot2005astree,ball2011decade,zave2012using,hawblitzel2015ironfleet,bhargavan2017everest}.
In particular, \emph{refinement types}
\cite{flanagan2006hybrid,xi1999dependent,freeman1991refinement,bengtson2011refinement},
which refine base types with logical predicates, have been shown to be a practical
approach for program verification that are amenable to
(sometimes full) automation \cite{rondon2008liquid,vazou2014refinement,vazou2013abstract,unno2009dependent}.
Despite promising advances \cite{rondon2010low,kahsai2017quantified,gordon2013rely},
the sound and precise application of refinement types (and program
verification in general) in settings with mutability and aliasing
(e.g., Java, Ruby, etc.) remains difficult.


\begin{figure}[t]
  \begin{minipage}[t]{0.45\textwidth}
  \begin{lstlisting}[numbers=left, numbersep=3pt, numberstyle=\tiny\color{black},numberblanklines=false]
mk(n) { mkref n }$\label{line:mk}$

let p = mk(3) in$\label{line:main-start}$
let q = mk(5) in
p := *p + 1;$\label{line:first-p-write}$
q := *q + 1;$\label{line:r-write}$
assert(*p = 4);$\label{line:rq-eq-assert}$
  \end{lstlisting}
  \caption{Example demonstrating the difficulty of effecting strong updates in the presence of aliasing. The function \imp{mk} is
    bound in the program from \crefrange{line:main-start}{line:rq-eq-assert}; its body is given within the braces.}
  \label{fig:strong-update-example}
\end{minipage}
\hfill
\begin{minipage}[t]{0.45\textwidth}
  \begin{lstlisting}[numbers=left, numbersep=3pt, numberstyle=\tiny\color{black},numberblanklines=false]
loop(a, b) {
  let aold = *a in
  b := *b + 1;$\label{line:b-write}$
  a := *a + 1;$\label{line:a-write}$
  assert(*a = aold + 1);
  if $\star$ then
    loop(b, mkref $\star$)$\label{line:alloc1}$
  else
    loop(b,a)$\label{line:call2}$
}
loop(mkref $\star$, mkref $\star$)$\label{line:start}$
\end{lstlisting}
  \caption{Example with non-trivial aliasing behavior.}
  \label{fig:hard-loop}
\end{minipage}
\end{figure}

One of the major challenges is how to precisely and soundly support
\emph{strong updates} for the invariants on memory cells.
In a setting with mutability, a single invariant may not necessarily
hold throughout the lifetime of a memory cell; while the program
mutates the memory the invariant may change or evolve.
To model these changes, a program verifier
must support different, incompatible invariants which hold at
different points during program execution. Further, precise
program verification requires supporting different invariants
on distinct pieces of memory.

One solution is to use refinement types on the static program names
(i.e., variables) which point to a memory location.
This approach can model evolving invariants
while tracking distinct invariants for each memory cell.
For example,
consider the (contrived) example in \Cref{fig:strong-update-example}.
This program is written in an ML-like language with mutable references;
references are updated with \imp{:=} and allocated with \imp{mkref}.
Variable \imp{p} can initially be given the type
$ \set{  \nu  \COL \TINT \mid  \nu \, \ottsym{=} \, \ottsym{3} } \TREF$, indicating it is a
reference to the integer 3.
Similarly, \imp{q} can be given the type $ \set{  \nu  \COL \TINT \mid  \nu \, \ottsym{=} \, \ottsym{5} }  \TREF$.
We can model the mutation of \imp{p}'s memory on \cref{line:first-p-write}
by strongly updating \imp{p}'s type to $ \set{  \nu  \COL \TINT \mid  \nu \, \ottsym{=} \, \ottsym{4} }  \TREF$.

Unfortunately, the precise application of this technique is confounded
by the existence of unrestricted aliasing.
In general, updating just the type of the mutated reference
is insufficient: due to aliasing, other variables may point to the mutated
memory and their refinements must be updated as well. However,
in the presence of conditional, \emph{may} aliasing, it is
impossible to strongly update the refinements on all possible
aliases; given the static uncertainty
about whether a variable points to the mutated memory, that variable's
refinement may only be \emph{weakly updated}. For example,
suppose we used a simple alias analysis that imprecisely (but soundly)
concluded all references allocated at the same program point \emph{might} alias.
Variables \imp{p} and \imp{q} share the allocation site on
\cref{line:mk}, so on \cref{line:first-p-write} we would have to weakly update
\imp{q}'s type to $ \set{  \nu  \COL \TINT \mid  \nu \, \ottsym{=} \, \ottsym{4}  \vee  \nu \, \ottsym{=} \, \ottsym{5} } $,
indicating it may hold either 4 \emph{or} 5.
Under this same imprecise aliasing assumption,
we would also have to weakly update
\imp{p}'s type on \cref{line:r-write},
preventing the verification of the example program.

Given the precision loss associated with weak updates,
it is critical that verification techniques built upon
refinement types use precise aliasing information and
avoid spuriously applied weak updates. Although it
is relatively simple to conclude that \imp{p} and \imp{q}
do not alias in \Cref{fig:strong-update-example}, consider
the example in \Cref{fig:hard-loop}. (In this example,
$\star$ represents non-deterministic values.) Verifying
this program requires proving \imp{a} and \imp{b} never
alias at the writes on \cref{line:b-write,line:a-write}.
In fact, \imp{a} and \imp{b} \emph{may} point to the same
memory location, but only in different invocations of \imp{loop};
this pattern may confound even sophisticated symbolic alias analyses.
Additionally, \imp{a} and \imp{b} share an allocation site on \cref{line:alloc1},
so an approach based on the simple alias analysis described above will also fail on
this example. This must-not alias proof obligation  \emph{can} be discharged with existing
techniques \cite{spath2016boomerang,spath2019context}, but requires an expensive, on-demand,
interprocedural, flow-sensitive alias analysis. 

This paper presents \name (CONtext Sensitive Ownership Refinement Types),
a type system for the automated verification
of program safety in imperative languages with mutability and aliasing.
\name is built upon the novel combination of refinement types and
fractional ownership types \cite{suenaga2009fractional,suenaga2012type}.
Fractional ownership types extend pointer types with a rational number in
the range $[0,1]$ called an \emph{ownership}.
These ownerships encapsulate the permission of the reference; only
references with ownership $1$ may be used for mutation. Fractional ownership
types also obey the following key invariant: any references with a mutable
alias must have ownership 0.
Thus, any reference with non-zero ownership \emph{cannot}
be an alias of a reference with ownership $1$. In other words,
ownerships encode precise aliasing information in the form of
\emph{must-not} aliasing relationships.\looseness=-1

To understand the benefit of this approach, let us return to
\Cref{fig:strong-update-example}. As \imp{mk} returns a freshly allocated
reference with no aliases, its type indicates it returns a
reference with ownership 1. Thus, our type system can initially
give \imp{p} and \imp{q} types $  \set{  \nu  \COL \TINT \mid  \nu \, \ottsym{=} \, \ottsym{3} }   \TREF^{ \ottsym{1} } $ and $  \set{  \nu  \COL \TINT \mid  \nu \, \ottsym{=} \, \ottsym{5} }   \TREF^{ \ottsym{1} } $ respectively. The ownership $\ottsym{1}$ on the reference type constructor $\TREF$ indicates both
pointers hold ``exclusive'' ownership of the pointed to reference cell;
from the invariant of fractional ownership types \imp{p} and \imp{q}
must \emph{not} alias. The types of both references
can be strongly updated \emph{without} requiring spurious weak updates. As a result,
at the assertion statement on \cref{line:rq-eq-assert}, \imp{p} has type
$  \set{  \nu  \COL \TINT \mid  \nu \, \ottsym{=} \, \ottsym{4} }   \TREF^{ \ottsym{1} } $ expressing the required invariant.\looseness=-1

Our type system can also verify the example in
\Cref{fig:hard-loop} \emph{without} expensive
side analyses. As \imp{a} and \imp{b} are both
mutated, they must both have ownership 1; i.e., they cannot alias.
This pre-condition is satisfied by all invocations of \imp{loop};
on \cref{line:alloc1}, \imp{b} has
ownership 1 (from the argument type), and the newly allocated
reference must also have ownership 1.
Similarly, both arguments on \cref{line:call2} have ownership $\ottsym{1}$
(from the assumed ownership on the argument types). % Finally,
% the two freshly allocated references on \cref{line:start} trivially have
% ownership 1.

Ownerships behave linearly;
they cannot be duplicated, only \emph{split}
when aliases are created. This linear behavior
preserves the critical ownership invariant. For example, if
we replace \cref{line:call2} in \Cref{fig:hard-loop} with
\imp{loop(b,b)}, the program becomes ill-typed; there
is no way to divide \imp{b}'s ownership of 1 to into \emph{two}
ownerships of 1.

Ownerships also obviate updating refinement information
of aliases at mutation. \name ensures that only the trivial refinement
$ \top $ is used in reference types with ownership $0$, i.e., mutably-aliased references.
When memory is mutated through a reference with ownership $1$,
\name simply updates the refinement of the mutated reference variable.
From the soundness of ownership types, all aliases have ownership 0
and must therefore only contain the $ \top $ refinement. Thus, the types of all aliases already
soundly describe \emph{all} possible contents.\footnote{This
  assumption holds only if updates do not change simple types, a condition our type-system enforces.}

\name is also \emph{context-sensitive}, and can
use different summaries of function behavior
at different points in the program.
For example, consider the variant
\begin{wrapfigure}{l}{0.29\textwidth}
  \begin{lstlisting}[numbers=left,numbersep=3pt,numberstyle=\tiny\color{black},numberblanklines=false]
get(p) { *p }

let p = mkref 3 in
let q = mkref 5 in
p := get(p) + 1;$\label{line:get1}$
q := get(q) + 1;$\label{line:get2}$
assert(*p = 4);
assert(*q = 6);
  \end{lstlisting}
  \caption{Example of context-sensitivity}
  \label{fig:running-example}
\end{wrapfigure}
of \Cref{fig:strong-update-example}
shown in \Cref{fig:running-example}.
The function \imp{get} returns the contents of its argument,
and is called on \cref{line:get1,line:get2}. To precisely verify
this program, on \cref{line:get1}
\imp{get} must be typed as a function that takes a reference to
3 and returns 3. Similarly, on \cref{line:get2}
\imp{get} must be typed as a function that
takes a reference to 5 and returns 5. Our type system can
give \imp{get} a function type that distinguishes between these two calling
contexts and selects the appropriate summary of \imp{get}'s behavior.

We have formalized \name as a type system for a small imperative calculus
and proved the system is sound: i.e., a well-typed program never encounters assertion
failures during execution. We have
implemented a prototype type inference tool targeting this
imperative language and found it can automatically verify several
non-trivial programs, including sorted lists and an array list data structure.

The rest of this paper is organized as follows. \Cref{sec:prelim} defines the
imperative language targeted by \name and its semantics. \Cref{sec:typesystem}
defines our type system and states our soundness theorem. \Cref{sec:infr} sketches
our implementation's inference algorithm and its current limitations.
\Cref{sec:eval} describes an evaluation of our prototype, \Cref{sec:rw} outlines
related work, and \Cref{sec:concl} concludes.\looseness=-1

%%% Local Variables:
%%% mode: latex
%%% TeX-master: t
%%% End:

%  LocalWords:  provers verifier mkref ownerships
