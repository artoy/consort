\section{Target Language}
\label{sec:prelim}
This section describes a simple imperative language with mutable references and first-order, recursive functions.

% \subsection{Notation}
% For a (meta-level, set-theoretic) function $f$ we will write $\DOM(f)$ to indicate the domain of
% the function and write
% $f \{ x \mapsto v \}$ where $x \not\in\DOM(f)$ to denote a map
% which takes all values in $\DOM(f)$ to their values in $f$ and which
% additionally takes $x$ to $v$. We will write $ f \{ x \hookleftarrow v \}$
% where $x\in\DOM(f)$ to denote a map equivalent to $f$ except that $x$ takes value $v$.
% We also write $\emptyset$ as a function with an empty domain.
% \JT{This is almost certainly not a sufficient overview. write more later.}

\subsection{Syntax}
\label{sec:language}
We assume a set of \emph{variables}, ranged over by $x,y,z,\dots$,
a set of \emph{function names}, ranged over by $f$, and
a set of \emph{labels}, ranged over by $\ell_{{\mathrm{1}}}, \ell_{{\mathrm{2}}}, \dots$.
The grammar of the language is as follows.
\[
  \begin{array}{rcl}
  \ottnt{d} &::=& \mathit{f}  \mapsto  \ottsym{(}  \mathit{x_{{\mathrm{1}}}}  \ottsym{,} \, ... \, \ottsym{,}  \mathit{x_{\ottmv{n}}}  \ottsym{)}  \ottnt{e} \\
  \ottnt{e} &::= &
               \mathit{x} \mid
                \LET  \mathit{x}  =  \mathit{y}  \IN  \ottnt{e}  \mid
                \LET  \mathit{x}  =  n  \IN  \ottnt{e}  \mid
                \IFZERO  \mathit{x}  \THEN  \ottnt{e_{{\mathrm{1}}}}  \ELSE  \ottnt{e_{{\mathrm{2}}}}  \\
        &\mid&  \LET  \mathit{x}  =   \MKREF  \mathit{y}   \IN  \ottnt{e}  \mid
                \LET  \mathit{x}  =   *  \mathit{y}   \IN  \ottnt{e}  \mid
                \LET  \mathit{x}  =   \mathit{f} ^ \ell (  \mathit{y_{{\mathrm{1}}}} ,\ldots, \mathit{y_{\ottmv{n}}}  )   \IN  \ottnt{e}  \\
        &\mid&  \mathit{x}  \WRITE  \mathit{y}  \SEQ  \ottnt{e}  \mid
                \ALIAS( \mathit{x}  =  \mathit{y} ) \SEQ  \ottnt{e}  \mid
                \ALIAS( \mathit{x}  = *  \mathit{y} ) \SEQ  \ottnt{e}  \mid
                \ASSERT( \varphi ) \SEQ  \ottnt{e}  \mid
                \ottnt{e_{{\mathrm{1}}}}  \SEQ  \ottnt{e_{{\mathrm{2}}}}  \\
%    \ottnt{v} &::=& n \mid \mathit{x} \\
    \ottnt{P} &::=&  \tuple{ \ottsym{\{}  \ottnt{d_{{\mathrm{1}}}}  \ottsym{,} \, ... \, \ottsym{,}  \ottnt{d_{\ottmv{n}}}  \ottsym{\}} ,  \ottnt{e} }  %\\
%    \ell &\in& \textbf{Labels}\\
  \end{array}
\]
$\varphi$ stands for a formula in propositional first-order
logic over variables, integers and contexts;
we discuss these formulas later in \Cref{sec:types}.

Variables are introduced by function parameters or let bindings.
Like ML, the variable bindings introduced by let expressions and parameters
are immutable.
Mutable variable declarations such as \texttt{int x = 1;} in C are achieved in our language with:\looseness=-1
\[
   \LET  \mathit{y}  =  \ottsym{1}  \IN  \ottsym{(}   \LET  \mathit{x}  =   \MKREF  \mathit{y}   \IN   \ldots    \ottsym{)} \ .
\]
As a convenience, we assume all variable names
introduced with let bindings and function parameters are distinct.

Unlike ML (and like C or Java) we do not allow general expressions on the right hand side of
let bindings. The simplest right hand forms are a variable $\mathit{y}$ or an integer literal $n$.
$ \MKREF  \mathit{y} $ creates a reference cell with value $\mathit{y}$, and
$ *  \mathit{y} $ accesses the contents of reference $\mathit{y}$. For simplicity,
we do not include an explicit null value;
an extension to support null is discussed in \Cref{sec:infr}.
Function calls must occur on the right hand side of a variable binding
and take the form $ \mathit{f} ^ \ell (  \mathit{x_{{\mathrm{1}}}} ,\ldots, \mathit{x_{\ottmv{n}}}  ) $, where $ \mathit{x_{{\mathrm{1}}}} ,\ldots, \mathit{x_{\ottmv{n}}} $ are distinct variables and $\ell$ is a
(unique) label. % drawn from \textbf{Labels}.
These labels are used to make our type system context-sensitive as
discussed in \Cref{sec:cs}.

The single base case for expressions is a single variable.
If the variable expression is executed in a tail position of a function,
then the value of that variable is the return value of the function,
otherwise the value is ignored.

The only intraprocedural control-flow operations in our language are if statements.
$\ottkw{ifz}$ checks whether the condition variable $\mathit{x}$ equals zero and chooses
the corresponding branch. Loops can be implemented with recursive functions and
we do not include them explicitly in our formalism.

Our grammar requires that side-effecting, result-free statements, \lstinline{assert}($\varphi$)
\lstinline{alias}($\mathit{x}$ = $\mathit{y}$), \lstinline{alias}($\mathit{x} = *\mathit{y}$) and
assignment $\mathit{x} := \mathit{y}$ are followed by a continuation expression.
We impose this requirement for technical reasons to ease our formal
presentation; this requirement does not reduce expressiveness
as dummy continuations can be inserted as needed.
The $ \ASSERT( \varphi ) \SEQ  \ottnt{e} $ form executes $\ottnt{e}$ if the predicate
$\varphi$ holds in the current state and aborts the program otherwise.
$ \ALIAS( \mathit{x}  =  \mathit{y} ) \SEQ  \ottnt{e} $ and $ \ALIAS( \mathit{x}  = *  \mathit{y} ) \SEQ  \ottnt{e} $ assert a must-aliasing relationship between $x$ and $y$ (resp. $\mathit{x}$ and $ *  \mathit{y} $)
and then execute $\ottnt{e}$.
\imp{alias} statements
are effectively \emph{annotations} that our type system exploits to gain
added precision.
$ \mathit{x}  \WRITE  \mathit{y}  \SEQ  \ottnt{e} $ updates the contents of the memory cell pointed to by $\mathit{x}$
with the value of $\mathit{y}$.
In addition to the above continuations, our language supports general sequencing with
$ \ottnt{e_{{\mathrm{1}}}}  \SEQ  \ottnt{e_{{\mathrm{2}}}} $.

A program is a pair $ \tuple{ \ottnt{D} ,  \ottnt{e} } $, where
$\ottnt{D}  \ottsym{=}  \ottsym{\{}  \ottnt{d_{{\mathrm{1}}}}  \ottsym{,} \, ... \, \ottsym{,}  \ottnt{d_{\ottmv{n}}}  \ottsym{\}}$ is a set of first-order, mutually recursive function definitions, and $\ottnt{e}$
is the program entry point. A function definition $\ottnt{d}$
maps the function name to a tuple of argument names $\mathit{x_{{\mathrm{1}}}}  \ottsym{,} \, ... \, \ottsym{,}  \mathit{x_{\ottmv{n}}}$
that are
bound within the function body $\ottnt{e}$.

\paragraph{Paper Syntax.}
In the remainder of the paper, we will write programs that
are technically illegal according to our grammar, but can be
easily ``de-sugared'' into an equivalent, valid program.
For example, we will write
\begin{lstlisting}
  let x = mkref 4 in assert(*x = 4)
\end{lstlisting}
as syntactic sugar for:
\begin{lstlisting}
  let f = 4 in let x = mkref f in
  let tmp = *x in assert(tmp = 4); let dummy = 0 in dummy
\end{lstlisting}
%We have used a dummy continuation for the assert statement in the desugaring.

% \begin{figure}[t]
% \[
%   \begin{array}{rcl}
%     \ottnt{E} & ::= &  \ottnt{E'} \SEQ \ottnt{e}  \mid  \HOLE  \\
%     F & ::= & \ottnt{E}  \ottsym{[}   \LET  \mathit{y}  =   \HOLE^ \ell   \IN  \ottnt{e}   \ottsym{]}
%   \end{array}
% \]
% \caption{Grammar of evaluation and return contexts}
% \label{fig:eval-context-grammar}
% \end{figure}

\bcprulessavespacetrue
%edit by shiki
\begin{figure}[t]
  \scriptsize
  \begin{center}
  \infrule[R-Var]{
  }{
      \tuple{ \ottnt{H} ,  \ottnt{R} ,  F  \ottsym{:}  \oldvec{F} ,  \mathit{x} }     \longrightarrow _{ \ottnt{D} }     \tuple{ \ottnt{H} ,  \ottnt{R} ,  \oldvec{F} ,  F  \ottsym{[}  \mathit{x}  \ottsym{]} }  
   }
   \infrule[R-Seq]{
   }{
       \tuple{ \ottnt{H} ,  \ottnt{R} ,  F  \ottsym{:}  \oldvec{F} ,  \ottnt{E}  \ottsym{[}   \mathit{x}  \SEQ  \ottnt{e}   \ottsym{]} }     \longrightarrow _{ \ottnt{D} }     \tuple{ \ottnt{H} ,  \ottnt{R} ,  \oldvec{F} ,  \ottnt{E}  \ottsym{[}  \ottnt{e}  \ottsym{]} }  
   }
  \infrule[R-Let]{
     \mathit{x'}  \not\in \DOM( \ottnt{R} ) 
  }{
     \begin{array}{l}  \tuple{ \ottnt{H} ,  \ottnt{R} ,  \oldvec{F} ,  \ottnt{E}  \ottsym{[}   \LET  \mathit{x}  =  \mathit{y}  \IN  \ottnt{e}   \ottsym{]} }   \\ \quad   \longrightarrow _{ \ottnt{D} }     \tuple{ \ottnt{H} ,  \ottnt{R}  \ottsym{\{}  \mathit{x'}  \mapsto  \ottnt{R}  \ottsym{(}  \mathit{y}  \ottsym{)}  \ottsym{\}} ,  \oldvec{F} ,  \ottnt{E}  \ottsym{[}    [  \mathit{x'}  /  \mathit{x}  ]    \ottnt{e}   \ottsym{]} }  \end{array} 
  }
  \infrule[R-LetInt]{
     \mathit{x'}  \not\in \DOM( \ottnt{R} ) 
  }{
     \begin{array}{l}  \tuple{ \ottnt{H} ,  \ottnt{R} ,  \oldvec{F} ,  \ottnt{E}  \ottsym{[}   \LET  \mathit{x}  =  n  \IN  \ottnt{e}   \ottsym{]} }   \\ \quad   \longrightarrow _{ \ottnt{D} }     \tuple{ \ottnt{H} ,  \ottnt{R}  \ottsym{\{}  \mathit{x'}  \mapsto  n  \ottsym{\}} ,  \oldvec{F} ,  \ottnt{E}  \ottsym{[}    [  \mathit{x'}  /  \mathit{x}  ]    \ottnt{e}   \ottsym{]} }  \end{array} 
  }
  \infrule[R-IfTrue]{
    \ottnt{R}  \ottsym{(}  \mathit{x}  \ottsym{)} \, \ottsym{=} \, \ottsym{0}
  }{
     \begin{array}{l}  \tuple{ \ottnt{H} ,  \ottnt{R} ,  \oldvec{F} ,  \ottnt{E}  \ottsym{[}   \IFZERO  \mathit{x}  \THEN  \ottnt{e_{{\mathrm{1}}}}  \ELSE  \ottnt{e_{{\mathrm{2}}}}   \ottsym{]} }   \\ \quad   \longrightarrow _{ \ottnt{D} }     \tuple{ \ottnt{H} ,  \ottnt{R} ,  \oldvec{F} ,  \ottnt{E}  \ottsym{[}  \ottnt{e_{{\mathrm{1}}}}  \ottsym{]} }  \end{array} 
  }
  \infrule[R-IfFalse]{
    \ottnt{R}  \ottsym{(}  \mathit{x}  \ottsym{)} \, \neq \, \ottsym{0}
  }{
     \begin{array}{l}  \tuple{ \ottnt{H} ,  \ottnt{R} ,  \oldvec{F} ,  \ottnt{E}  \ottsym{[}   \IFZERO  \mathit{x}  \THEN  \ottnt{e_{{\mathrm{1}}}}  \ELSE  \ottnt{e_{{\mathrm{2}}}}   \ottsym{]} }   \\ \quad   \longrightarrow _{ \ottnt{D} }     \tuple{ \ottnt{H} ,  \ottnt{R} ,  \oldvec{F} ,  \ottnt{E}  \ottsym{[}  \ottnt{e_{{\mathrm{2}}}}  \ottsym{]} }  \end{array} 
  }
  \infrule[R-MkRef]{
     \ottmv{a}  \not\in \DOM( \ottnt{H} )  \andalso  \mathit{x'}  \not\in \DOM( \ottnt{R} ) 
  }{
     \begin{array}{r}  \tuple{ \ottnt{H} ,  \ottnt{R} ,  \oldvec{F} ,  \ottnt{E}  \ottsym{[}   \LET  \mathit{x}  =   \MKREF  \mathit{y}   \IN  \ottnt{e}   \ottsym{]} }     \longrightarrow _{ \ottnt{D} }   \\   \tuple{ \ottnt{H}  \ottsym{\{}  \ottmv{a}  \mapsto  \ottnt{R}  \ottsym{(}  \mathit{y}  \ottsym{)}  \ottsym{\}} ,  \ottnt{R}  \ottsym{\{}  \mathit{x'}  \mapsto  \ottmv{a}  \ottsym{\}} ,  \oldvec{F} ,  \ottnt{E}  \ottsym{[}    [  \mathit{x'}  /  \mathit{x}  ]    \ottnt{e}   \ottsym{]} }  \end{array} 
  }
  \infrule[R-Deref]{
    \ottnt{R}  \ottsym{(}  \mathit{y}  \ottsym{)} \, \ottsym{=} \, \ottmv{a} \andalso \ottnt{H}  \ottsym{(}  \ottmv{a}  \ottsym{)} \, \ottsym{=} \, \ottnt{v} \andalso  \mathit{x'}  \not\in \DOM( \ottnt{R} ) 
  }{
     \begin{array}{r}  \tuple{ \ottnt{H} ,  \ottnt{R} ,  \oldvec{F} ,  \ottnt{E}  \ottsym{[}   \LET  \mathit{x}  =   *  \mathit{y}   \IN  \ottnt{e}   \ottsym{]} }     \longrightarrow _{ \ottnt{D} }   \\   \tuple{ \ottnt{H} ,  \ottnt{R}  \ottsym{\{}  \mathit{x'}  \mapsto  \ottnt{v}  \ottsym{\}} ,  \oldvec{F} ,  \ottnt{E}  \ottsym{[}    [  \mathit{x'}  /  \mathit{x}  ]    \ottnt{e}   \ottsym{]} }  \end{array} 
  }
\end{center}
\caption{Transition Rules (1).}
\label{fig:transitionRules1}
\end{figure}
\begin{figure}[t]
  \scriptsize
  \begin{center}
  \infrule[R-Call]{
     \mathit{f}  \mapsto  \ottsym{(}  \mathit{x_{{\mathrm{1}}}}  \ottsym{,} \, .. \, \ottsym{,}  \mathit{x_{\ottmv{n}}}  \ottsym{)}  \ottnt{e}  \in  \ottnt{D} 
  }{
     \begin{array}{l}  \tuple{ \ottnt{H} ,  \ottnt{R} ,  \oldvec{F} ,  \ottnt{E}  \ottsym{[}   \LET  \mathit{x}  =   \mathit{f} ^ \ell (  \mathit{y_{{\mathrm{1}}}} ,\ldots, \mathit{y_{\ottmv{n}}}  )   \IN  \ottnt{e'}   \ottsym{]} }   \\ \quad   \longrightarrow _{ \ottnt{D} }     \tuple{ \ottnt{H} ,  \ottnt{R} ,   \ottnt{E} [\LET  \mathit{x}  =   \HOLE^ \ell   \IN  \ottnt{e'}  ]   \ottsym{:}  \oldvec{F} ,     [  \mathit{y_{{\mathrm{1}}}}  /  \mathit{x_{{\mathrm{1}}}}  ]  \cdots  [  \mathit{y_{\ottmv{n}}}  /  \mathit{x_{\ottmv{n}}}  ]     \ottnt{e}  }  \end{array} 
  }
  \infrule[R-Assign]{
    \ottnt{R}  \ottsym{(}  \mathit{x}  \ottsym{)} \, \ottsym{=} \, \ottmv{a} \andalso  \ottmv{a}  \in \DOM( \ottnt{H} ) 
  }{
     \begin{array}{r}  \tuple{ \ottnt{H} ,  \ottnt{R} ,  \oldvec{F} ,  \ottnt{E}  \ottsym{[}   \mathit{x}  \WRITE  \mathit{y}  \SEQ  \ottnt{e}   \ottsym{]} }     \longrightarrow _{ \ottnt{D} }   \\   \tuple{ \ottnt{H}  \ottsym{\{}  \ottmv{a}  \hookleftarrow  \ottnt{R}  \ottsym{(}  \mathit{y}  \ottsym{)}  \ottsym{\}} ,  \ottnt{R} ,  \oldvec{F} ,  \ottnt{E}  \ottsym{[}  \ottnt{e}  \ottsym{]} }  \end{array} 
  }
  \infrule[R-Alias]{
    \ottnt{R}  \ottsym{(}  \mathit{x}  \ottsym{)} \, \ottsym{=} \, \ottnt{R}  \ottsym{(}  \mathit{y}  \ottsym{)}
  }{
     \begin{array}{l}  \tuple{ \ottnt{H} ,  \ottnt{R} ,  \oldvec{F} ,  \ottnt{E}  \ottsym{[}   \ALIAS( \mathit{x}  =  \mathit{y} ) \SEQ  \ottnt{e}   \ottsym{]} }   \\ \quad   \longrightarrow _{ \ottnt{D} }     \tuple{ \ottnt{H} ,  \ottnt{R} ,  \oldvec{F} ,  \ottnt{E}  \ottsym{[}  \ottnt{e}  \ottsym{]} }  \end{array} 
  }
  \infrule[R-AliasPtr]{
    \ottnt{R}  \ottsym{(}  \mathit{y}  \ottsym{)} \, \ottsym{=} \, \ottmv{a} \andalso \ottnt{H}  \ottsym{(}  \ottmv{a}  \ottsym{)} \, \ottsym{=} \, \ottnt{R}  \ottsym{(}  \mathit{x}  \ottsym{)}
  }{
      \tuple{ \ottnt{H} ,  \ottnt{R} ,  \oldvec{F} ,  \ottnt{E}  \ottsym{[}   \ALIAS( \mathit{x}  = *  \mathit{y} ) \SEQ  \ottnt{e}   \ottsym{]} }     \longrightarrow _{ \ottnt{D} }     \tuple{ \ottnt{H} ,  \ottnt{R} ,  \oldvec{F} ,  \ottnt{E}  \ottsym{[}  \ottnt{e}  \ottsym{]} }  
  }
  \infrule[R-AliasFail]{
    \ottnt{R}  \ottsym{(}  \mathit{x}  \ottsym{)} \, \neq \, \ottnt{R}  \ottsym{(}  \mathit{y}  \ottsym{)}
  }{
      \tuple{ \ottnt{H} ,  \ottnt{R} ,  \oldvec{F} ,  \ottnt{E}  \ottsym{[}   \ALIAS( \mathit{x}  =  \mathit{y} ) \SEQ  \ottnt{e}   \ottsym{]} }     \longrightarrow _{ \ottnt{D} }     \mathbf{AliasFail}  
  }
  \infrule[R-AliasPtrFail]{
    \ottnt{R}  \ottsym{(}  \mathit{x}  \ottsym{)} \, \neq \, \ottnt{H}  \ottsym{(}  \ottnt{R}  \ottsym{(}  \mathit{y}  \ottsym{)}  \ottsym{)}
  }{
      \tuple{ \ottnt{H} ,  \ottnt{R} ,  \oldvec{F} ,  \ottnt{E}  \ottsym{[}   \ALIAS( \mathit{x}  = *  \mathit{y} ) \SEQ  \ottnt{e}   \ottsym{]} }     \longrightarrow _{ \ottnt{D} }     \mathbf{AliasFail}  
  }
  \infrule[R-Assert]{
    \models  \ottsym{[}  \ottnt{R}  \ottsym{]} \, \varphi
  }{
     \begin{array}{l}  \tuple{ \ottnt{H} ,  \ottnt{R} ,  \oldvec{F} ,  \ottnt{E}  \ottsym{[}   \ASSERT( \varphi ) \SEQ  \ottnt{e}   \ottsym{]} }   \\ \quad   \longrightarrow _{ \ottnt{D} }     \tuple{ \ottnt{H} ,  \ottnt{R} ,  \oldvec{F} ,  \ottnt{E}  \ottsym{[}  \ottnt{e}  \ottsym{]} }  \end{array} 
  }
  \infrule[R-AssertFail]{
    \not\models  \ottsym{[}  \ottnt{R}  \ottsym{]} \, \varphi
  }{
      \tuple{ \ottnt{H} ,  \ottnt{R} ,  \oldvec{F} ,  \ottnt{E}  \ottsym{[}   \ASSERT( \varphi ) \SEQ  \ottnt{e}   \ottsym{]} }     \longrightarrow _{ \ottnt{D} }     \mathbf{AssertFail}  
  }
  \end{center}
\caption{Transition Rules (2).}
\label{fig:transitionRules2}
\end{figure}
\bcprulessavespacefalse

%%% Local Variables:
%%% mode: latex
%%% TeX-master: t
%%% End:


\subsection{Operational Semantics}
\label{sec:semantics}
We now introduce the operational semantics for our language.
We assume a finite domain of heap addresses \textbf{Addr}:
we denote an arbitrary address with $\ottmv{a}$.
A runtime state is represented by a configuration $ \tuple{ \ottnt{H} ,  \ottnt{R} ,  \oldvec{F} ,  \ottnt{e} } $, which consists of a heap,
register file, stack, and currently reducing expression respectively.
The register file maps variables to runtime values $v$, which are
either integers $n$ or addresses $\ottmv{a}$. The heap maps a finite subset
of addresses to runtime values. The runtime stack represents pending
function calls as a sequence of return contexts, which we
describe below.
While the final configuration component is an expression,
the rewriting rules are defined in terms of $\ottnt{E}  \ottsym{[}  \ottnt{e}  \ottsym{]}$, which is
an evaluation context $\ottnt{E}$ and redex $\ottnt{e}$, as is standard.
The grammar for evaluation contexts is defined by: 
\(
    \ottnt{E}  ::=   \ottnt{E'} \SEQ \ottnt{e}  \mid  \HOLE .
  \)

Our operational semantics is given in \Cref{fig:transitionRules1,fig:transitionRules2}.
We write $ \DOM( \ottnt{H} ) $ to indicate the domain of
a function and $\ottnt{H}  \ottsym{\{}  \ottmv{a}  \mapsto  v  \ottsym{\}}$ where $ \ottmv{a}  \not\in   \DOM( \ottnt{H} )  $ to denote a map
which takes all values in $\DOM(H)$ to their values in $H$ and which
additionally takes $\ottmv{a}$ to $v$.
We will write $ \ottnt{H}  \ottsym{\{}  \ottmv{a}  \hookleftarrow  v  \ottsym{\}}$
where $ \ottmv{a}  \in   \DOM( \ottnt{H} )  $ to denote a map equivalent to $\ottnt{H}$ except that $\ottmv{a}$ takes value $v$.
We use similar notation for $ \DOM( \ottnt{R} ) $ and $\ottnt{R}  \ottsym{\{}  \mathit{x}  \mapsto  v  \ottsym{\}}$.
We also write $\emptyset$ for the empty register file and heap.
%
The step relation  $ \longrightarrow _{ \ottnt{D} } $ is parameterized
by a set of function definitions $\ottnt{D}$; a program $ \tuple{ \ottnt{D} ,  \ottnt{e} } $ is executed
by stepping the initial configuration $ \tuple{  \emptyset  ,   \emptyset  ,   \cdot  ,  \ottnt{e} } $ according
to $\longrightarrow_{\ottnt{D}}$.
The semantics is mostly standard; we highlight some important points below.

Return contexts $F$ take the form $\ottnt{E}  \ottsym{[}   \LET  \mathit{y}  =   \HOLE^ \ell   \IN  \ottnt{e}   \ottsym{]}$. A return context represents a pending
function call with label $\ell$, and indicates that $\mathit{y}$ should be bound to
the return value of the callee during the execution of $\ottnt{e}$ within the larger execution context $\ottnt{E}$.
The call stack $\oldvec{F}$ is a sequence of these contexts, with
the first such return context representing the most recent function call.
The stack grows at function calls as described by rule \rn{R-Call}.
For a call $\ottnt{E}  \ottsym{[}   \LET  \mathit{x}  =   \mathit{f} ^ \ell (  \mathit{y_{{\mathrm{1}}}} ,\ldots, \mathit{y_{\ottmv{n}}}  )   \IN  \ottnt{e}   \ottsym{]}$ where
$f$ is defined as $\ottsym{(}  \mathit{x_{{\mathrm{1}}}}  \ottsym{,} \, ... \, \ottsym{,}  \mathit{x_{\ottmv{n}}}  \ottsym{)}  \ottnt{e'}$, the return context $\ottnt{E}  \ottsym{[}   \LET  \mathit{y}  =   \HOLE^ \ell   \IN  \ottnt{e}   \ottsym{]}$ is
prepended onto the stack of the input configuration.
The substitution of formal arguments for parameters in $e'$, denoted by $   [  \mathit{y_{{\mathrm{1}}}}  /  \mathit{x_{{\mathrm{1}}}}  ]  \cdots  [  \mathit{y_{\ottmv{n}}}  /  \mathit{x_{\ottmv{n}}}  ]     \ottnt{e'} $,
becomes the currently reducing expression in the output configuration.
Function returns are handled by \rn{R-Var}.
Our semantics return values by name; when the currently executing function fully reduces to a single variable $x$,
$x$ is substituted into the return context on the top of the stack,
denoted by $\ottnt{E}  \ottsym{[}   \LET  \mathit{y}  =   \HOLE^ \ell   \IN  \ottnt{e}   \ottsym{]}[x]$.

In the rules \rn{R-Assert} we write $\models  \ottsym{[}  \ottnt{R}  \ottsym{]} \, \varphi$ to mean that the formula
yielded by substituting the concrete values in $\ottnt{R}$ for the variables in $\varphi$
is valid within some chosen logic (see \Cref{sec:types}); in \rn{R-AssertFail} we write $\not\models  \ottsym{[}  \ottnt{R}  \ottsym{]} \, \varphi$
when the formula is \emph{not} valid.
The substitution operation $\ottsym{[}  \ottnt{R}  \ottsym{]} \, \varphi$ is defined inductively as $\ottsym{[}   \emptyset   \ottsym{]} \, \varphi  \ottsym{=}  \varphi, \ottsym{[}  \ottnt{R}  \ottsym{\{}  \mathit{x}  \mapsto  n  \ottsym{\}}  \ottsym{]} \, \varphi  \ottsym{=}  \ottsym{[}  \ottnt{R}  \ottsym{]} \, \ottsym{[}  n  \ottsym{/}  \mathit{x}  \ottsym{]}  \varphi, \ottsym{[}  \ottnt{R}  \ottsym{\{}  \mathit{x}  \mapsto  \ottmv{a}  \ottsym{\}}  \ottsym{]} \, \varphi  \ottsym{=}  \ottsym{[}  \ottnt{R}  \ottsym{]} \, \varphi$.
In the case of an assertion failure, the semantics steps to a distinguished
configuration $ \mathbf{AssertFail} $. The goal of our type system is to show that no
execution of a well-typed program may reach this configuration.
The \imp{alias} form checks whether the two references actually alias;
i.e., if the must-alias assertion provided by the programmer is correct.
If not, our semantics steps to the distinguished $ \mathbf{AliasFail} $ configuration.
Our type system does \emph{not} guarantee that $ \mathbf{AliasFail} $ is unreachable;
aliasing assertions are effectively trusted annotations that are assumed to hold.

In order to avoid duplicate variable names in our register file due to recursive functions,
we refresh the bound variable $\mathit{x}$ in a let expression to $\mathit{x'}$.
Take expression $ \LET  \mathit{x}  =  \mathit{y}  \IN  \ottnt{e} $ as an example;
we substitute a fresh variable $\mathit{x'}$ for $\mathit{x}$ in $\ottnt{e}$, then bind $\mathit{x'}$ to the value
of variable $\mathit{y}$.
We assume this refreshing of variables preserves our assumption that all variable
bindings introduced with let and function parameters are unique, i.e. $\mathit{x'}$ does
not overlap with variable names that occur in the program.
% \JT{Do we want to mention this at all?}
% \AI{We may want to remove if we need more space.}
% Finally, although $\rn{R-MkRef}$ allocates memory, our language does not
% support memory reclamation.
% We assume a garbage collected environment and consider memory reclamation an orthogonal concern.
% \JT{Do we want
%   to say that we could have explicit free and use suenaga's
%   approach, but don't want to because it unnecessarily complicates
%   the system?}
  

%%% Local Variables:
%%% mode: latex
%%% TeX-master: t
%%% End:

%  LocalWords:  ownerships mkref redex
