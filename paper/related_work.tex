\section{Related Work}
\label{sec:rw}

The difficulty in handling programs with mutable references and
aliasing has been well-studied. Like JayHorn, many approaches model the heap explicitly
at verification time, approximating concrete heap locations with allocation site
labels \cite{kahsai2017quantified,kahsai2016jayhorn,chugh2012dependent,rondon2010low,fink2008effective};
each \emph{abstract location} is also associated with a refinement.
As abstract locations summarize many concrete locations, this approach
does not in general admit strong updates and flow-sensitivity; in particular, the refinement
associated with an abstract location is fixed for the lifetime of the program.
The techniques cited above include various workarounds for this limitation. For
example, \cite{rondon2010low,chugh2012dependent} temporarily allows
breaking these invariants through a distinguished program name
as long as the abstract location is not accessed through another name.
The programmer must therefore eventually bring the invariant back in sync with the summary location.
As a result, these systems ultimately cannot precisely handle programs that require
evolving invariants on mutable memory.\looseness=-1

A similar approach was taken in
CQual \cite{foster2002flow} by Aiken et al. \cite{aiken2003checking}.
They used an explicit \emph{restrict} binding for pointers.
Strong updates are permitted through pointers bound
with \emph{restrict}, but the program is forbidden from
using any pointers which share an allocation site while the restrict binding is live.

A related technique used in the field of object-oriented verification is
to declare object invariants at the class level and allow these
invariants on object fields to be broken during a limited period of time
\cite{barnett2011specification,flanagan2002extended}. In particular, the work on
Spec\# \cite{barnett2011specification} uses an ownership system which tracks
whether object $a$ owns object $b$; like \name's ownership system,
these ownerships contain the effects of mutation. However, Spec\#'s ownership
is quite strict and does not admit references to $b$ outside of the owning object $a$.
% These techniques implicitly use \emph{type-based} abstract heap; like the
% techniques above, any temporarily broken invariants on an abstract location
% (i.e., class invariant) must be eventually brought back in sync with the
% fixed invariant.

Viper \cite{heule2013verification,mueller2016viper}
(and its related projects \cite{heule2013abstract,leino2010deadlock}) uses
access annotations (expressed as permission predicates) to explicitly transfer access/mutation
permissions for references between static program names. Like \name,
permissions may be fractionally transferred, allowing temporary shared, immutable
access to a mutable memory cell. However,
while \name automatically infers many ownership transfers, Viper requires extensive
annotations for each transfer.

F*, a dependently typed dialect of ML, includes an update/select
theory of heaps and requires explicit annotations summarizing the heap
effects of a method
\cite{protzenko2017verified,swamy2016dependent,swamy2013verifying}.
This approach enables modular reasoning and precise specification of
pre- and post-conditions with respect to the heap, but precludes full automation.

The work on rely--guarantee reference types by Gordon et
al. \cite{gordon2013rely,gordon2017verifying} uses refinement
types in a language mutable references and aliasing.
Their approach extends reference types with rely/guarantee predicates;
the rely predicate describes possible mutations via aliases,
and the guarantee predicate describes the admissible mutations through the current reference.
If two references may alias, then the guarantee predicate of one
reference implies the rely predicate of the other and vice versa. 
This invariant is maintained with a splitting operation that
is similar to our $+$ operator.
Further, their type system allows strong updates to reference refinements
provided the new refinements are preserved by the rely predicate.
Thus, rely--guarantee refinement support multiple
mutable, aliased references with non-trivial refinement information. Unfortunately this
expressiveness comes at the cost of automated inference and verification; an embedding
of this system into Liquid Haskell \cite{vazou2014refinement} described in
\cite{gordon2017verifying} was forced to sacrifice strong updates.

Work by Degen et al. \cite{degen2007tracking} introduced linear \emph{state annotations} to Java.
To effect strong updates in the presence of aliasing, like \name,
their system requires annotated memory locations are mutated only through
a distinguished reference. Further, all aliases of this mutable reference give
no information about the state of the object much like our $0$ ownership pointers.
However, their system cannot handle multiple, immutable aliases with non-trivial
annotation information; \emph{only} the mutable reference may
have non-trivial annotation information.


The fractional ownerships in \name and their counterparts in
\cite{suenaga2009fractional,suenaga2012type} have a clear relation to
linear type systems. Many authors have explored the use of linear type
systems to reason in contexts with aliased mutable references
\cite{fahndrich2002adoption,deline2001enforcing,smith2000alias}, and
in particular with the goal of supporting strong updates
\cite{ahmed20073}. A closely related approach is RustHorn by
Matsushita et al. \cite{matsushita2020rusthorn}. Much like
\name, RustHorn uses CHC and linear aliasing information for
the sound and---unlike \name---complete verification of
programs with aliasing and mutability. However, their approach depends on Rust's
strict \emph{borrowing discipline}, and cannot handle programs
where multiple aliased references are used 
in the same lexical region. In contrast, \name supports
fine-grained, per-statement changes in mutability and even further
control with \imp{alias} annotations, which allows it to verify larger
classes of programs.

The ownerships of \name also have
a connection to separation logic \cite{reynolds2002separation};
the separating conjunction isolates write effects to local subheaps,
while \name's ownership system isolates effects to local updates of pointer types.
Other researchers have used separation logic to precisely
support strong updates of abstract state.
For example, in work by Kloos et al. \cite{kloos2015asynchronous}
resources are associated with static, abstract names; each
resource (represented by its static name) may be owned (and thus,
mutated) by exactly one thread. Unlike \name, their ownership
system forbids even temporary immutable,
shared ownership, or transferring ownerships at arbitrary program points.
An approach proposed by Bakst and Jhala \cite{bakst2016predicate}
uses a similar technique,
combining separation logic with refinement types. Their approach gives
allocated memory cells abstract names, and associates these names with
refinements in an abstract heap. Like the approach of Kloos et al.
and \name's ownership 1 pointers, they
ensure these abstract locations are distinct
in all concrete heaps, enabling sound, strong updates.

The idea of using a rational number to express permissions to access a
reference dates back to the type system of \emph{fractional
permissions} by Boyland~\cite{DBLP:conf/sas/Boyland03}.  His work used
fractional permissions to verify race freedom of a concurrent program
without a may-alias analysis.  Later,
Terauchi~\cite{terauchi2008checking} proposed a type-inference
algorithm that reduces typing constraints to a set of linear
inequalities over rational numbers.  Boyland's idea also inspired a
variant of separation logic for a concurrent programming
language~\cite{DBLP:conf/popl/BornatCOP05} to express sharing of read
permissions among several threads.  Our previous
work~\cite{suenaga2009fractional,suenaga2012type}, inspired by that
in \cite{terauchi2008checking,DBLP:conf/popl/BornatCOP05}, proposed methods for type-based
verification of resource-leak freedom, in which a rational number
expresses an \emph{obligation} to deallocate certain resource, not
just a permission.\looseness=-1

The issue of context-sensitivity (sometimes called \emph{polyvariance})
is well-studied in the field of abstract interpretation
(e.g., \cite{kashyap2014jsai,hardekopf2014widening,shivers1991control,smaragdakis2011pick,milanova2005parameterized}, see \cite{gilray2013survey}
for a recent survey).
Polyvariance has also been used in type systems to assign
different behaviors to the same function depending on its call site
\cite{banerjee1997modular,wells2002calculus,amtoft2000faithful}.
In the area of refinement type systems, Zhu and Jagannathan developed
a context-sensitive dependent type system for a functional language \cite{zhu2013compositional}
that indexed function types by unique labels attached to call-sites. Our
context-sensitivity approach was inspired by this work.
In fact, we could have formalized context-polymorphism within the framework of full dependent
types, but chose the current presentation for simplicity.\looseness=-1

%%% Local Variables:
%%% mode: latex
%%% TeX-master: t
%%% End:

%  LocalWords:  ownerships CQual Haskell Degen polyvariance Zhu
%  LocalWords:  Jagannathan
